\section{Giacomo Leopardi}
\section{La scapigliatura}
\section{Giovanni Verga}
\section{Giovanni Pascoli}
\section{Gabriele d'Annunzio}
\section{Luigi Pirandello}
\section{Italo Svevo}
\section{I crepuscolari}
\section{I futuristi}
\section{Giuseppe Ungaretti}
\section{Umberto Saba}
\section{Eugenio Montale}

\section{Carlo Emilio Gadda}

\subsection{La vita}
Carlo Emilio Gadda nacque a Milano nel 1983. 
Fin da giovane era attirato dagli studi letterari ma dovette dedicarsi all'ingegneria per via delle difficoltà economiche della famiglia. 
Nel 1915 venne arruolato nell'esercito per combattere la Prima Guerra Mondiale. 
In guerra morì suo fratello, questo evento scuoterà fortemente i rapporti familiari di Gadda fino a farlo approdare a una forte misoginia e infine alla misantropia. 

\subsection{Le opere}


\section{L'ermetismo}
%\section{Spazio e tempo}
%\subsection{in Leopardi}
%contro il progresso, pessimismo storico? 
%\subsection{in Verga}
%contro il progresso
%\subsection{nel futurismo}

%\section{La guerra}
%\subsection{in Leopardi}
%social catena, la ginestra
%\subsection{in Ungaretti}
%fratellanza tra tutti gli uomini
%\subsection{in Montale}
%la primavera Hitleriana, Clizia
%\subsection{in Svevo}
%fine de "La coscienza di Zeno"
%\subsection{nel futurismo}
%esaltazione della guerra
%\subsection{in d'Annunzio}
%esaltazione della guerra 

%\section{Scienza ed etica}
%\subsection{in Gadda}
%gnomero, impossibilità di districare la realtà

%\section{Democrazie e totalitarismi}
%\subsection{in Montale}
%la primavera Hitleriana, Clizia
%\subsection{in Gadda}
%parapagal, la cognizione del dolore

%\section{Ambiente e risorseeee}
%\subsection{in d'Annunzio}
%panismo estetizzante, c'era qualcun altro che parlava di panismo, forse montale? 

%\section{Salute e malattia}
%\subsection{nella scapigliatura}
%gusto dell'orrido
%\subsection{in Pirandello}
%la pazzia
%\subsection{in Svevo}
%la psicanalisi