\section{Arthur Schopenhauer (1788 - 1860)}
Schopenhauer fu capace di sintetizzare numerosi spunti filosofici e spirituali diversi, tra cui: 
\begin{itemize}
    \item la teoria delle idee di Platone; 
    \item l'impostazione soggettivistica della gnoseologia kantiana; 
    \item il filone materialistico illuminista;
    \item il tema del dolore romantico. 
\end{itemize}

\subsection{Il mondo come volontà e rappresentazione}
Il punto di partenza della filosofia di Schopenhauer è la distinzione kantiana tra \emph{fenomeno} e \emph{noumeno}, ovvero tra la cosa \say{come appare} e \say{in sé}. 
Secondo Schopenhauer, il fenomeno è solo un'illusione, che descrive con l'immagine orientale del \emph{velo di Maya}, mentre il noumeno è ciò che si nasconde dietro. 
Il fenomeno di cui parla Schopenhauer esiste solo in quanto \emph{rappresentazione} soggettiva all'interno di una coscienza. 
Il soggetto può conoscere il mondo per mezzo di tre forme a priori: 
\begin{itemize}
    \item spazio; 
    \item tempo; 
    \item causalità.
\end{itemize}
le quali sono come delle lenti attraverso cui si osservano le cose. 
Esiste però una realtà oltre il fenomeno su cui solo l'uomo, in quanto \say{animale metafisico}, si interroga. 

Vivendo il corpo dal suo interno possiamo giungere alla conclusione che l'essenza profonda del nostro io è la \emph{volontà} di vivere, che si manifesta esteriormente in tutti i nostri organi. 
Ma poiché vivendo il proprio corpo si smette di usare le forme a priori della conoscenza, l'esperienza della volontà non è più individuale ma risulta l'essenza della realtà stessa. 
La volontà che governa la realtà ha alcune caratteristiche: 
\begin{itemize}
    \item è inconscia;
    \item è unica;
    \item è eterna;
    \item è incausata;
    \item è senza scopo.
\end{itemize}

Essendo manifestazione di una volontà infinita, secondo Schopenhauer la vita è intrinsecamente dolore. 
Infatti, la volontà comporta desiderio che, per definizione, significa privazione e mancanza di qualcosa. 
Inoltre, il piacere che ogni tanto gli uomini provano non è altro che una cessazione temporanea del dolore che caratterizza la vita. 
Quando viene meno il dolore causato dal desiderio subentra la noia: secondo il filosofo quindi la vita è come un pendolo che oscilla incessantemente tra il dolore e la noia, passando attraverso l'intervallo fugace del piacere. 
Come ci si può liberare da questo dolore? 
Secondo Schopenhauer, fuggire la vita con il suicidio non è la risposta corretta perché: 
\begin{itemize}
    \item il suicidio è un atto di affermazione della volontà 
    \item il suicidio non scalfisce la volontà universale, ma solo una sua manifestazione fenomenica
\end{itemize}
Liberarsi dal dolore della vita significa quindi liberarsi dalla volontà stessa di vivere, trasformare in \emph{noluntas} la \emph{voluntas} attraverso l'arte, la morale e l'ascesi. 

L'arte, in quanto conoscenza libera e disinteressata delle idee, sottrae l'individuo ai desideri quotidiani e permette all'uomo di contemplare la vita, ovvero di iniziare quell'iter che permette di annullare la volontà di vivere. 
L'arte è però un conforto fugace, temporaneo e parziale. 
Per liberarsi davvero dal dolore della vita bisogna quindi intraprendere il sentiero della morale e dell'ascesi. 
La morale secondo Schopenhauer implica un impegno di compassione nei confronti del prossimo, ovvero avvertire come proprie le sofferenze degli altri. 
La morale si concretizza in: 
\begin{itemize} 
    \item giustizia, ovvero il non fare il male agli altri 
    \item carità, ovvero il fare del bene al prossimo 
\end{itemize} 
La morale però presuppone comunque un attaccamento alla vita, pertanto l'unico vero modo di terminare il dolore si identifica con l'ascesi. 
L'ascesi consiste nella rinuncia ai piaceri, nell'umiltà, nel digiuno, nella povertà, nel sacrificio e nell'automacerazione ed è l'unica possibilità di sopprimere la volontà di vivere. 

\section{Karl Marx (1818 - 1883)}
La filosofia di Marx si propose come un'analisi globale della società e della storia. 
Sarebbe infatti riduttivo marchiarlo unicamente come filosofo in quanto le sue idee influenzarono economia, filosofia, storia, teoria del diritto e dello Stato e oltre. 
Le influenze culturali alla base del marxismo furono:
\begin{itemize}
    \item la filosofia classica tedesca, da Hegel a Feuerbach;
    \item l'economia politica borghese, da Adam Smith a Ricardo;
    \item il pensiero socialista.
\end{itemize}

La filosofia di Marx prende un forte spunto da quella di Hegel, seppur muovendo sin da subito pesanti critiche. 
Da Hegel riprende infatti il processo dialettico con cui considera la storia, ma contesta in lui quello che è un \say{misticismo logico}. 
Secondo Marx infatti, Hegel avrebbe reso il concreto come la manifestazione dell'astratto dopo aver desunto l'astratto dal concreto. 
Oltre ad essere fallace dal punto di vista razionale, la filosofia di Hegel tende anche ad essere politicamente conservatrice perché rende manifestazioni necessarie dello Spirito quelli che sono in realtà dei semplici dati di fatto. 

Alla base dell'ideale di Marx è presente la critica della civiltà moderna e dello Stato liberale. 
Secondo Marx infatti, si è verificata una forte frattra tra la società civile e lo Stato. 
Quest'ultimo non è che uno strumento in mano alle classi dominanti per affermare il proprio potere e, proclamando l'uguaglianza \emph{formale} dei propri cittadini di fronte alla legge non fa altro che ratificare la loro disuguaglianza \emph{sostanziale}. 
Il sistema borghese sarebbe una società del \emph{bellum omnium contra omnes}, come aveva già affermato Hegel, in cui Marx identifica come caratteristiche fondamentali l'individualismo e l'atomismo. 
Quello che il filosofo auspica è una compenetrazione perfetta tra stato e società, in una democrazia \say{totale} che garantisca l'uguaglianza sostanziale eliminando il fondamento di ogni disuguaglianza: la proprietà privata dei mezzi di produzione. 

L'economia borghese, secondo Marx, ha il grande difetto di \say{eternizzare} il sistema capitalistico, considerandolo \emph{il} e non \emph{un} sistema economico. 
L'economia borghese inoltre non è in grado di scorgere le conflittualità che caratterizzano il capitalismo, le scissioni che Marx (riprendendo Feuerbach) chiama \say{alienazione}. 
Sempre riprendendo Feuerbach, secondo Marx l'alienazione è una condizione patologica di scissione, dipendenza ed estraniazione. 
Secondo Marx, l'operaio è alienato: 
\begin{itemize}
    \item \textbf{rispetto al prodotto} della sua attività, in quanto egli produce un oggetto che non gli appartiene; 
    \item \textbf{rispetto all'attività} stessa, in quanto viene vista come un lavoro forzato; 
    \item \textbf{rispetto alla propria essenza} di essere umano, che consiste nel lavoro libero, creativo e universale; 
    \item \textbf{rispetto al prossimo}, che egli vede come il capitalista sfruttatore. 
\end{itemize}
La causa dell'alienazione è la proprietà privata dei mezzi di produzione, che permettono che il capitalista prevarichi sugli operai. 
La disalienazione starebbe quindi nel superamento della proprietà privata dei mezzi di produzione con l'avvento del comunismo. 

Nella sua analisi storica, Marx innanzitutto riconosce l'esistenza delle \say{ideologie}, ovvero false rappresentazioni della realtà. 
Marx quindi identifica come chiave del processo storico un processo materiale (da cui il nome della sua filosofia, detta \say{materialismo storico}) fondato sulla dialettica bisogno-soddisfacimento. 
Alla base della storia ci sarebbe quindi il lavoro, creatore di civiltà e cultura nonché ciò che differenzia l'uomo dalle bestie. 

Le varie società storiche sono caratterizzate da certi \emph{modi di produzione}, a loro volta costituiti da \emph{forze produttive} (forza-lavoro, mezzi di produzione e conoscenze tecniche) e \emph{rapporti di produzione} che si instaurano tra gli uomini. 
L'insieme dei rapporti di produzione costituirebbe, secondo Marx, la \emph{struttura} di una data società sopra la quale viene costruita una \emph{sovrastruttura} di leggi, politica, etica, arte, religione e filosofia. 
Questi quindi non sono realtà a sé stanti ma espressioni dei rapporti economici di una certa società storica. 
Marx quindi svela che le vere forze motrici della Storia sono unicamente di natura economica. 

Per studiare il capitalismo, Marx analizza merce, lavoro e plusvalore. 
Una merce deve possedere un \say{valore d'uso} (ovvero un'utilità pratica) e un \say{valore di scambio} che le permetta di essere scambiata con altre merci. 
Il valore di scambio di una merce dovrebbe derivare dalla quantità di lavoro necessaria per produrre la merce. 
Marx contesta quindi il \say{feticismo delle merci}, che consiste nel considerare una merce come avente valore per sé. 

La caratteristica del capitalismo è la produzione finalizzata all'accumulo di denaro e non finalizzata al consumo (schema Denaro - Merce - Denaro). 
Ma da dove arriva questo plusvalore? 
Il capitalista compra la forza-lavoro come qualsiasi merce, ma l'operaio ha la capacità di produrre più di quanto gli sia corrisposto come salario. 
Da questo plusvalore derivera il profitto del capitalista. 
La società capitalista è quindi fondata sullo sfruttamento e sulla logica del profitto privato. 

Le contraddizioni della società borghese sono la base per teorizzare la \emph{rivoluzione proletaria}, per attuare il passaggio dal capitalismo al comunismo. 
Dopo aver abbattuto lo Stato borghese e socializzato i mezzi di produzione, avverrà necessariamente la \emph{dittatura del proletariato}, una fase di transizione per giungere al comunismo autentico. 
Il comunismo autentico infatti, si avrà solo quando l'uomo cesserà di intrattenere con il mondo rapporti di puro possesso e consumo, in una società per cui \say{ognuno secondo le sue capacità, a ognuno secondo i suoi bisogni}. 

\section{Friederich Nietzsche (1844 - 1900)}
Il pensiero di Nietzsche fu influenzato dalla lettura de \say{Il mondo come volontà e rappresentazione} di Schopenhauer. 
Il percorso di studi del filosofo fu inizialmente improntato alla filologia per poi rivolgersi alla filosofia. 
Nei suoi scritti Nietzsche attacca ferocemente la cultura occidentale, distruggendo tutte le certezze del passato e proponendo un nuovo tipo di umanità. 
Il suo stile di scrittura fu perlopiù aforistico e il suo pensiero asistematico. 

\subsection{Il periodo giovanile}

\subsubsection{La nascita della tragedia}
Il tema centrale di questo scritto, a metà tra la filologia e la filosofia, è la distinzione tra \emph{apollineo} e \emph{dionisiaco}. 
Con questi due termini Nietzsche intende indicare i due impulsi di base identificabili nella tragedia greca: 
\begin{itemize}
    \item l'\textbf{apollineo}, un atteggiamento di razionalità ed equilibrio; 
    \item il \textbf{dionisiaco}, un atteggiamento di irrazionalità e caos.
\end{itemize}
Secondo Nietzsche, la grande tragedia nasce dalla fusione perfetta di questi due elementi (per esempio in Eschilo e Sofocle). 
Questa sintesi viene meno quando inizia a prevalere l'apollineo, come nelle tragedie di Euripide. 
La prevaricazione dell'apollineo sul dionisiaco fu da attribuire all'insegnamento razionalistico di Socrate, in cui Nietzsche identifica l'inizio della decadenza della civiltà occidentale. 

\subsubsection{Le \emph{Considerazioni inattuali}}
Nella seconda delle quattro \emph{Considerazioni inattuali}, Nietzsche si schiera apertamente contro lo storicismo. 
Infatti nella vita è necessario il \say{fattore oblio} perché:
\begin{itemize}
    \item senza incoscienza non c'è felicità;
    \item per poter agire nel presente occorre saper dimenticare il passato.
\end{itemize}
Questo però non significa che la storia sia sempre nociva per la vita. 
Il filosofo infatti identifica tre modi di rapportarsi con la storia: 
\begin{itemize}
    \item la storia \textbf{monumentale}, tipica di chi guarda al passato alla ricerca di modelli;
    \item la storia \textbf{antiquaria}, tipica di chi guarda al passato con fedeltà e amore;
    \item la storia \textbf{critica}, tipica di chi guarda al passato come un peso di cui liberarsi.
\end{itemize}
Ognuno di questi tre tipi di storia può essere valido a patto che non venga utilizzato esclusivamente. 

\subsection{La filosofia del mattino}
In questo periodo, Nietzsche adotta un metodo critico e genealogico, ovvero eleva il sospetto a metodo di indagine e ricerca i processi all'origine di realtà etiche e metafisiche. 

Secondo Nietzsche, Dio è sostanzialmente: 
\begin{itemize}
    \item il simbolo di ogni trascendenza oltre a questo mondo; 
    \item la personificazione delle certezze dell'umanità. 
\end{itemize}
Dio e l'aldilà hanno sempre rappresentato una fuga dell'uomo di fronte alla vita. 
L'immagine di un Dio benevolo e di un cosmo ordinato sono solo costruzioni della mente per sopportare la durezza dell'esistenza. 
Le metafisiche e le religioni quindi non sono altro che menzogne millenarie da smascherare e distruggere. 
Nella \emph{Gaia scienza} il filosofo narra la \emph{morte di Dio} attraverso il racconto dell'\emph{uomo folle}, un profeta che va al mercato ad annunciare che Dio è morto e \say{siamo stati noi ad ucciderlo}. 
La morte di Dio provoca nell'uomo, non ancora pronto per questo annuncio, un senso di vertigine e smarrimento che può sopportare solo facendosi \emph{superuomo}. 
Infatti, solo chi ha il coraggio di guardare in faccia la realtà e il crollo delle certezze può compiere quel salto che separa l'uomo dal superuomo. 

\subsection{La filosofia del meriggio}
Nell'opera \emph{Così parlò Zarathustra}, il filosofo affronta tre temi principali:
\begin{itemize}
    \item il \textbf{superuomo};
    \item la \textbf{volontà di potenza}; 
    \item l'\textbf{eterno ritorno}.
\end{itemize}

\subsubsection{Il superuomo}
Il superuomo (\emph{Übermensch}) è colui che è in grado di accettare la dimensione tragica e dionisiaca dell'esistenza. 
È da notare che \emph{Übermensch} può essere tradotto anche con \say{oltreuomo}, indicando non un \say{supereroe}, un uomo potenziato, ma un uomo che supera i limiti dell'umanità. 
Nel discorso delle tre metamorfosi, Zarathustra elenca le tre metamorfosi a cui lo spirito deve sottoporsi per diventare oltreuomo:
\begin{itemize}
    \item il \textbf{cammello}, che sopporta i pesi della tradizione;
    \item il \textbf{leone}, che si libera dai fardelli metafisici ed etici;
    \item il \textbf{fanciullo}, che rappresenta l'oltreuomo, dionisiaco, un vero spirito libero.
\end{itemize}

\subsubsection{La volontà di potenza}
Secondo Nietzsche, la volontà di potenza si identifica con la vita stessa, è la spinta dell'autoaffermazione. 
È una forza creativa che può anche manifestarsi come sopraffazione e dominio. 

\subsubsection{L'eterno ritorno}
Il superuomo deve infine accettare l'eterno ritorno all'uguale, ovvero vivere la propria vita uguale per l'eternità. 

\subsection{La filosofia del crepuscolo}
Nelle sue ultime opere, Nietzsche critica la morale e il cristianesimo. 
La moralità infatti non è altro che \say{l'istino del gregge nel singolo}, ovvero la tendenza dell'uomo ad assoggettarsi a determinate pratiche sociali. 
Quella che anticamente era una morale \say{dei signori} successivamente, con l'avvento dell'ebraismo e del cristianesimo, viene ribaltata in una morale \say{degli schiavi}, che consiste in un risentimento contro la vita. 

Nietzsche tenta di superare il problema del nichilismo, da lui identificato come quella situazione di sgomento e nulla di fronte alla morte di Dio. 
Il nichilismo può essere suddiviso in:
\begin{itemize}
    \item \textbf{nichilismo incompleto}, in cui vengono distrutti i vecchi valori e ne vengono creati nuovi uguali ai precedenti;
    \item \textbf{nichilismo completo}, che si impegna a distruggere ogni rimasuglio di credenza rimasto. 
\end{itemize}

Nell'ultimo periodo, Nietzsche radicalizzò notevolmente il suo prospettivismo secondo cui non esistono cose o fatti, ma solo interpretazioni. 
Pertanto, il filosofo si schiera contro la scienza moderna, che tenta di dare un'interpretazione unica e meccanicistica a ciò che è in realtà libero e plurale. 

\section{Sigmund Freud (1856 -1939)}
Lavorando come psichiatra a fianco di Breuer (un dottore che sperimentava con tecniche ipnotiche per trattare i pazienti affetti da isteria), Freud ipotizzò che la maggior parte della vita mentale si svolgesse al di fuori della coscienza. 
Nella prima topica psicologicha (ovvero lo studio dei luoghi della psiche), Freud identifica tre \say{sistemi}: 
\begin{itemize}
    \item \textbf{conscio};
    \item \textbf{preconscio}, costituito da ricordi sopiti che possono essere facilmente richiamati alla memoria;
    \item \textbf{inconscio}, ovvero tutto ciò che è stato rimosso e può riemergere solo attraverso speciali tecniche. 
\end{itemize}
Nella seconda topica psicologica invece identifica tre \say{istanze}: 
\begin{itemize}
    \item l'\textbf{Es}, ovvero la forza impersonale e caotica delle pulsioni, che obbedisce solo al \say{principio di piacere};
    \item il \textbf{Super-io}, ovvero tutti i divieti instillati nell'individuo dall'esterno;
    \item l'\textbf{Io}, la parte organizzata della personalità che deve controllare le altre due istanze.
\end{itemize}
Un Io incapace di governare Super-io ed Es porta alla nevrosi. 

Nell'opera \emph{L'interpretazione dei sogni}, Freud identifica nel sogno una via per conoscere l'inconscio. 
Secondo lui infatti, i fenomeni onirici consisterebbero in un appagamento camuffato di un desiderio rimosso. 
All'interno dei sogni si possono infatti individuare un \say{contenuto manifesto} (ciò che vive il soggetto nel sogno) e un \say{contenuto latente} (l'insieme delle tendenze che danno luogo al sogno).
Un altra cosa che Freud prende in esame sono gli \say{atti mancati}, tutta quella serie di dimenticanze quotidiane che secondo lo psicanalista sono manifestazioni dell'inconscio. 

Freud elaborò anche una nuova teoria della sessualità in grado di spiegare atti come la sessualità infantile, la sublimazione e la perversione (ovvero la ricerca del piacere indipendentemente dal fine riproduttivo). 
Egli inannzitutto amplia il concetto di sessualità definendolo un'energia che può essere diretta verso le mete più diverse, ovvero la libido. 
Secondo Freud, il bambino è \say{un essere perverso e polimorfo}, che sviluppa la propria sessualità in tre fasi: 
\begin{itemize}
    \item \textbf{fase orale} (fino a 1.5 anni), la zona erogena è la bocca e l'attività principale il poppare;
    \item \textbf{fase anale} (da 1.5 a 3 anni), la zona erogena è l'ano ed è collegata all'escrezione;
    \item \textbf{fase genitale} (dai 3 anni in poi) in cui la zona erogena sono i genitali ed è a sua volta articolata in  
            \begin{itemize}
            \item \textbf{fase fallica}, in cui il bambino e la bambina scoprono il pene e soffrono di un complesso di castrazione; 
            \item \textbf{fase genitale in senso stretto}, in cui le pulsioni sessuali sono organizzate con il primato della zona genitale.
        \end{itemize}
\end{itemize}
Durante la fase fallica si sviluppa inoltre il complesso edipico, che consiste in un \say{attaccamento libidico verso il genitore di sesso opposto e atteggiamento ambivalente verso il genitore di egual sesso}. 
La risoluzione di tale complesso determina la futura personalità dell'infante. 

Secondo Freud, l'arte è analoga alla produzione onirica: anche questa è una manifestazione di desideri insoddisfatti. 
Il soddisfacimento di questo desiderio proibito avviene attraverso la \emph{sublimazione}, ovvero il trasferimento di pulsioni sessuali su oggetti non sessuali. 
L'artista è colui che è capace di sublimare questi desideri proibiti in forme socialmente accettabili. 
L'arte quindi funge come una sorta di terapia sia per l'artista che per lo spettatore. 

\section{Karl Popper (1902 - 1994)}
Popper nella sua filosofia combinò elementi neopositivistici e anti-neopositivistici per giungere a una teoria epistemologica completamente originale. 
I problemi affrontati da Popper furono quelli della demarcazione tra scienza e pseudoscienza e della certezza del sapere scientifico. 
Il pensiero di Popper può inoltre essere visto come una diretta conseguenza della rivoluzione scientifica operata da Einstein. 
Da Einstein infatti trae i principi di fondo della propria epistemologia: il \emph{falsificazionismo} e il \emph{fallibilismo}. 

In primo luogo, Popper procede a una riabilitazione della filosofia, ribadendone la necessità. 
Infatti esistono problemi di natura strettamente filosofica e, in un modo o nell'altro, la filosofia ha sempre a che fare con la conoscenza della realtà. 

Per quanto riguarda la demarcazione tra scienza e non-scienza, Popper riconosce che il verificazionismo è utopico: per verificare davvero una legge scientifica bisognerebbe avere presenti tutti i casi e questo non è possibile (quindi non è possibile il metodo induttivo). 
Di conseguenra, Popper propone il principio della \emph{falsificabilità}, per cui una teoria è scientifica solo quando questa può venir smentita dall'esperienza. 
La scienza si fonda quindi su un certo numero di asserzioni-base (su cui la comunità scientifica concorda) che possono essere sempre messe in discussione: da qui Popper deriva l'immagine della scienza \say{come un edificio costruito su palafitte} e non più come qualcosa di immutabile e perfetto. 
Popper inoltre riconosce l'asimmetria presente tra verificabilità e falsificabilità: miliardi di conferme non rendono certa una teoria ma una sola confutazione la rende invalida. 
Tuttavia, sebbene le ipotesi non possano venir verificate, possono essere \emph{corroborate} quando superano un'esperienza possibilmente falsificante. 
La corroborazione non può fungere da criterio di giustificazione delle teorie ma può servire come criterio di scelta tra teorie rivali. 

Secondo quanto detto prima, la metafisica non è una scienza poiché non è falsificabile. 
Questo però non significa che essa non abbia un senso: 
\begin{itemize}
    \item le scoperte scientifiche sono spesso spinte da credenze metafisiche; 
    \item le dottrine metafisiche possono comunque essere razionalmente discutibili. 
\end{itemize}
Popper si scaglia invece contro il marxismo e la psicanalisi freudiana perché prive di sufficiente falsificabilità e dirette ad aggirare qualsiasi smentita. 

Secondo Popper, non esiste alcun metodo per scoprire una teoria scientifica, in quanto queste sono frutto di congetture audaci ed intuizioni creative. 
Il \say{metodo scientifico} proposto da Popper sarebbe un processo di \emph{trial and error}, prova ed errore che va sottoposto al vaglio dell'esperienza. 
L'errore è quindi parte integrante del sapere scientifico ed ha un'importante funzione di crescita. 
Il modo in cui si evolve la scienza sarebbe quindi analogo all'evoluzione biologica teorizzata da Darwin. 

Il filosofo rifiuta inoltre l'\emph{osservazionismo} secondo cui lo scienziato osserva la natura senza alcuna ipotesi precostituita. 
Popper propone infatti l'immagine della mente come un faro, che a seconda delle ipotesi preconcette illumina la realtà con luce diversa. 
L'osservazione quindi non può mai essere completamente distaccata ma viene sempre eseguita con delle ipotesi e aspettative a monte. 
Popper rifiuta inoltre il \emph{fondazionalismo} e il \emph{giustificazionismo} del sapere, per cui la scienza avrebbe basi certe che la filosofia deve giustificare, affermando che:
\begin{itemize}
    \item il nostro sapere è strutturalmente incerto;
    \item la scienza è intrinsecamente fallibile e autocorreggibile;
    \item il problema di come possiamo giustificare la nostra conoscenza è privo di senso;
    \item l'uomo non potrà mai possedere la verità ma solo ricercarla senza conclusione (cfr. Seneca).
\end{itemize}
Lo scopo della scienza quindi non può essere la verità ma il raggiungimento di sempre maggiore verosimiglianza. 
È quindi necessario stabilire un criterio di preferenza tra teorie. 
È innnanzitutto ovvio che teorie scientifiche siano preferibili a teorie non scientifiche perché le prime possono essere controllate empiricamente. 
Tra teorie scientifiche invece la decisione dev'essere frutto di una discussione critica che tenga conto delle ipotesi in gioco. 
Per poter affermare questo, Popper stabilisce inoltre che teorie scientifiche che rispondano allo stesso problema possono essere confrontate. 
Popper approda quindi ad un'epistemologia evoluzionistica in cui le teorie migliori sopravvivono. 

Secondo Popper l'indeterminismo è un requisito necessario per ogni dottrina della libertà. 

Il pensiero di Popper si estese anche alla sfera politica. 
Nei suoi scritti politici, il filosofo difende la \say{società aperta} e critica ogni forma di assolutismo. 
Per prima cosa, Popper si schierra contro lo \say{storicismo}, ovvero quelle dottrine filosofiche con la pretesa di identificare un senso globale della storia (cfr Hegel, Marx). 
Non esiste un senso della storia precostituito perché gli uomini possono attribuirle ogni significato. 
Un altro errore dello storicismo è quello di fare confusione tra leggi e tendenze: per poter fare previsioni scientifiche bisogna basarsi sulle leggi. 
Secondo Popper quindi, nello storicismo vi sarebbero pretese totalitarie che produrrebbero solo sofferenza agli uomini. 
La società aperta è quella società fondata sulla salvaguardia delle libertà individuali attraverso istituzioni democratiche autocorreggibili. 
Una democrazia, secondo Popper, non è solo quello che viene tradizionalmente identificato come il \say{potere del popolo}, ma è quel sistema di governo in cui i governati hanno la possibilità di controllare i governanti senza ricorrere alla violenza. 
Il rifiuto della violenza è infatti categorico, con l'unica eccezione del ribaltamento di una tirannide. 
Il filosofo inoltre critica l'atteggiamento rivoluzionario esaltando il metodo riformista, superiore perché:
\begin{itemize}
    \item evita di promettere \say{paradisi};
    \item non pone fini assoluti che giustifichino qualsiasi mezzo;
    \item procede per via sperimentale;
    \item riesce a dominare meglio i mutamenti sociali;
    \item è in grado di salvaguardare la libertà.
\end{itemize}
Per questo, secondo Popper l'unico valore da conservare è il metodo della libertà e della democrazia, ovvero l'equivalente politico del metodo della scienza. 

\section{Carl Schmitt (1888 - 1985)}
La meditazione di Schmitt fu incentrata sulla politica. 
In particolare, in \emph{Teologia politica}, afferma che la sovranità non risiede nella norma bensì nella decisione (da cui \emph{decisionismo}) che la pone in essere. 
Tale decisione secondo il filosofo avviene in uno stato d'eccezione. 
Ne \emph{Il concetto di politico}, Schmitt tenta di chiarire l'essenza della politica, ormai divenuto necessario poiché Stato e società si compenetrano a vicenda. 
La risposta a cui giunge è che la politica è responsabile di determinare la coppia antitetica \emph{amico-nemico} su cui si fonda l'identità dello Stato. 
Per questa definizione, la politica è intrinsecamente conflittuale. 
La guerra è quindi una possibilità umana sempre presente. 
Schmitt evidenzia inoltre i limiti del parlamentarismo e del liberalismo, auspicando uno stato \say{totale} come quello nazista. 

Schmitt evidenzia inoltre che la società occidentale si sia sempre organizzata intorno a determinati \emph{centri di riferimento}, che condizionano la vita politica. 
Storicamente questi sono stati:
\begin{itemize}
    \item il teologico;
    \item il metafisico-scientifico;
    \item il morale-umanitario;
    \item l'economico;
    \item il tecnologico.
\end{itemize}

Nello \emph{Ius Publicum Europaeum}, Schmitt denuncia la crisi del diritto pubblico europeo, realizzatasi a partire dalla nascita della Società delle Nazioni, un'istituzione universale che si propone di abolire la guerra in tutto il mondo. 
Quest'organizzazione muterebbe il significato della guerra che, da un modo di relazionarsi tra Stato e Stato, diventerebbe un crimine contro l'umanità da bandire in modo assoluto. 

\section{Hannah Arendt (1906 - 1975)}
Nata da famiglia ebrea, Hannah Arendt fu costretta a fuggire dalla Germania dopo l'avvento del nazismo, rifugiandosi prima in Francia e poi negli Stati Uniti. 
Lei stessa si è sempre definita una \say{pensatrice politica}, autrice di opere come \emph{Le origini del totalitarismo}, \emph{Vita activa} e \emph{La banalità del male}. 

\subsection{Le origini del totalitarismo}
In quest'opera, Arendt analizza le cause e il funzionamento dei regimi totalitari, considerati una conseguenza diretta della società di massa. 
Il totaslitarismo è quel regime in cui:
\begin{itemize}
    \item tutto appare politico;
    \item tutto diventa pubblico;
    \item tutto è riferito a una legge superiore;
    \item viene dato enorme valore all'azione (per tenere le masse mobilitate);
    \item regna il discorso;
    \item si propone di creare un'umanità \say{nuova}.
\end{itemize}
Tuttavia questa apparenza va smascherata perché se non esiste alcun confine tra il politico il pubblico e il privato, tra il politico e il non politico, la politica scompare. 
Secondo la filosofa, è possibile identificare alcuni momenti significativi di vita politica, primo tra tutti la \emph{polis} greca antica. 
Qui infatti gli individui si riconoscono come uguali attraverso la discussione e la deliberazione comuni. 
Ne \emph{Le origini del totalitarimo}, Arendt inoltre riprende e amplia la riflessione di Montesquieu ne \emph{Lo spirito delle leggi}, identificando le caratteristiche fondamentali di ogni sistema di governo. 
\begin{center}
\begin{tabularx}{1.0 \textwidth}{ 
    >{\raggedright\arraybackslash}X 
  | >{\raggedright\arraybackslash}X 
  | >{\raggedright\arraybackslash}X 
  | >{\raggedright\arraybackslash}X }
    &\textbf{natura} &\textbf{principio d'azione} &\textbf{esperienza fondamentale}\\
    \hline
    \textbf{repubblica} &sovranità popolare &virtù &uguaglianza per nascita\\
    \hline
    \textbf{monarchia} &governo di uno solo subordinato a leggi &onore &disuguaglianza per nascita\\
    \hline
    \textbf{dispotismo} &governo assoluto di uno solo &paura &angoscia di fronte all'isolamento\\
    \hline
    \textbf{totadlitarismo} &terrore &ideologia &desolazione
\end{tabularx}
\end{center}
La novità rispetto all'antico dispotismo è che il totalitarismo distrugge anche la vita privata delle persone, estraniandole dalla società e rendendole nemiche tra loro. 
Il totalitarismo accentua quindi l'isolamento tipico degli uomini nella società di massa. 

\subsection{Vita activa}
In quest'opera, Arendt vuole insegnare a riconoscere e tutelare la \emph{res publica} e la politica. 
L'oggetto del saggio è la vita attiva, contrapposta alla vita contemplativa. 
Secondo Arendt, la \emph{vita activa} si articola in tre forme fondamentali: 
\begin{itemize}
    \item il \textbf{lavoro} (labour, tipico dell'\emph{animal laborans}), energia che si sprigiona e viene subito consumata; 
    \item l'\textbf{operare} (work, tipico dell'\emph{homo faber}), che tende a produrre trasformazioni durature;
    \item l'\textbf{agire} (action, tipico dello /emph{zóon politikón}), ciò che mette in relazione gli umani.
\end{itemize}
Altre due esperienze tipiche della condizione umana sono la natalità e la mortalità, entrambe strettamente collegate alla sfera dell'azione. 

L'azione ha alcune caratteristiche proprie:
\begin{itemize}
    \item è inizio, novità;
    \item rivela il "chi è" di una persona;
    \item è diversa dal comportamento abituale.
\end{itemize}
Essa presenta inoltre alcuni rischi:
\begin{itemize}
    \item ha effetti incontrollabili;
    \item non può essere compresa se non quando è totalmente compiuta;
    \item l'azione è rischio, necessita di coraggio di fronte all'ignoto.
\end{itemize}

Arendt identifica in Platone e Aristotele gli iniziatori di quel processo che porterà alla svalutazione della vita attiva a favore di quella contemplativa. 
Questa negazione si è affermata poi con il cristianesimo, con il quale l'agire politico è diventato impossibile. 

Alla fine emergono due modi radicalmente diversi di concepire la politica: 
\begin{itemize}
    \item come dominio di qualcuno sugli altri che richiede violenza;
    \item come organizzazione del potere da parte di uomini parlanti e agenti.
\end{itemize}
Secondo Arendt, il potere non dipende dal possesso di mezzi violenti, anzi, se si ricorre alla violenza non si ha o si è già perso il potere.