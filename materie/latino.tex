\section{Seneca (4 a.C. - 65 d.C.)}

\subsection{La vita}
Lucio Anneo Seneca nacque a Cordova nel 4 a.C. 
Nella sua vita fu filosofo e letterato, ma anche un politico scaltro e spregiudicato.
Fu educato a Roma e nel 31 iniziò il suo cursus honorum diventando questore e successivamente senatore. 
La sua abilità gli attirò la gelosia di Caligola, che considerò di farlo uccidere, ma fu salvato dall'intervento di un'amica del principe.
Successivamente, Messalina, moglie del nuovo imperatore Claudio, lo fece esiliare in Corsica nel 41 con l'accusa di adulterio.
Per riconciliarsi con Claudio, durante l'esilio scriverà la \emph{Consolatio ad Polybium}, liberto dell'imperatore.
La sorte di Seneca cambiò nuovamente nel 49 quando Agrippina Minore, la nuova moglie di Claudio, convinse l'imperatore a richiamare a Roma il filosofo.
Così, Seneca fu nominato precettore di Nerone. 
Nel 54, alla morte di Claudio, Seneca scrisse una violenta opera satirica contro Claudio, l'\emph{Apokolokỳntosis}.
Per qualche anno dopo la salita al trono di Nerone, il governo era in mano ad Agrippina, Seneca ed Afranio Burro (il prefetto del pretorio), che governarono con prudenza e moderazione.
Questo periodo felice durò fino al 59, quando Nerone ordinò l'assassionio della madre Agrippina.
Il matricidio segnò un profondo cambiamento nei rapporti tra l'imperatore e il suo precettore, che iniziò ad essere emarginato assieme ad Afranio Burro.
Quest'ultimo morì in circostanze sospette nel 62 e, lo stesso anno, Seneca decise di ritirarsi a vita privata per dedicarsi ai suoi studi.
La vita di Seneca fu stroncata nel 65 quando, dopo la scoperta della \emph{congiura di Pisone}, Seneca fu accusato di complicità e fu costretto a suicidarsi.
Il raccontro della sua morte è narrato da Tacito negli \emph{Annales}.

\subsection{La filosofia}
Prima di parlare delle opere di Seneca, è opportuno evidenziare l'importanza del suo pensiero filosofico.
La sua grande originalità è dovuta al fatto che, nonostante sia evidente l'impianto stoico della sua filosofia, egli fu un pensatore \emph{asistematico}, capace di includere elementi di altre filosofie.
La filosofia per Seneca è una guida all'azione, un percorso di miglioramento continuo: egli stesso è ben cosciente di non essere il saggio che descrive nelle sue opere.
L'obiettivo del suo pensiero è giungere ad essere \emph{sapiens}, ovvero un uomo che non si lascia turbare dalle circostanze e non si abbandona alle passioni, vivendo con moderazione. 
Infatti, così come la natura è governata dal \emph{Lògos}, la mente divina che ordina l'universo, anche l'uomo deve vivere seguendo la stessa razionalità.
Un altro tema centrale nella riflessione di Seneca è l'importanza del \emph{tempo}: l'unico bene di cui disponiamo davvero e che quindi non può assolutamente venire sprecato. 

\subsection{Le opere}
Le opere di Seneca abbracciano numerosi generi e argomenti: dieci dei suoi scritti filosofici furono raccolti nei \emph{Dialogi}, a parte ci sono opere come il \emph{De beneficiis}, il \emph{De clementia} e, soprattutto, le \emph{Epistulae morales ad Lucilium}.
Seneca scrisse inoltre otto tragedie, tutte \emph{cothurnatae}. 
Sotto suo nome sono tràdite anche due tragedie considerate apocrife. 
Un altro opuscolo importante è, come già detto, l'\emph{Apokolokỳntosis}, una parodia satirica della divinizzazione di Claudio.
\subsubsection{Le \emph{Consolationes}}
All'interno dei Dialogi sono presenti tre \emph{Consolationes}, rivolte rispettivamente a Marcia per la perdita del figlio, alla propria madre Elvia per l'esilio, e al liberto Polibio.
Questi testi risalgono al periodo di esilio in Corsica, e sono a metà strada tra la retorica e l'ammonimento morale, pertanto risultano piene di luoghi comuni sulla sventura e sulla necessità di affrontarla con coraggio.
\subsubsection{Il controllo delle passioni}
Sempre nei \emph{Dialogi}, sono presenti le opere \emph{De ira}, \emph{De constantia sapientis}, \emph{De tranquillitate animi} e \emph{De vita beata}.
In queste opere Seneca affronta i temi dell'ira, che altro non è che una \emph{pazzia momentanea}, e della sapienza, intesa come capacità di non cedere all'ignoranza e agli impulsi. 
Inoltre, Seneca prescrive alcuni esercizi spirituali per avvicinarsi alla saggezza: frequentare uomini buoni, impegnarsi per il bene comune, praticare la moderazione e attendere serenamente la morte.
\subsubsection{De providentia}
Nel \emph{De providentia}, Seneca affronta il problema della sventura: perché gli uomini buoni possono essere afflitti da terribili sciagure?
La risposta è la seguente: le persone, attraverso le tragedie, possono dare prova delle proprie virtù e, se le sofferenze risultassero insopportabili, è possibile darsi la morte volontariamente.
\subsubsection{De brevitate vitae}
Come già detto, il tema del tempo è fondamentale nel pensiero di Seneca. 
Al contrario di ciò che suggerisce il titolo, Seneca giunge alla conclusione che la vita non è breve, ma è l'uomo a sprecarla in attività inutili, particolarmente gli \emph{occupati}.
Il senso della vita quindi non sta nella quantità, ma nella qualità del tempo che si vive.
Per questo, Seneca invita ad utilizzare con cura il proprio tempo e a non farsi dominare da esso.
\subsubsection{Gli scritti politici}
Seneca scrisse anche alcune opere di carattere politico, come il \emph{De clementia}, dedicato a Nerone, in cui tratta del buon governo, all'insegna della moderazione e della giustizia. 
Il \emph{princeps} inotre dovrebbe garantire la pace e la prosperità, ispirato dalla ragione.
Un altro scritto con un fine politico è il \emph{De otio}, in cui Seneca elogia la vita intellettuale come utile al miglioramento morale: aiutando se stesso, il sapiente aiuta indirettamente il resto dell'umanità.
\subsubsection{Le \emph{Epistulae morales ad Lucilium}}
L'ultima opera di Seneca, scritta tra il 62 e il 65, consiste in 124 lettere indirizzate all'amico Lucilio.
Le lettere che Seneca scrive sono però scritte soprattutto per se stesso e per i posteri: in queste è infatti possibile identificare un vero e proprio testamento spirituale del filosofo.
Partendo da episodi comuni e quotidiani, Seneca affronta svariate riflessioni filosofiche in cui è presete la sintesi di tutto il suo pensiero con parole semplici e dirette. 
Qui il filosofo ribadisce che il percorso morale e intellettuale di un uomo non ha termine, ma è un continuo cammino. 
Nelle lettere torna il tema della felicità, che coincide con la ragione perfetta, e il tema del tempo e della morte: dare un senso alla morte infatti significa dare un senso alla vita stessa. 
\subsubsection{Le tragedie}
Eventi sanguinosi e incredibilmente violenti raccontati con toni estremamente crudi sono il fulcro delle tragedie di Seneca, in cui si assiste a un ribaltamento totale dei valori stoici che il filosofo predica.
Di fatto, queste tragedie ritraggono l'opposto del sapiente senecano, dominato dai propri impulsi e senza freni morali. 
Un personaggio tipico di queste tragedie è il tiranno, un concentrato di qualsiasi vizio ed infamia, che doveva fungere da monito a Nerone.
Il modello da cui Seneca attinge è Euripide.
\subsubsection{L'\emph{Apokolokỳntosis}}
L'\emph{Apokolokỳntosis}, o la \emph{Zucchificazione di Claudio}, è una satira menippea scritta nel 54, subito dopo la morte di Claudio. 
In quest'opera, l'imperatore viene deriso per i propri tic e deformità. 
È quindi un'opera diffamatoria, scritta però con il favore della corte e, probabilmente, fu sia una vendetta contro l'imperatore, sia una giustificazione di Agrippina e Nerone, gli assassini di Claudio.

\subsection{Lo stile}
Lo stile di Seneca è stato definito \emph{post-classico}, per la sua distanza dall'equilibrio di altri autori come Cicerone.
Il filosofo utilizza infatti una prosa nervosa e movimentata, con frasi brevi e paratattiche, piena di \emph{variatio}.
Per convogliare meglio i propri pensieri, Seneca utilizza spesso un effetti patetici, personificazioni, domande retoriche e immagini.

\section{Lucano (39 - 65 d.C.)}

\subsection{La vita}
Marco Anneo Lucano nacque a Cordova nel 39 d.C.
Fu educato a Roma e ad Atene e, grazie al potere dello zio Seneca, entrò presto nella stretta cerchia di Nerone.
Nel 65 d.C., Lucano, spinto dai suoi ideali repubblicani, prese parte alla congiura di Pisone e fu quindi costretto a suicidarsi a meno di 26 anni.

\subsection{Il \emph{Bellum civile}}
L'unica opera (incompiuta) a noi pervenuta di Lucano è il \emph{Bellum civile}, un poema epico-storico che narra la guerra civile tra Cesare e Pompeo.
Le fonti a cui attinse per la propria opera furono probabilmente le opere di Tito Livio, Seneca il Retore e Asinio Pollione, ma nonostante ciò il poema non pretende di avere un'attendibilità storica.

Lucano, con il \emph{Bellum civile}, si distacca radicalmente dalla tradizione virgiliana, in particolare:
\begin{itemize}
    \item rifiuta ogni giustificazione provvidenziale della storia;
    \item non è presente alcun personaggio completamente positivo;
    \item gli dei non intervengono, sono indifferenti alle vicende umane.
\end{itemize}
L'elemento fondamentale del poema è la tragicità degli eventi narrati, guerre \emph{plus quam civilia}, poiché i due contendenti sono parte della stessa famiglia.
Lucano vuole denunciare il disordine del mondo, da lui identificato nel crollo delle istituzioni e della legalità repubblicane, a favore delle forze distruttive di Cesare. 

I protagonisti del poema sono Cesare e Pompeo, antitetici ma accomunati dalla loro volontà di instaurare un dominio assoluto a Roma.
Cesare è presentato come l'incarnazione del male, spinto unicamente da sete di potere. 
Capovolgendo Enea, la caratteristica distintiva di Cesare è la sua empietà, raccontata nel banchetto dopo la battaglia di Farsàlo.
Pompeo invece è il difensore delle istituzioni repubblicane, ma viene presentato come debole e incapace di opporsi a Cesare.
Dopo la morte di Pompeo, Catone il Giovane prende la guida dell'esercito repubblicano. 
Egli rappresenta i valori che hanno reso grande la Roma repubblicana, è un esempio di saggio stoico. 
Catone stesso riconosce la malvagità del fato e l'indifferenza degli dèi e si impegna per tentare di contrastarli.

\section{Petronio (27 - 66 d.C.)}

\subsection{La vita}
Petronio nacque intorno al 27 d.C. e morì nel 66 d.C., in seguito alla congiura di Pisone. 
L'unica opera che scrisse fu il romanzo \emph{Satyricon}.
Negli \emph{Annales}, Tacito dipinge un ritratto a doppia faccia di Petronio: \say{dedicava le ore del giorno al sonno, quelle della notte ai suoi doveri e alle gioie della vita}.
Nonostante fosse un uomo ozioso, era anche estremamente raffinato, tanto da essere considerato \emph{arbiter elegantiae} nella corte di Nerone. 
Petronio fu nominato anche proconsole della Bitinia e governò con grande abilità e saggezza. 
Tuttavia, questa vita condotta nell'ambiguità a cavallo tra vizi e virtù si scontrò con la gelosia di Tigellino, prefetto del pretorio, che lo accusò di aver preso parte alla congiura di Pisone. 
Petronio fu quindi costretto a suicidarsi, ma lo fece con la stessa stravaganza con cui aveva vissuto: 
\say{non si tolse la vita con precipitazione, ma, secondo il suo capriccio, si fece tagliare le vene, poi richiudere, poi aprire di nuovo, mentre conversava con gli amici} solo di argomenti leggeri.
Come ultimo atto, decise di denunciare per iscritto tutte le malefatte del \emph{princeps} e, dopo aver apposto il proprio sigillo, lo spezzò \say{perché non dovesse più tardi servire a provocare altre vittime}

\subsection{Il \emph{Satyricon}}
Il titolo dell'opera deriva dall'aggettivo greco \emph{satyrikòs}, ovvero \say{relativo ai satiri, satiresco}. 
Questo sarebbe un riferimento al carattere piccante e lascivo della narrazione, proprio come i \emph{sàtyroi}.
Un'altra possibile origine sarebbe da ricondursi alla \emph{satura menippea}, un genere letterario caratterizzato dall'uso del prosimetro.
Gli unici stralci a noi pervenuti sono relativi ai libri XIV, XV e XVI. 
Questi frammenti sono stati recuperati dai cosiddetti \emph{excerpta maiora} e \emph{minora}, a cui manca la cena di Trimalchione, rinventa da Marino Statilio nel 1654 nel \emph{codex Traguriensis}.

\subsubsection{La trama}
Il \emph{Satyricon} può essere letto come una sorta di parodia dell'\emph{Odissea}: un terzetto di giovani, Encolpio, Ascilto e Gitone, si imbarcano in un viaggio incalzati dall'ira del dio Priapo. 
La narrazione inizia in una \emph{graeca urbs}, probabilmente Cuma. 
Dopo alcune peripezie e un'orgia espiatoria, i tre protagonisti (Encolpio, Ascilto e Gìtone) vanno a cena da Trimalchione, un liberto ricchissimo ma rozzo. 
Di ritorno dalla cena, i tre conoscono il poetastro Eumolpo, che recita una \emph{Troiae halosis}. 
Questa è probabilmete una beffa nei confronti di Nerone, che anch'egli aveva composti la \emph{Presa di Troia}.
Ad ogni modo, dopo aver scoperto che Ascilto è un violento ed averlo abbandonato, Encolpio, Gìtone ed Eumolpo partono con una nave. 
Dopo alcune tensioni dovute a vecchie conoscenze, la nave naufraga nei pressi di Crotone, dove Eumolpo si fa passare per un ricco possidente per farsi offrire cene e regali dai cacciatori di eredità. 
Quando gli abitanti della città sospettano l'inganno, Eumolpo scrive il proprio testamento, nel quale dichiara che la sua eredità andrà a chi si ciberà della sua carne di fronte al popolo.
I frammenti del \emph{Satyricon} si interrompono qui, pertanto è impossibile conoscere la conclusione della storia.

\subsubsection{I personaggi}
Nessun personaggio del romanzo può essere considerato completamente positivo, Petronio racconta figure dei bassifondi della società.
Tuttavia, nella sua narrazione Petronio non lascia mai trasparire alcuna condanna morale, ma si colloca al di sopra delle vicende che narra con sguardo ironico e divertito.
Petronio vuole probabilmente raccontare tutti gli strati della società con le sue sfumature. 

\subsubsection{La \emph{Cena Trimalchionis}}
La Cena di Trimalchione è l'episodio più famoso del \emph{Satyricon} e racconta di un banchetto stravagante a casa di Trimalchione. 
Lo stesso nome Trimalchio è un nome parlante: potrebbe infatti essere tradotto come \say{tre volte arricchito}. 
La narrazione di banchetti filosofici era un tema ricorrente nella letteratura antica, basti pensare al \emph{Simposio} di Platone, ma Petronio ne presenta una parodia comica e grottesca. 
Il banchetto è esagerato, così come l'anfitrione della serata, è il trionfo del cattivo gusto e della volgarità di cui Trimalchione e i suoi amici liberti sono il simbolo. 
Il racconto di questo episodio è anche una critica da parte di un aristocratico nei confronti dei liberti arricchiti che stavano scalando le gerarchie sociale. 

\subsubsection{I generi del \emph{Satyricon}}
A primo acchito, il \emph{Satyricon} può essere inquadrato nel genere del romanzo: l'opera è effettivamente un rovesciamento del romanzo greco, fatto di trame stereotipate caratterizzate da amori contrastati, viaggi, avventure, intrecci complessi e separrazioni e ricongiungimenti. 
Tuttavia, all'interno di questo romanzo è possibile riconoscere anche l'influenza delle \emph{fabulae Milesiae}, una raccolta di racconti erotici (come quello della matrona di Efeso) scritti da Aristide di Mileto e tradotti da Cornelio Sisenna in latino. 
Inoltre, il ricorso al prosimetro rimanda alla \emph{satura Menippea}, dove la parodia si univa alla varietà di contenuti e toni.
Insomma, è impossibile inquadrare il \emph{Satyricon} in un unico genere letterario.
Trimalchione

\subsubsection{Lo stile}
Nella sua narrazione, Petronio mescola il comico e il grottesco, una deformazione esagerata di situazioni e personaggi. 
Il suo raccontare molte sfaccettature della società si rispecchia anche nel linguaggio usato: un vero e proprio pastiche di registri. 
Un importante merito di Petronio è la documentazione accurata del \emph{sermo vulgaris}, la lingua parlata dei ceti più bassi, che si evolverà nel volgare. 
A questo si alterna un linguaggio aulico tipico dei personaggi più colte, numerosi diminutivi tipici del linguaggio parlato e alcuni \emph{hapax legomena}, parole utilizzate una volta sola nella letteratura. 

\section{Tacito}

\subsection{La vita}
Tacito nacque nel 55 d.C., la sua terra natia è incerta, forse la Gallia Narbonense oppure Terni. 
Suo suocero era Giulio Agricola, generale romano che conquistò la Britannia, a cui Tacito dedicò un'opera. 
La sua fu un'eduazione eccellente e ben presto si dedicò all'avvocatura e alla politica. 
Fu sempre di ideali repubblicani, ma riconosceva che l'impero fosse ormai l'unica forma di governo possibile. 
Lui stesso ci racconta del suo \emph{cursus honorum}: \say{quanto agli onori della carriera, non potrei negare che Vespasiano li abbia inaugurati, Tito accresciuti, Domiziano spinti  ancora più in là}. 
Sotto Domiziano infatti, Tacito fu propretore e successivamente, sotto Nerva, fu nominato \emph{consul suffectus} nel 97, e infine proconsole in Asia nel 112. 
Tacito morì tra il 116 e il 120 d.C.

\subsection{Le opere}
Di Tacito ci sono pervenute tre opere conservate per intero, l'\emph{Agricola}, la \emph{Germania} e il \emph{Dialogus de oratoribus}. 
Le sue due opere più importanti invece, le \emph{Historiae} e gli \emph{Annales}, ci sono giunte incomplete. 

\subsubsection{L'\emph{Agricola}}
L'\emph{Agricola} è un'opera biografica scritta nel 98 d.C. e dedicata a Giulio Agricola, suocero di Tacito morto in circostanze sospette nel 93 d.C. 
L'opera non è tuttavia priva di scopo: questa è infatti un encomio alla figura di Agricola, attraverso cui loda indirettamente se stesso e la propria famiglia, allontanandosi dai sospetti di complicità con il regime di Domiziano. 
Agricola viene rappresentato come un \emph{vir bonus}, che incarnava l'antica \emph{virtus} romana e che era stato capace di mantenere la propria onestà e correttezza anche sotto il tirannico governo di Domiziano. 

Già dall'inizio, Tacito loda il cambiamento di regime: adesso, sotto Nerva, è stata ristabilita la libertà di pensiero e di espressione. 
Agricola nacque nel 40 d.C. nell'odierna Fréjus e, dopo aver intrapreso una carriera militare, fu nominato governatore della Britannia nel 77 d.C. 
Il generale riuscì a sottomettere la tribù ribelle dei Caledoni, contro cui Tacito ci racconta la battaglia finale, segnata da un drammatico discorso tenuto da Calgàco, il loro comandante. 
Prima di scendere in battaglia infatti, Calgàco prende la parola e critica ferocemente l'imperialismo romano, affermando che sono \say{i rapinatori del mondo}. 
Dopo questa vittoria, Domiziano richiamò Agricola a Roma, dove morì in circostanze misteriose nel 93 d.C.

\subsubsection{La \emph{Germania}}
La \emph{Germania} è un trattato etnografico scritto nel 98 d.C. (forse in occasione della spedizione in Germania di Traiano) che racconta le abitudini e i costumi delle popolazioni germaniche. 
Questo trattato è diviso in due parti, una (capitoli 1-27) prettamente etnografica e l'altra (capitoli 28-46) più geopolitica. 
Tacito utilizzò come fonti il \emph{De bello Gallico} di Giulio Cesare e i \emph{Bella Germaniae} di Plinio il Vecchio. 
Oltre alla descrizione dei popoli germanici, il problema affrontato da Tacito è capire perché il potere di Roma si sia bloccato di fronte a queste tribù. 
Insieme alla narrazione degli usi barbari quindi sono presenti numerose riflessioni etiche e morali che portano Tacito ad identificare in loro quelle stesse virtù che avevano caratterizzato gli antichi Romani. 

In tempi più recenti, la \emph{Germania} di Tacito è stata utilizzata a supporto della teoria razziale tedesca. 
Infatti, l'inglese tedeschizzato H.S. Chamberlain, propose una traduzione incentrata sul travisamento intenzionale di un \emph{tamquam} nel capitolo 4. 
Questa piccola variazione modifica radicalmente la frase: da \say{Hanno anche le stesse caratteristiche fisiche, per quanto possibile in un numero così grande di persone} essa diventa \say{Hanno anche le stesse caratteristiche fisiche, nonostante il numero così grande di persone}. 
La variazione è stata utilizzata per supportare la presunta purezza della razza ariana. 
È da notare però che, nonostante il rispetto che Tacito provava per le virtù dei Germani, l'autore non intendeva esaltarli, riconoscendo in essi un popolo rozzo, ozioso e dedito a vizi come il bere e il gioco d'azzardo. 

\subsubsection{Le \emph{Historiae}}
Le \emph{Historiae} raccontano il periodo che va dal 69 d.C., il cosiddetto anno dei quattro imperatori, fino al 96 d.C., anno della morte di Domiziano e quindi della fine della dinastia flavia. 
Erano composte da quattordici libri, di cui tuttavia disponiamo solo dei primi quattro e i primi ventisei capitoli del quinto. 
L'opera è di impianto annalistico e, dopo aver tracciato un quadro dei fatti che intende narrare, Tacito racconta della nomina di Galba a imperatore. 
Galba è però un vecchio fragile e non riesce a mantenere il potere, cadendo vittima di una congiura ordita da Otone, che prese il trono. 
Ma il suo regno non era destinato a durare: le legioni della Germania avevano acclamato Vitellio imperatore. 
Seguì una feroce guerra civile tra le armate di Otone e quelle di Vitellio, in cui quest'ultimo ebbe la meglio con una battaglia nella valle del Po. 
Otone preferisce suicidarsi piuttosto che cadere nelle mani del nemico. 
Il potere di Vitellio non fu mai accettato da tutto l'esercito, infatti alcune legioni avevano acclmanato imperatore il generale Vespasiano, che in quel periodo stava sedando la rivolta degli Ebrei in Giudea. 
Le legioni di Vespasiano arrivarono in Italia e sconfissero le armate di Vitellio nello stesso luogo in cui esse avevano sconfitto Otone. 
Quando l'esercito di Vespasiano giunse a Roma, Vitellio fu portato in giro per la città e infine linciato dal popolo. 
Dal principato di Vespasiano, la narrazione passa a due rivolte che stavano minacciando l'impero: quella dei Batavi e quella dei Giudei. 
Si apre quindi un \emph{excursus} sugli usi e costumi della civiltà giudaica imbevuto di antisemitismo. 

\subsubsection{Gli \emph{Annales}}
Gli \emph{Annales} raccontano in sedici libri gli anni degli imperatori compresi tra la morte di Augusto e quella di Nerone. 
Anche quest'opera, al pari delle \emph{Historiae}, segue un impianto annalistico. 
I libri tramandati fino ad oggi riguardano il principato di Tiberio, parte di quello di Claudio e quasi tutto quello di Nerone. 

Tiberio era stato nominato erede da Augusto, era il miglior generale della sua epoca ma aveva un carattere cupo e sospettoso. 
Tacito dipinge il progressivo declino morale dell'imperatore verso la follia, che lo spinge a diventare un tiranno. 
Contrapposto al personaggio di Tiberio c'è l'eroe Germanico, generale che conduceva la guerra in Germania. 
La guerra raccontata nei libri I e II fu condotta per vendicare la disfatta di Quintilio Varo, che nel 9 d.C. era stato sconfitto dai Germani nella selva di Teutoburgo. 
Germanico stesso visiterà questa selva in un episodio commovente. 
Proprio quando pareva che Germanico stesse per domare tutta la Germania, Tiberio lo richiamò a Roma, geloso delle sue imprese. 
Germanico fu quindi inviato in Oriente, dove morì (forse avvelenato). 
Nei libri III e IV si racconta invece di Seiano, spietato prefetto del pretorio. 
Questi avviò un violento periodo di repressione del dissenso e, nel 27 d.C., Tiberio si ritirò a Capri, lasciando di fatto tutto il potere nelle mani di Seiano. 
Accortosi del pericolo rappresentato dal prefetto, decide di farlo uccidere ma questa parte della storia non ci è pervenuta. 
Della morte di Tiberio si racconta che una sera egli sembrò esalare l'ultimo respiro. 
Mentre il futuro Caligola si stava appropriando delle insegne imperiali, però, l'imperatore si risvegliò chiedendo del cibo. 
Macrone risolve la situazione tesa ordinando di soffocare Tiberio sotto un cumulo di panni. 
Ecco che l'imperatore Tiberio in morte conservò la \emph{dissimulatio} che l'aveva caratterizzato in vita. 

Non ci è pervenuto il racconto del principato di Caligola. 
Quando la narrazione riprende, è imperatore Claudio, fratello di Germanico.  
Egli non fu un cattivo imperatore ma Tacito lo giudica comunque con severità per via della sua insicurezza. 
Claudio venne sedotto da Agrippina Minore, che riuscì a contrarre matrimonio con lui. 
Questa riuscì a convincere Claudio ad adottare Nerone e nominarlo proprio erede al trono e, in seguito, avvelenò il marito. 

Il principato di Nerone fu costellato di delitti: egli fece uccidere Britannico e la madre Agrippina. 
L'imperatore non ha alcun freno morale, si affidò al prefetto del pretorio Tigellino e dilapidò le casse statali per compiacersi la plebe. 
Dopo l'incendio di Roma del 64 d.C., Nerone incolpò la comunità cristiana, di cui si raccontano le prime persecuzioni. 
Gli ultimi frammenti degli \emph{Annales} riguardano la congiura di Pisone, che costò la vita a figure come Seneca, Lucano e Petronio. 

\subsection{Gli ideali}
Tacito scrive la storia \emph{perché le virtù non siano passate sotto silenzio}. 
Riconosce quindi la precarietà della vita umana e l'affermarsi sempre più prepotente della smania di potere. 
La visione di Tacito è quindi tragicamente pessimista. 
Nell'\emph{incipit} degli \emph{Annales}, Tacito afferma l'imparzialità di ciò che scrive, criticando i suoi predecessori che si erano abbandonati all'ignoranza o al servilismo. 
Tuttavia, Tacito stesso rappresenta una prospettiva parziale, quella senatoria: per esempio egli non riconosce a Tiberio il merito di un'amministrazione oculata o a Claudio quello di saggio legislatore. 
Tacito considerava il senato come fondamentale all'interno delle istituzioni romane, ma tuttavia riconosceva impossibile un ritorno alla repubblica e inevitabile un governo assoluto imperiale. 
Di conseguenza, egli auspicava un buon principe, capace di far coesistere le libertà personali con la stabilità di potere. 
Per questo motivo, egli scelse come protagonisti della sua storiografia singoli personaggi straordinari, perché la storia era ormai decisa da pochi individui. 
Nelle sue opere, Tacito è stato fortemente influenzato da Sallustio, riprendendo da lui: 
\begin{itemize}
    \item la concezione pessimista della natura umana; 
    \item l'intento moralistico delle opere; 
    \item la centralità del singolo nel racconto; 
    \item l'indagine psicologica che evidenzia vizi e virtù dei personaggi; 
    \item l'\emph{inconcinnitas} della prosa. 
\end{itemize}

\section{Apuleio}

\subsection{La vita}
Apuleio nacque attorno al 125 d.C. a Madaura, nell'attuale Algeria. 
Della sua biografia sappiamo ben poco, principalmente da ciò che lui stesso racconta, soprattutto nella sua \emph{Apologia} e nei \emph{Florida}. 
Apuleio apparteneva a una famiglia agiata, grazie alla quale potè studiare a Cartagine e successivamente ad Atene. 
Nella sua vita si dedicò a numerosi culti misterici, come quello di Asclepio, di Demetra e successivamente di Iside e Osiride. 
Nell'\emph{Apologia} possiamo scoprire alcune informazioni sulla vita privata di Apuleio. 
Infatti, nel 155 d.C., Ponziano, amico di Apuleio, lo convinse a sposare la madre Pudentilla (di recente rimasta vedova) per metterla al riparo dai cacciatori di eredità. 
Poco dopo il matrimonio tuttavia Ponziano morì e Apuleio fu accusato dal resto della famiglia di aver praticato stregoneria per sedurre Pudentilla e farsi designare da lei come unico erede. 
L'\emph{Apologia} fu il suo discorso di difesa ed era articolata in tre parti in cui rispettivamente delegittimava l'accusa, spiegava i due tipi di magia esistenti (nera e bianca) e infine rivelava che nel testamento non era stato scelto come erede. 
Ritornò poi a Cartagine, dove morì dopo il 170 d.C.

\subsection{Le opere}
Apuleio fu un autore estremamente prolifico sia in greco che in latino, tuttavia le opere giunte a noi sono scarse. 
Tra le opere che si sono conservate, oltre ai \emph{Florida} e all'\emph{Apologia}, possiamo nominarne alcune a carattere filosofico: 
\begin{itemize}
    \item \emph{De mundo}, in cui tratta di questioni cosmologiche e teologiche; 
    \item \emph{De Platone ed eius dogmate}, in cui racconta la biografia di Platone e il pensiero del filosofo; 
    \item \emph{De deo Socratis}, in cui spiega il ruolo dei dèmoni, mediatori tra uomini e dei. 
\end{itemize}
In questi testi Apuleio si fa portavoce della filosofia del \say{medio platonismo}, una fusione tra elementi platonici e aristotelici. 
Secondo questa teoria, il cosmo è diviso in una sfera divina, caratterizzata dalla razionalità, e in una sfera umana, caratterizzata dalla passionalità. 
Le due sarebbero messe in comunicazione dai dèmoni, immortali come gli dèi ma passionali come gli uomini. 

\subsubsection{Le \emph{Metamorfosi} (o \emph{L'asino d'oro})}
Il capolavoro di Apuleio furono senza dubbio le \emph{Metamorfosi}, un romanzo scritto in undici libri probabilmente dopo il 155 d.C. 
La vicenda è raccontata in prima persona da Lucio, protagonista dell'opera.

Nei primi tre libri, il giovane Lucio si presenta e racconta del suo viaggio di affari a Ipata, in Tessaglia (una terra nota per la magia). 
Già dall'inizio del romanzo, è evidente la caratteristica che sarà il motore dell'azione: la \emph{curiositas} del protagonista. 
Durante il viaggio viene messo in guardia circa la pericolosità delle stregonerie che avvengono a Ipata, ma Lucio non si cura degli ammonimenti. 
In città viene ospitato da Milone e da sua moglie Pànfile. 
Lucio intreccia una relazione con Fòtide, serva di Milone e Panfile, grazie alla quale può osservare la padrona di casa trasformarsi in uccello per incontrare i propri amanti. 
Esterrefatto, Lucio vuole provare l'incantesimo su di sé ma per un errore viene trasformato in un asino. 
Per ritornare uomo, gli basterebbe mangiare delle rose contenute in giardino, ma durante la notte viene rubato: dovrà faticare a lungo prima di ottenere l'agognato ritorno all'umanità. 

All'interno del romanzo viene raccontata la vicenda di Amore e Psiche. 
Psiche è una ragazza talmente bella da suscitare l'invidia di Venere, che invia suo figlio Amore a farla innamorare di un mostro. 
Tuttavia, lo stesso Amore si invaghisce della giovane e i due intraprendono una relazione amorosa che sarebbe continuata a patto che Psiche non conoscesse l'identità del suo amante. 
Una sera, spinta dalla \emph{curiositas}, Psiche cerca di spiare Amore nel sonno. 
Cupido però si sveglia per via di una goccia di olio bollente e si allontana immediatamente. 
Dopo numerose peripezie, Amore e Psiche potranno finalmente sposarsi. 

Finito questo racconto nel racconto, riprendono le peripezie di Lucio, che passa di mano in mano a svariati proprietari. 
Dopo essere riuscito a fuggire, Lucio vede in sogno la dea Iside, che gli fornisce istruzioni per ritornare umano in occasione di una celebrazione religiosa nel corso della quale riuscirà finalmente a mangiare le fatidiche rose. 
Ritornato uomo, Lucio diviene sacerdote di Osiride e si dedica con successo all'avvocatura. 

Gli argomenti principali della storia narrata e dei vari racconti nel racconto sono la magia, l'adulterio, l'inganno e l'omicidio. 
Il motore dell'azione è la \emph{curiositas}, che non è solo rovinosa ma funge anche da consolazione a Lucio-asino. 
Un problema aperto delle \emph{Metamorfosi} è l'intervento della dea Iside nel libro XI. 
Quest'ultima parte risulta infatti discorde con il resto del romanzo, le due interpretazioni che ne seguono sono opposte:
\begin{itemize}
    \item secondo alcuni, l'ultimo libro è solo un'appendice seria per dare credibilità a un testo altrimenti completamente frivolo; 
    \item altri ritengono invece che l'ultimo libro conterrebbe la chiave di lettura di tutto il romanzo, che sarebbe un'allegoria al percorso di iniziazione ai misteri isiaci (molto in voga all'epoca perché proponevano una vita dopo la morte). 
\end{itemize}

Gli obiettivi principale di Apuleio con il suo romanzo erano \emph{delectare} il pubblico e al contempo \emph{docere} qualosa. 
Le \emph{Metamorfosi} possono essere inoltre inserite nel filone delle \emph{fabulae Milesiae}. 
Apuleio potrebbe inoltre aver preso spunto da un'opera di Luciano di Samosata (la cui trama di fondo è la stessa delle \emph{Metamorfosi}, ma meno complessa) che a sua volta andrebbe connesso alle opere di Lucio di Patre, ma i rapporti tra questi tre testi non sono ancora chiari. 