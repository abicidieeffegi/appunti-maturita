\section{Il magnetismo}

\subsection{Il campo magnetico}
Il campo magnetico è un campo vettoriale che descrive l'influenza magnetica su cariche in movimento e correnti elettriche. 
In un certo punto dello spazio, il campo magnetico ha:
\begin{itemize}
    \item come \textbf{direzione} quella lungo la quale si disporrebbe un ago magnetico libero di ruotare in un dato punto; 
    \item come \textbf{verso} quello che va dal polo sud al polo nord dell'ago.
\end{itemize}
Un punto di un filo percorso da corrente risente di una forza $\dd \vec{F}$ pari a:
\begin{equation}
    \label{forza magnetica}
    \dd \vec{F} = I \dd \vec{\ell} \times \vec{B}
\end{equation}
Dove $I$ è l'intensità di corrente che scorre nel filo, $\dd \vec{l}$ è l'intervallo infinitesimo di filo orientato nello stesso verso della corrente e $\vec{B}$ è il vettore campo magnetico. 
La forza totale che agisce sul filo $\gamma$ sarà quindi:
\begin{equation}
    \int_{\gamma} \dd \vec{F}
\end{equation}
Che, nel caso di un filo rettilineo immerso in un campo uniforme, può essere semplificata come:
\begin{equation}
    \vec{F} = I \vec{\ell} \times \vec{B}
\end{equation}
Dall'equazione \ref{forza magnetica}, possiamo inoltre ricavare il valore del modulo del campo magnetico in un determinato punto:
\begin{equation}
    \abs{\vec{B}} = \frac{\abs{\dd \vec{F}}}{I\abs{\dd \vec{\ell}} \sin{\alpha}}
\end{equation}
Dove $\alpha$ è l'angolo tra il vettore $\dd \vec{L}$ e il vettore $\dd \vec{B}$. L'unità di misura per il campo magnetico risulta:
\begin{equation}
    \left[\vec{B}\right] = \frac{N}{A \cdot m} = T
\end{equation}

\subsection{La forza di Lorentz}
Una particella carica immersa in un campo magnetico subisce una forza $\vec{F}$ pari a:
\begin{equation}
    \label{forza di Lorentz}
    \vec{F} = q \vec{v} \times \vec{B}
\end{equation}
Considerando i moduli invece avremo:
\begin{equation}
    F = \abs{q}vB\sin{\alpha}
\end{equation}
Dove $\alpha$ è l'angolo tra i vettori $\vec{v}$ e $\vec{B}$. 
Dalle equazioni precedenti possiamo dedurre che:
\begin{itemize}
    \item se $\vec{v} \parallel \vec{B}$ la carica non risente di alcuna forza (poiché $\vec{v} \times\vec{B} = 0$); 
    \item se $\vec{v} \perp \vec{B}$ la carica si muove di moto circolare uniforme; 
    \item se $\vec{v}$ e $\vec{B}$ sono incidenti la carica si muove di moto elicoidale. 
\end{itemize}
Analizziamo meglio gli ultimi due casi. 
Quando $\vec{v} \perp \vec{B}$, $\vec{F} \perp \vec{v}$, pertanto $\abs{\vec{v}}$ è costante. 
La carica si muove quindi di moto circolare uniforme e la forza di Lorentz è la forza centripeta. 
Poiché $\alpha = 90 \degree$, $\sin{\alpha} = 1$ e possiamo quindi affermare che: 
\begin{equation}
    F = \abs{q}vB = \frac{mv^2}{r}
\end{equation}
Da cui possiamo ricavare il raggio della traiettoria circolare che la carica seguirà:
\begin{equation}
    r = \frac{mv}{\abs{q}B}
\end{equation}
Da cui possiamo poi ricavare altre misure come la velocità angolare $\omega$ e la frequenza $f$.
Qualora $\vec{v}$ sia incidente a $\vec{B}$, possiamo scomporre $\vec{v}$ in $v_{\parallel} = v \cos{\alpha}$ e $v_{\perp} = v \sin{\alpha}$. 
Il procedimento è poi analogo a quanto riportato sopra. 
Possiamo inoltre calcolare il passo dell'elica, ovvero la distanza tra due \say{spire} consecutive. 

\subsection{Il ciclotrone}
Il ciclotrone è un macchinario utilizzato per accelerare particelle. 
Fu inventato da Ernest Orlando Lawrence nel 1932 e viene sfruttato ancora oggi per accelerare ioni leggeri. 
Due conduttori semicilindrici cavi (in gergo chiamati D) sono sottoposti a un campo magnetico uniforme (perpendicolare alle basi delle D). 
Le due D sono collegate a un alternatore che crea un campo elettrico tra loro. 
Nella zona centrale, una sorgente emette particelle cariche che vengono messe in moto dal campo elettrico. 

Se si sincronizza il moto delle particelle con la frequenza della tensione alternata, ogni volta che la particella torna nella zona centrale incontra un campo favorevole che la accelera ulteriormente. 
Perché questo avvenga è necessario che il tempo $\Delta t$ sia: 
\begin{equation}
    \Delta t = \frac{T}{2} = \frac{\pi m}{\abs{q} B}
\end{equation}
Questo è indipendente dalla velocità, quindi la particella mantiene la stessa frequenza nonostante cambi la sua energia cinetica. 
La particella e la tensione alternata resteranno quindi sincronizzate purché quest'ultima sia: 
\begin{equation}
f = \frac{\abs{q} B}{2\pi m}
\end{equation}
Dal macchinario esce quindi un fascio di ioni ad alta energia utilizzati per ricerca o per fini medici come la cura di tumori. 

\begin{figure}[ht]
    \centering
    \includegraphics[width=\linewidth]{cyclotron.jpg}
    \caption{Il funzionamento schematico di un ciclotrone}
\end{figure}

\subsection{Il selettore di velocità}
Questo macchinario sfrutta la sovrapposizione del campo magnetico e del campo elettrico in una stessa regione di spazio per ottenere un fascio di particelle alla stessa velocità. 
Per costruire tale macchinario consideriamo la forza a cui è soggetta una carica immersa in un campo elettrico e magnetico: 
\begin{equation}
    \vec{F} = q \vec{E} + q \vec{v} \times \vec{B} = q(\vec{E} + \vec{v} \times \vec{B})
\end{equation}
Perché una particella carica ($q \neq 0$) non risulti né deflessa né accelerata, è necessario che $\vec{F} = 0$. 
Questo può succedere solo quando $\vec{E} + \vec{v} \times \vec{B} = 0$ ovvero:
\begin{equation}
    \vec{E} \perp \vec{B} \\
\end{equation}
\begin{equation}
    v = \frac{E}{B}
\end{equation}

\newpage

\subsection{Lo spettrometro di massa}

\begin{wrapfigure}{l}{0.5\textwidth}
    \includegraphics[width=0.5\textwidth]{spettrometro}
\end{wrapfigure}

Lo spettrometro di massa è uno strumento utilizzato per misurare la massa degli ioni. 
Il campione da esaminare viene innanzitutto ionizzato. 
Per semplicità, consideriamo la prima ionizzazione, ovvero le particelle avranno carica $q=e$. 
Gli ioni prodotti vengono accelerati con un campo elettrico e successivamente vengono fatti passare in un selettore di velocità. 
Successivamente, gli ioni entrano in una regione dove è presente un campo magnetico uniforme che devia le particelle lungo traiettorie circolari. 
Il raggio di tali traiettorie, come abbiamo osservato precedentemente, è:
\begin{equation}
    r = \frac{v}{eB} m
\end{equation}
È quindi possibile separare ioni di massa diversa a seconda del raggio ($r \propto m$). 

\section{L'elettromagnetismo}

\subsection{Le leggi di Maxwell}
Lo studio dell'elettromagnetismo si basa sulle quattro equazioni di Maxwell.
\begin{gather}
    \label{gauss elettrico} \Phi_{S_C}(\vec{E}) = \frac{Q}{\epsilon_0} \\
    \label{gauss magnetico} \Phi_{S_C}(\vec{B}) = 0 \\
    \label{circuitazione elettrico} C_{\gamma}(\vec{E}) = -\diff{\Phi_{\gamma}(\vec{B})}{t} \\
    \label{circuitazione magnetico} C_{\gamma}(\vec{B}) = \mu_0\left(\sum I_k + \epsilon_0 \diff{\Phi_S(\vec{E})}{t}\right)
\end{gather}
Analizziamo una per una cosa significano. 

La legge di Gauss per il campo elettrico (\ref{gauss elettrico}), afferma che il flusso del campo elettrico uscente da una superficie chiusa $S_C$ è direttamente proporzionale alla carica $Q$ racchiusa al suo interno. 
Da questa legge si deduce che le linee di campo elettrico possono essere aperte e quindi possiamo affermare che esistono \say{pozzi} e \say{sorgenti} di campo elettrico. 

La legge di Gauss per il campo magnetico (\ref{gauss magnetico}) afferma che il flusso del campo magnetico uscente da una superficie chiusa $S_C$ è sempre nullo. 
Questa legge ci dice che non esistono monopoli magnetici e che le linee di campo magnetico sono sempre chiuse: non possono esistere né \say{pozzi} né \say{sorgenti}. 

La circuitazione del campo elettrico (\ref{circuitazione elettrico}) lungo una linea chiusa e orientata $\gamma$ è direttamente proporzionale alla derivata del flusso del campo magnetico concatenato con $\gamma$. 
Da questa possiamo dedurre che il campo elettrostatico è conservativo poiché $C_{\gamma}(\vec{E}_{statico}) = 0$, mentre il campo elettrico indotto no ($C_{\gamma}(\vec{E}_{indotto}) \neq 0$). 

La circuitazione del campo magnetico (\ref{circuitazione magnetico}) è direttamente proporzionale alla somma tra correnti conduzione e correnti di spostamento. 
Questa legge comporta che, così come il campo elettrico può essere generato da un campo magnetico variabile, anche il campo magnetico può essere generato da un campo elettrico variabile. 

\subsection{Le onde elettromagnetiche}
\begin{figure}[h]
    \centering
    \includegraphics[width=\textwidth]{onda}
    \caption{Un'onda elettromagnetica piana sinusoidale}
\end{figure}
Un'importante previsione delle leggi di Maxwell fu l'esistenza delle onde elettromagnetiche. 
Queste onde hanno alcune caratteristiche particolari:
\begin{itemize}
    \item possono propagarsi anche nel vuoto;
    \item sono onde trasversali, ovvero i campi elettrico e magnetico vibrano perpendicolarmente alla direzione di propagazione dell'onda;
    \item il campo elettrico $\vec{E}$ e quello magnetico $\vec{B}$ sono sempre perpendicolari;
    \item il verso di propagazione dell'onda è quello del prodotto vettoriale $\vec{E} \times \vec{B}$;
    \item $\vec{E}$ e $\vec{B}$ oscillano in fase, $E = c B$;
    \item la velocità delle onde elettromagnetiche nel vuoto è $c = \frac{1}{\sqrt{\epsilon_0 \mu_0}} = 299\,792\,458 m/s$;
\end{itemize}