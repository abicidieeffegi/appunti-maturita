\section{Romanticism}

\subsection{William Blake (1757 - 1827)}

\subsubsection{Life an´ works}
Blake hailed from a humble family and he remained poor throughout all his life. 
He worked as an engraver and painter, creating a new art that emphasized the power of imagination over reason. 
This meant that imagination was the only to get to know the world, making the poet a sort of prophet who can see more deeply into reality. 
Blake was always in favour of the French Revolution and he remained a radical all his life. 
He believed that progress corrupted man's soul and that the artist should be the guardian of the spirit and imagination. 
He had a strong sense of religion and his greates literary influence was the Bible. 
He invented the "illuminated printing", an artistic form combining poetry and pictures. 
His most notable works are the \emph{Songs of Innocence} (1789) and the \emph{Songs of Experience} (1794), but he also wrote prophetic books. 
Blake also theorized that \say{complementary opposites} (such as Innocence and Experience) exist in parallel and the possibility of growth lies in the tension between these two opposites (cfr Hegel). 
Blake's style is simple, full of symbols. 
His verse is linear and rythmical and is characterized by a heavy use of repetition. 

\subsubsection{\emph{Songs of Innocence}}
The \emph{Songs of Innocence} were written before the French Revolution. 
The narrator is a shepherd who, inspired by a child in a cloud, makes songs celebrating the divine presence in everything. 
The symbol of innocence is childhood, depicted as a period of happiness, freedom and imagination. 

\subsubsection{\emph{Songs of Experience}}
Blake wrote the \emph{Songs of Experience} during the period of the Terror after the French Revolution. 
This is in fact the counterpart of the \emph{Songs of Innocence}: here a bard questions the themes tackled in the first collection. 
These new songs are meant to be read in pair with the older ones and it can be seen that a more pessimistic view of life emerges. 
Experience is identified with adulthood but this does not replace childhood, in fact it merely coexists, providing another point of view on reality. 

\subsection{Mary Shelley (1797 - 1851)}

\subsubsection{Life and works}
Mary Shelley was the daughter of Mary Wollstonecraft (a feminist philosopher) and William Godwin (an anarchist and philosopher). 
Her mother died shortly after her birth and her father later remarried. 
Godwin's house was visited by some of the most notable charactes of the time, such as Samuel Taylor Coleridge and Percy Bysshe Shelley. 
The latter fell in love with Mary and the couple fled to Switzerland, where Mary found the inspiration to write \emph{Frankenstein, or The Modern Prometheus} which was published anonimously in 1818. 
In 1822 Percy set sail and was found drowned after a storm and the next year Mary returned to England, where she lived the rest of her life. 

\subsubsection{\emph{Frankenstein, or The Modern Prometheus}}
Victor Frankenstein, a Swiss scientist, is able to create a human being stitching together different parts of corpses. 
The result of the experiment however is appalling: he becomes a murderer and kills his creator. 
The story is told through a series of letters that Walton (a young explorer on a journey to the North Pole) writes to his sister Margaret. 
The novel is set throughout Europe, but the most important place is the North Pole, where Frankenstein is found chasing his creation. 

The theme of the double is recurring throughout the novel (cfr Pirandello, Svevo). 
Walton is a double of Frankenstein for they share the same ambition to overcome human limits, the same loneliness and the same pride. 
Frankenstein and his creature are complementary: they both suffer from isolation and alienation, they both desire to be good but get obsessed with hate and revenge. 
Frankenstein's rejection of his creature is what makes it an outcast and a murderer. 

Mary Shelley was heavily influenced by the latest advancements in science and it proposes the problem of its responsibility toward humanity. 
The creature can also be considered Rousseau's natural man, in a primitive state. 
She also took inspiration from Coleridge's \emph{The Rime of the Ancient Mariner} as both depict a crime against nature (the killing of the albatross is akin to the creation of the monster). 
The Greek myth of Prometheus was also an important influence for the novel. 

\subsection{William Wordsworth (1770 - 1850)}

\subsubsection{Life and works}
Born in England, Wordsworth came in contact with revolutionary France and he was filled with enthusiasm for democratic ideals. 
The subsequent developments of the Revolution brought him on the verge of a nervous breakdown. 
In 1795 he met Samuel Taylor Coleridge,  with whom he developed a strong friendship. 
Together they wrote the \emph{Lyrical Ballads}, a collection of poems published anonimously in 1798. 
The second edition (1800) also contained Wordsworth \emph{Preface}, which later became the Manifesto of English Romanticism. 
He carried on writing and growing his poetic reputation until he was made Poet Laureate in 1843. 
The last years of his life were characterized by a growing political conservatism. 

\subsubsection{The Manifesto of English Romanticism}
Wordsworth regarded poetry as a solitary act originating in the ordinary. 
In his preface, he describes it as \say{emotion recollected in tranquillity}. 
Hence, the subject matter should deal with everyday situations and ordinary people and the language should be simple. 
Therefore, the poet should be a man among men. 
Wordsworth also believed in the goodness of nature and that of the child. 
He also thought that man and nature are inseparable, offering a sort of pantheistic world view. 

\subsection{Samuel Taylor Coleridge (1772 - 1834)}

\subsubsection{Life and works}
Coleridge received an excellent education in the classics but failed to graduate at Cambridge. 
As a student, he was heavily influenced by the ideals of the French Revolution, becoming an enthusiastic republican. 
After his disillusionment with the Revolution, he and the poet Robert Southey planned to create a utopian society (called \emph{Pantisocracy}) in America in which private ownership would not exist (cfr Marx). 
The project would never come to life. 
In 1795 he met Wordsworth with whom he composed the \emph{Lyrical Ballads}. 
His masterpiece, \emph{The Rime of the Ancient Mariner} was written in 1798 and it is the first poem in the \emph{Lyrical Ballads}. 
Other of his works include \emph{Christabel} (an unfinished poem set in the Middle Ages), \emph{Kubla Khan} (an unfinished poem probably written under the influence of opium), and \emph{Biographia Literaria} (an autobiography and literary critic). 
Coleridge's concern was to write about extraordinary events in a credible way. 

\subsubsection{\emph{The Rime of the Ancient Mariner}}
In the first part, the ancient Mariner stops a wedding guest to tell him about his journey. 
The protagonist and his fellow reached the South Pole after a violent storm. 
After a few days, an albatross appeared and was considered a sign of good omen. 
The Mariner shot dead the albatross. 
This is seen as a crime against nature and in the next part the Mariner is punished for his misdeed. 
In the third part, the Mariner becomes conscious of what he has done. 
In this section, a phantom ship approaches the crew and Death and Life-in-Death cast dice. 
Death wins the Mariner's fellows who die, while Life-in-Death wins the Mariner's life. 
In the fourth part, the Mariner is alone and trying to reconcile with nature. 
In the fifth part, the process of revival of the soul continues. 
In the sixth part, the purification seems impeded by an unknown obstacle. 
In the seventh part, the Mariner gains the wedding guest's sympathy and is still haunted by a sense of guilt that will only end with his death. 

Coleridge saw nature as essential to poetic creativity for it stimulated the poet to find symbols that could reflect his feelings. 
The poem shares many of the features of medieval ballads such as the structure, the archaic language, the use of alliteration, repetition, and onomatopoeia. 
\emph{The Rime of the Ancient Mariner} has been interpreted in many different ways. 
It may be the description of a dream, an allegory of the life of the soul, or a description of the poetic journey of Romanticism. 

\subsection{George Gordon Byron (1788 - 1824)}

\subsubsection{Life and works}
Rich and handsome, he had a deformed foot and was quite unconventional. 
He was a brilliant mind and forced himself to become excellent at sports. 
He went on his Grand Tour in 1809 and during this time he found the inspiration to write the first two cantos of \emph{Childe Harold's Pilgrimage} which he published in 1812. 
This gave Byron great fame until he fled England in 1816 because a scandal broke out from his incestuous relationship with his half-sister. 
In 1816 he therefore moved to Geneva, becoming a close friend with Percy Bysshe Shelley. 
He also wrote the third canto of \emph{Childe Harold}. 
He then moved to Venice, where he wrote the tragedy \emph{Manfred}, the fourth and last canto of \emph{Childe Harold}, the mock-heroic poem \emph{Beppo}, and the mock-epic \emph{Don Juan}. 
In 1819 he moved to Milan where he plotted against Austrian rule over Italy. 
Later, he committed to Greek struggle of independence from Turkey. 
For his actions he is regarded as a hero in Greece, where his heart is buried. 
Byron never considered himself a romantic poet, in fact he criticized Wordsworth, Coleridge, and Keats. 
Nevertheless, he was the only poet of his time to gain international recognition and to influence the work of authors such as Dostoevskij, Pushkin, Goethe, and de Balzac. 

Byron firmly believed in individual liberty and fought against any kind of constraint. 
He wished all men to be free so he devoted his life to fight against tyrants across the world. 
The protagonists of his works are isolated men struggling against nature (cfr Leopardi) whose feelings are reflected in their surroundings. 
He denounced the evils of his society through a satirical style. 

\subsubsection{\emph{Childe Harold's Pilgrimage}}
The poem is divided in four independent cantos. 
The \emph{fil rouge} is given by the protagonist, Harold, a nobleman awaiting knighthood (this is the meaning of \say{Childe}) who travels around the world. 
Harold's boredom and disillusionment cause him to leave England and set off to exotic places. 
The first two cantos are set through Spain, Portugal, Albania, and Greece and they evoke the glorious past of these nations and the scenery. 
In the third canto, Byron reflects upon human ability to forget while narrating his own journey after he left England. 
In the last canto, Byron depicts the sea as the image of the sublime and of eternity. 

\subsection{Percy Bysshe Shelley (1792 - 1822)}

\subsubsection{Life and works}
Percy Bysshe Shelley was the son of a wealthy and conservative Member of Parliament. 
He rebelled against his conservative family publishing a pamphlet named \emph{The Necessity of Atheism} in 1811, which caused him to get expelled from Oxford University. 
He married Harriet Westbrook at age 19 and they moved to Ireland, where Shelley made revolutionary propaganda against Catholicism and English authority. 
Even though the enthusiasm for the French Revolution had died down, Shelley was a republican, vegetarian and advocate of free love. 
He was interested in occult sciences and scientific experiments. 
When he and his wife came back to England, they realized their marriage was not working so they separated and Percy later remarried with Mary Wollstonecraft Godwin. 
They went to live in Switzerland and later in Italy, where Percy wrote \emph{Ode to the West Wind} in 1819 and other works. 
His life ended tragically: after having set sail from Livorno, his ship was struck by a storm which drowned him. 

In his essay \emph{A Defence of Poetry}, Shelley regards poetic activity as an expression of imagination capable of revolutionary activity in an increasingly materialistic world. 
In Shelley's opinion, nature is not the real world but a beautiful veil concealing the eternal truth (cfr Schopenhauer). 
Nature also represents a shelter from the disappointment caused by the ordinary world. 
The poet for Shelley is both a prophet and a titan challenging the universe (cfr Leopardi). 
His task is to help humanity reach a world characterized by freedom, love, and beauty. 

\subsubsection{\emph{Ode to the West Wind}}
Published with \emph{Prometheus Unbound}, this is a lyrical composition with an elevated tone. 
In this work, Shelley identifies with Prometheus himself, the heroic titan. 
Much like Prometheus, Shelley hopes that his fire (his liberal philosophy) will enlighten humanity and liberate it from intellectual imprisonment. 

\section{The Victorian Age}

\subsection{Alfred Tennyson (1809 - 1892)}

\subsubsection{Life and works}
Alfred was the fourth son (out of twelve) of a clergyman. 
He was educated at Trinity College in Cambridge, showing off his intelligence and humour, but he dropped off without graduating. 
During his years in Cambridge he met Arthur Hallam, with whom he would travel to the Continent. 
Arthur died in Vienna in 1833 and Alfred would spend several years meditating on this tragic loss. 
In 1850 he was made Poet Laureate and in 1884 he was nominated Baron for his literary merits, joining the House of Lords. 

His first remarkable works were the dramatic monologues included in his collection \emph{Poems} (1842). 
His masterpiece \emph{Ulysses} is part of this collection. 
He also wrote the poem \emph{The Princess}, in favour of women's right to education, and the elegy \emph{In Memoriam}. 

\subsubsection{Ulysses} 
The inspiration for this monologue comes from Dante, who tells the story of Ulysses's last adventure in his \emph{Divina Commedia}. 
Ulysses is an overreacher, thirsty for knowledge at any cost. 
Tennyson depicts two different kinds of life through Ulysses and his son, Telemachus. 
In fact, while Ulysses represents an active, adventurous life, Telemachus embodies the typical Victorian man. 

\subsection{Charles Dickens (1812 - 1870)}

\subsubsection{Life and works}
Charles Dickens lived an unhappy childhood: his father was imprisoned for debt when Charles was 12. 
This forced him to go to work in a factory. 
When his father was freed, he was sent to a school in London. 
He began studying shorthand writing and by 1832 he became a successful reporter of parliamentary debates and began working as a writer for a newspaper. 
In 1833 he published his very first story. 
He wrote \emph{Sketches by Boz} and later \emph{The Pickwick Papers}. 
After its latest success, he began a full-time career as a novelist, producing works such as \emph{Oliver Twist} (1838), \emph{A Christmas Carol} (1843), \emph{David Copperfield} (1850), and \emph{Hard Times} (1854). 
He gained immense fame for his novels and was buried in Westminster Abbey. 

Dickens used to tell stories regarding the lower class world and he was always on the side of the poor. 
Children are often the most important characters in his novels. 
His aim was to school the upper classes and the rulers about the poor and the problems they faced. 

\subsubsection{\emph{Oliver Twist}}
This novel is heavily autobiographical, representing the financial insecurities and humiliation Dickens had to endure as a child. 
The name \say{Twist} itself represents the reversals of fortune he will experience. 
Oliver Twist is a poor boy, son of unknown parents, born in a workhouse in a small town near London. 
He is brought up in said workhouse but one day he commits the offence of asking for more food. 
He is therefore sent as an apprentice to anyone willing to take him. 
He is first sold to an undertaker who is cruel and makes Oliver run away to London. 
In London he falls into the hands of a gang of pickpockets trained by Fagin, who runs a school for thieves. 
Oliver is caught on his first attempt at theft: the victim, Mr Brownlow, rather than charging him with theft takes him home and takes care of him. 
Oliver is then kidnapped by Fagin's gang and forced to commit burglaries. 
During one job he is shot and wounded. 
Oliver is then adopted by Mr Brownlow and he receives the love and affection he has always lacked. 
Eventually, investigations are made about Oliver's origins and he is discovered to be of noble descent. 
In the end, the gang of pickpockets and Oliver's half-brother (who paid the thieves in order to ruin Oliver) are all arrested. 

The most important setting in this novel is London, of which he depicts three different social levels. 
First, the world of the parochial workhouse which is insensitive and rigid. 
Second, the criminal world of violent people driven by poverty and hunger. 
Last, the world of the Victorian middle class, a world of respectable people with high moral values. 

\subsubsection{\emph{Hard Times}}
The novel is set in Coketown, an imaginary industrial town. 
Thomas Gradgrind is an educator only interested in facts and statistics. 
He has founded a school and brigs up his two children, Louisa and Tom, repressing their imagination and feelings. 
Louisa is married to a rich banker and Tom is given a job at the bank. 
However, he robs his employer and is forced to leave the country. 
Mr Gradgrind eventually understands the harm he has done his children and gives up his philosophy. 

\emph{Hard Times} is divided into three sections, \say{Sowing}, \say{Reaping}, and \say{Garnering}. 
Coketown represents any real industrial town in the mid 1800s: a brick jungle, heavily polluted, and covered in soot. 
Dickens wants to warn against the dangers of the pilosophy of Utilitarianism. 

\subsection{Walt Whitman (1819 - 1892)}

\subsubsection{Life and works}
Born in Long Island, New York, Walt had little education and at 11 he started working first as an office boy and later as a printer's apprentice. 
He then became a journalist, supporting his radical democratic views. 
When he was about 30, he started travelling and studying by himself. 
In 1855 he published the first (out of nine) edition of \emph{Leaves of Grass}. 
After the third edition he caused the indignation of puritanical readers for his dealing with obscenity and homosexuality. 
During the Civil War he devoted himself to visiting wounded soldiers in the army hospitals. 
He strongly believed in the abolition of slavery and national integrity. 
He then retired and died in Camden, New Jersey. 
He became especially popular in Europe in the Aesthetic movemet. 
He is usually regarded as the father of American poetry. 

All his work is incorporated in \emph{Leaves of Grass}, giving a sense of unity and growth throughout the collection. 
In his poetry, Whitman regarded nature as a shelter from the stress of the materialistic world. 
The natural world is also the body of the Earth. 
\emph{Leaves of Grass} is written in free verse, it often gives a sense of fragmentation. 

Whitman's poetry is pervaded by optimism and a great faith in the future of the American nation. 
He celebrated America in all its variety, giving voice to common men. 
He also deals with the theme of physical love and sex. 
What he valued most is the dignity of the individual. 

\subsection{Rudyard Kipling (1865 - 1936)}

\subsubsection{Life and works}
Rudyard Kipling was born in Bombay, India. 
He was able to learn Hindi (alongside with English) and witness Indian life. 
When he was 6 he was sent to England to attend to school. 
In 1882 he returned to India, where he began working as a journalist and published collection of poems and short stories about Indian life. 
He then returned to London and eventually moved to the USA with his American wife. 
He went back to England in 1896 and he continued writing and in 1907 he was the first Englishman to receive the Nobel prize for Literature. 
During the First World War he worked as a correspondent. 
His ashes are buried in Poets' Corner in Westminster Abbey. 
In his works, Kipling exalted imperial power and the white men's superiority. 

\subsection{Oscar Wilde (1854 - 1900)}

\subsubsection{Life and works}
Oscar Wilde was born in Dublin, where he attended Trinity College before being sent to Oxford, where he gained a degree in Classics. 
He was known for being eccentric and he accepted the theory of \say{Art for Art's sake}. 
After his graduation, Oscar moved to London where he became famous for his style as a \emph{dandy} (cfr d'Annunzio). 
In 1881 he published a collection called \emph{Poems}, which won him a speaking tour in the USA. 
When he returned to Europe, he married Constance Lloyd, who bore him two children. 
In 1891 he published his masterpiece \emph{The Picture of Dorian Gray}. 
After this, he developed an interest in drama and produced a series of plays. 
In 1895, Wilde's triumph ended dramatically when he was put to trial and then sentenced to two-years' hard labour for his homosexuality. 
When he was released in 1997, he went into voluntary exile in France, where he died of meningitis in absolute poverty. 

\subsubsection{\emph{The Picture of Dorian Gray}}
The novel is set in London at the end of the 19\textsuperscript{th} century. 
A painter, Basil Hallward, is fascinated by the beauty of young man Dorian Gray and decides to paint his portrait. 
Dorian then throws himself into a life of pleasure, led by the corrupt Lord Henry Wotton. 
All the young man's wishes are satisfied, including that of eternal youth. 
In fact, the signs of age appear on the portrait instead of Dorian. 
He is completely insensitive to the pain he causes around him. 
Eventually Dorian wants to free himself of the painting, witness of all his misdeeds. 
He proceeds to stab the painting, but in doing so he kills himself and the painting goes back to its original purity. 

The picture represents Dorian's dark side, which he tries to forget by locking it in a room. 
The moral of this novel is that in the end every excess must be punished and there is no possible escape from it. 
The corrupted picture can be seen as a symbol of the immorality and bad conscience of the Victorian middle class. 
In the end we can see Wilde's theory of art: art survives people and is eternal. 

\section{Modernism}

\subsection{The War Poets}

\subsubsection{Rupert Brooke (1887 - 1915)}
Rupert Brooke was educated at Rugby School and later went to King's College. 
He was a great student and athlete and was deemed very handsome. 
He got acquainted with literary circles such as the Bloomsbury Group and other important intellectual and political figures. 
A convinced jingoist, he joined the army at the beginning of the conflict but died of blood poisoning as soon as he was deployed to Greece. 
In his sonnets, Brooke claims that war is clean and cleansing. 

\subsubsection{Wilfred Owen (1893 - 1918)}
Wilfred Owen was working as an English teacher in France when he visited a war hospital and decided to return to England and enlist. 
In 1917 he was deployed to France and took part the fight. 
In March of the same year, he was sent to Edinburgh to recover from shell shock. 
There he met Sigfried Sassoon, who encouraged him to continue writing and later contributed to him getting recognized as a poet. 
In 1918, Owen returned to the battlefield and was killed in a German machine gun attack, just seven days before the armistice. 
His poems are dramatic as they offer a truthful description of the pain the soldiers had to endure. 

\subsubsection{Sigfried Sassoon (1886 - 1967)}
Sigfried Sassoon joined the war in 1915 and was sent to France. 
Sassoon expressed his disillusionment with the war through the irony in his poems. 
He also protested publicly against the war. 
A friend of his was able to prevent him from being cour-martialled by convincing everyone Sassoon was suffering from shell shock. 
Sassoon was therefore sent to Edinburgh to recover, where he met Wilfred Owen. 
In his poems he denounced the reality soldiers were being put through. 

\subsection{Thomas Stearns Eliot (1888 - 1965)}

\subsubsection{Life and works}
T.S. Eliot was born in St Louis, Missouri, and educated at Harvard, but his cultural background was English and European. 
He studied Metaphysical poets, John Donne, and he learned Italian by reading Dante. 
In 1910 Eliot went to Paris where he attended Bergson's lectures and was exposed to the works of French Symbolists. 
As the First World War broke out, Eliot stayed in London and began working as a clerk in a bank. 
In 1915 he married Vivienne Haigh-Wood, despite his parents' worries about her mental instability. 
His first important work was the collection of poems \emph{Prufrock and Other Observations}. 
He then founded \emph{The Criterion}, a literary magazine, and later became director for the publishers \emph{Faber \& Faber}. 
During this time his wife's mental health deteriorated and Eliot himself was under considerable emotional strain. 
He therefore spent some time in a sanatorium in Switzerland and poetry became his refuge from the world. 
In his poems, he expresses the crisis of Western culture (cfr Nietzsche). 
In 1922 he published his masterpiece \emph{The Waste Land}. 
In 1927 Eliot became a citizen of the United Kingdom and in the same year he joined the Church of England. 
In this period he wrote some religious poems. 
Eventually Eliot decided to separate from his wife, who died in an asylum in 1947. 
Guilt haunted Eliot for her death. 
In those years, the poet had got closer to theatre, becoming one of the major exponents of poetic drama. 
In 1948 he won the Nobel Prize for Literature. 
He died in London and his ashes are buried in St Michael and All Angels Church in East Coker.

Before his conversion to Anglicanism his works were characterized by a pessimistic view of the world, while purification, hope, and joy are found after his conversion. 

\subsubsection{\emph{The Waste Land}}
The poem consists of five sectionn:
\begin{itemize}
    \item \textbf{The Burial of the Dead}, dealing with the coming of spring in a barren land; 
    \item \textbf{A Game of Chess}, which compares the present squalor to and ambiguous past splendor; 
    \item \textbf{The Fire Sermon}, where the theme of alienation is explored through a loveless sexual encounter; 
    \item \textbf{Death by Water}, about a drowned Phoenician sailor, Phlebas; 
    \item \textbf{What the Thunder Said}, which evokes religios from all over the world but ends in utter desolation. 
\end{itemize}

It deals with the contrast between the fertility of a mythical past and the present spiritual sterility. 
The fragmentation of the poem reflects the breakdown of the social and cultural order caused by the First World War. 
The present and the past exist simultaneously in \emph{The Waste Land}. 
Eliot used the technique of the \emph{object correlative} (cfr Montale), using a combination of objects to evoke the appropriate emotion. 

\subsubsection{\emph{The Love Song of J. Alfred Prufrock}}

\subsection{James Joyce (1882 - 1941)}

\subsubsection{Life and works}
James was born in Dublin and was educated at Jesuits schools before attending University College, where he gained a Bacelor of Arts with a focus on modern languages. 
He was interested in a broad European culture, leading him to think of himself as a European rather than an Irishman. 
Joyce believed that the only way to raise Ireland's awareness was to offer a realistic portrait of it from a cosmopolitan point of view. 
On the 16\textsuperscript{th} of June 1904 he had his first date with his would-be wife, Nora Barnacle. 
The following year the couple moved to Trieste, where Joyce began teaching English and befriended Italo Svevo. 
While in Trieste, he finished writing his masterpiece \emph{Dubliners} and published it in 1914. 
In 1915 Joyce and his family moved to Zurich to flee from the First World War. 
In 1920 they moved again to Paris, where Joyce was able to publish his novel \emph{Ulysses}. 
This work drew both high praise and strong criticism. 
In 1940 Joyce and his family moved back to Zurich as the nazis occupied France. 
He died as a result of an intestinal operation and is buried in Zurich. 

Joyce wanted to give a realistic portrait of the life of ordinary people in his work. 
He did so while challenging Catholicism and traditions. 
His style was full of symbols, free direct speech, and streams of consciousness. 

\subsubsection{\emph{Dubliners}}
\emph{Dubliners} consists of 15 short stories organized into 4 nuclei:
\begin{itemize}
    \item childhood;
    \item adolescence;
    \item maturity;
    \item public life;
\end{itemize}
The Dublin he describes is a place where true feelings and compassion for others do not exist, as city life degrades its citizens. 
Everyone in Dublin is oppressed by religious, political, cultural, and economic forces and everyone is in a state of \emph{paralysis}. 
The description of each story is realistic but full of symbolism. 
Joyce makes heavy use of the \emph{epiphany}, a trivial moment in one's life that causes a sudden revelation. 

\subsubsection{\emph{Ulysses}}

\subsection{Virginia Woolf (1882 - 1941)}

\subsubsection{Life and works}
Virginia Stephen was born in London, daughter of an eminent Victorian intellectual. 
Her education consisted in private Greek lessons and, most important, access to her father's library. 
She spent her summers in Cornwall and the sea remained a central theme in her art (cfr Montale). 
Her mother's death led to Virginia's first nervous breakdown and to her rebelling against her father's being aggressive and tyrannical. 
When her father died in 1904, Virginia felt free to pursue her own literary career. 
She became a member of the Bloomsbury Group, an important literary avant-garde that disdained Victorian traditions. 
In 1912 Virginia married Leonard Woolf and in 1915 she published her first novel, \emph{The Voyage Out}. 
During this period she attempted suicide by taking drugs. 
In 1925 she published \emph{Mrs Dalloway}, a novel in which she experimented with new narrative techniques. 
Woolf was also a talented literary critic and she published a volume of essays named \emph{The Common Reader}. 
She was also devoted to the feminist movement, delivering two lectures which then became \emph{A Room of One's Own}. 
As World War II broke out, her anxiety increased and she became haunted by the fear of losing her mind. 
She drowned herself in the River Ouse in 1941. 

Woolf was interested in giving voice to the inner world of her characters. 
What mattered for her were not the events that made up a story, but the impressions they left on who experienced them. 
In her novels, the point of view shifted between the different characters' minds. 

\subsubsection{\emph{Mrs Dalloway}}
In the morning, Clarissa Dalloway goes to buy some flowers for a party she is giving that same evening. 
Her attention shifts toward Septimus and Lucrezia Warren Smith, a war veteran suffering from shell shock and an Italian girl walking on the street. 
Septimus is affected by mental disorder. 
Clarissa goes back home and she is visited by Peter Walsh, a man she loved in her youth. 
Peter then goes to Regent's Park, where he sees Septimus and Lucrezia going to see a doctor. 
In the evening, Septimus jumps out of his window. 
Everyone is present at Clarissa's party. 
Clarissa hears of Septimus's death and she feels a strong connection with him. 

The novel, much like Joyce's \emph{Ulysses}, takes place on an ordinary day in June. 
\emph{Mrs Dalloway} is set in a small area of London. 
Through what she defines as \say{tunneling technique}, Woolf allows the reader to access her characters' recollection of their past, providing background and personal history 

The characters all belong to the upper-middle class. 
Clarissa is married to Richard Dalloway, a conservative Member of Parliament. 
She is characterized by opposing feelings: a need for freedom and independence and a class consciousness. 
Septimus Warren Smith is a young poet and an extremely sensitive man. 
He is prone to panicking and haunted by guilt for the death of his best friend Evans during World War I. 

The novel deals with the way people react to new situations and it provides insights on social changes at the time. 
Woolf makes use of cinematic devices such as close-ups, flashbacks, and tracking shots. 
She also adopts a motif: the striking of Big Ben and clocks in general, representing the passing of time and its flowing into death. 

Woolf makes heavy use of streams of consciousness, but unlike Joyce, she maintains a logical and grammatical structure. 
She also describes \emph{moments of being}, rare occasions of insight similar to Joyce's epiphanies. 

\subsection{George Orwell (1903 - 1950)}

\subsubsection{Life and works}
George Orwell was born Eric Blair in India, son of a colonial officer. 
When he was little, he was taken to England by his mother. 
He was educated at Eton College, where he could not stand the lack of privacy and the humiliating punishments. 
He grew indifferent to socially accepted values, developing an independent-minded personality. 
He was also an atheist and socialist. 
He became a member of the Indian Imperial Police but in 1927 he went on leave and never returned to \say{escape from every form of man's dominion over man}. 
Back in London, he started a social experiment to experience poverty first-hand. 
He began publishing his works in 1933, starting with \emph{Down and Out in Paris and London}, about his experiences among the poor, and \emph{Burmese Days}, about his service as a colonial officer. 
He married Eileen O'Shaughnessy in 1936, a like-minded socialist and intellectual. 
In the same year he investigated the living conditions to which miners were subject and wrote the report \emph{The Road to Wigan Pier}. 
In December 1936 he moved to Catalonia to report on the Spanish Civil War and fought in the trenches of the Aragon front with the Socialist militia. 
Back in England, George and his wife adopted a child and called him Richard.  
When World War II broke out, Orwell moved to London and joined the BBC as a speaker. 
In 1943 he resigned and became the editor of \emph{The Tribune}, a socialist weekly magazine. 
He also began writing \emph{Animal Farm} (published 1945) and \emph{Nineteen Eighty-four} (published 1949). 
Orwell died of tubercolosis in 1950. 

Thanks to his experiences abroad, Orwell was able to see his country from the outside and give a realistic account of its strenghts and weaknesses. 
He was also receptive to new ideas and impressions. 
He believed that a writer should be completely independet and that writing intepreted reality and hence served a useful social function. 
Orwell was inspired by Charles Dickens to write about social themes and to use a simple, factual language. 
He insisted on tolerance and justice and warned against the artificiality of urban civilization.

\subsubsection{\emph{Nineteen Eighty-four}}
The world is divided into three blocks: Oceania, Eurasia and Eastasia. 
Oceania is ruled by the Party, led by \emph{Big Brother}. 
The Party is creating a new language, \emph{Newspeak}, in order to control what people can say and think, also with the aid of \emph{Though Police}. 
Winstom Smith illegaly buys a diary and starts to write his thoughts and memories on it. 
His job is to rewrite history to bend it at the Party's will. 
At the \emph{Ministry of Truth} he meets an attractive girl called Julia. 
They begin a secret affair but are eventually discovered and arrested because of O'Brien, a Party spy. 
Winston is taken to the \emph{Ministry of Love}, where he is tortured ruthlessly for months. 
He has to face his biggest fear in \emph{room 101} and his will is completely broken. 
In the end, he meets Julia but he no longer loves her. 
He has given up on his identity and devoted himself completely to \emph{Big Brother}. 

The novel is set in a state of perpetual war akin to World War II. 
The idea for the three countries came to Orwell after the \emph{Tehran Conference} where Roosvelt, Churchill, and Stalin decided about the fate of the world after the war. 
The society reflects the political atmosphere present in totalitarism (cfr Arendt). 
Winston Smith represents a common Englishman. 
He is the last with a desire for spiritual and moral integrity. 
His main concern is the manipulation of history committed by the Party. 

\section{The Present Age}

\subsection{Samuel Beckett (1906 - 1989)}

\subsubsection{Life and works}
Samuel Beckett was born in Dublin, where he attended a boarding school and later Trinity College. 
After graduating in French and Italian, he moved to Paris where he taught English and came to know about existentialist philosophy. 
He also became friends with James Joyce and his literary circle. 
His works were mostly written in French and later traslated into English. 
He started writing short stories and novels and then settled onto theatrical plays. 
Beckett was part of a group of dramatists that developed the so-called \emph{Theatre of the Absurd}, in which man's life appears to be meaningless and purposeless. 
His play \emph{Waiting for Godot} (1952) was the first one in this style and it achieved immense success. 
He went on writing plays and he gained a Nobel Prize for Literature in 1969. 
He died in France in December 1989. 

\subsubsection{\emph{Waiting for Godot}}
Two tramps, Vladimir \say{Didi} and Estragon \say{Gogo}, meet by a leafless tree to wait for a misterious man called Godot. 
Godot sends a boy to inform them that he would not arrive that thay but will surely come the following day. 
The two bums are enraged and want to leave, but remain still. 
They also meet Pozzo and Lucky, a traveller and his slave. 
The following day, Didi and Gogo think about separation and even suicide but do not accomplish anything. 
The boy arrives to tell them that Godot cannot come, but he will surely come the following day. 
In the end, Vladimir and Estragon want to leave but they remain still as the scene fades to black. 

The play has no development in time: it has neither past nor future, only a meaningless and repetitive present. 
It has no setting either but a country road and a bare tree. 
What gives the play its unity is its symmetry: the stage is divided by the tree, humanity is split into two (Didi and Gogo). 
The characters' actions are also symmetrical. 

The characters are two humans concerned about the nature of the self, the world, and God. 
Vladimir is more practical, while Estragon is a dreamer. 
Estragon cannot remember anything about his past while Vladimir has some memories but he distrusts them. 
Both of them serve to remember the other of his own existence. 
Pozzo and Lucky are bonded to each other by a rope. 
Lucky is the slave and represents the power of the mind, while Pozzo is the oppressor and represents the power of the body. 

The language of the play is informal. 
The lack of communication is depicted through the use of para-verbal language such as pauses, silences, and gaps. 
Repeated phrases, lines, and words represent the senseless repetition and the relentless flow of time. 