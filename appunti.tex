\documentclass{book}
\usepackage[italian]{babel}
\usepackage{dirtytalk}
\usepackage[a4paper, hmargin=3.5cm, vmargin=4.0cm]{geometry}
\usepackage[skip=8pt]{parskip}
\usepackage{tabularx}
\usepackage{amsmath}
\usepackage{physics}
\usepackage{textcomp}
\usepackage{gensymb}
\usepackage{graphicx}
\usepackage{wrapfig}
\usepackage{esdiff}
\usepackage{qrcode}
\usepackage[hidelinks]{hyperref}

\title{\textbf{Appunti per l'esame di Maturità}}
\author{Matteo Zoppetti}
\date{}
\graphicspath{ {./immagini/} }

\begin{document}

\maketitle
\tableofcontents

\chapter*{Introduzione}
Questi appunti sono una raccolta incompleta, e forse anche piena di errori, di argomenti
trattati durante il mio quinto anno di liceo scientifico. Ogni miglioramento è ben accetto. 

Se questi appunti ti sono stati utili, considera di offrirmi un caffè :)

\begin{center}
    \begin{tabular}{ c c }
        \qrcode{https://paypal.me/mzoppetti} & \qrcode{https://revolut.me/mzoppetti} \\
        \href{https://paypal.me/mzoppetti}{PayPal} & \href{https://revolut.me/mzoppetti}{Revolut}
    \end{tabular}
\end{center}

\chapter{Latino}
\section{Seneca (4 a.C. - 65 d.C.)}

\subsection{La vita}
Lucio Anneo Seneca nacque a Cordova nel 4 a.C. 
Nella sua vita fu filosofo e letterato, ma anche un politico scaltro e spregiudicato.
Fu educato a Roma e nel 31 iniziò il suo cursus honorum diventando questore e successivamente senatore. 
La sua abilità gli attirò la gelosia di Caligola, che considerò di farlo uccidere, ma fu salvato dall'intervento di un'amica del principe.
Successivamente, Messalina, moglie del nuovo imperatore Claudio, lo fece esiliare in Corsica nel 41 con l'accusa di adulterio.
Per riconciliarsi con Claudio, durante l'esilio scriverà la \emph{Consolatio ad Polybium}, liberto dell'imperatore.
La sorte di Seneca cambiò nuovamente nel 49 quando Agrippina Minore, la nuova moglie di Claudio, convinse l'imperatore a richiamare a Roma il filosofo.
Così, Seneca fu nominato precettore di Nerone. 
Nel 54, alla morte di Claudio, Seneca scrisse una violenta opera satirica contro Claudio, l'\emph{Apokolokỳntosis}.
Per qualche anno dopo la salita al trono di Nerone, il governo era in mano ad Agrippina, Seneca ed Afranio Burro (il prefetto del pretorio), che governarono con prudenza e moderazione.
Questo periodo felice durò fino al 59, quando Nerone ordinò l'assassionio della madre Agrippina.
Il matricidio segnò un profondo cambiamento nei rapporti tra l'imperatore e il suo precettore, che iniziò ad essere emarginato assieme ad Afranio Burro.
Quest'ultimo morì in circostanze sospette nel 62 e, lo stesso anno, Seneca decise di ritirarsi a vita privata per dedicarsi ai suoi studi.
La vita di Seneca fu stroncata nel 65 quando, dopo la scoperta della \emph{congiura di Pisone}, Seneca fu accusato di complicità e fu costretto a suicidarsi.
Il raccontro della sua morte è narrato da Tacito negli \emph{Annales}.

\subsection{La filosofia}
Prima di parlare delle opere di Seneca, è opportuno evidenziare l'importanza del suo pensiero filosofico.
La sua grande originalità è dovuta al fatto che, nonostante sia evidente l'impianto stoico della sua filosofia, egli fu un pensatore \emph{asistematico}, capace di includere elementi di altre filosofie.
La filosofia per Seneca è una guida all'azione, un percorso di miglioramento continuo: egli stesso è ben cosciente di non essere il saggio che descrive nelle sue opere.
L'obiettivo del suo pensiero è giungere ad essere \emph{sapiens}, ovvero un uomo che non si lascia turbare dalle circostanze e non si abbandona alle passioni, vivendo con moderazione. 
Infatti, così come la natura è governata dal \emph{Lògos}, la mente divina che ordina l'universo, anche l'uomo deve vivere seguendo la stessa razionalità.
Un altro tema centrale nella riflessione di Seneca è l'importanza del \emph{tempo}: l'unico bene di cui disponiamo davvero e che quindi non può assolutamente venire sprecato. 

\subsection{Le opere}
Le opere di Seneca abbracciano numerosi generi e argomenti: dieci dei suoi scritti filosofici furono raccolti nei \emph{Dialogi}, a parte ci sono opere come il \emph{De beneficiis}, il \emph{De clementia} e, soprattutto, le \emph{Epistulae morales ad Lucilium}.
Seneca scrisse inoltre otto tragedie, tutte \emph{cothurnatae}. 
Sotto suo nome sono tràdite anche due tragedie considerate apocrife. 
Un altro opuscolo importante è, come già detto, l'\emph{Apokolokỳntosis}, una parodia satirica della divinizzazione di Claudio.
\subsubsection{Le \emph{Consolationes}}
All'interno dei Dialogi sono presenti tre \emph{Consolationes}, rivolte rispettivamente a Marcia per la perdita del figlio, alla propria madre Elvia per l'esilio, e al liberto Polibio.
Questi testi risalgono al periodo di esilio in Corsica, e sono a metà strada tra la retorica e l'ammonimento morale, pertanto risultano piene di luoghi comuni sulla sventura e sulla necessità di affrontarla con coraggio.
\subsubsection{Il controllo delle passioni}
Sempre nei \emph{Dialogi}, sono presenti le opere \emph{De ira}, \emph{De constantia sapientis}, \emph{De tranquillitate animi} e \emph{De vita beata}.
In queste opere Seneca affronta i temi dell'ira, che altro non è che una \emph{pazzia momentanea}, e della sapienza, intesa come capacità di non cedere all'ignoranza e agli impulsi. 
Inoltre, Seneca prescrive alcuni esercizi spirituali per avvicinarsi alla saggezza: frequentare uomini buoni, impegnarsi per il bene comune, praticare la moderazione e attendere serenamente la morte.
\subsubsection{De providentia}
Nel \emph{De providentia}, Seneca affronta il problema della sventura: perché gli uomini buoni possono essere afflitti da terribili sciagure?
La risposta è la seguente: le persone, attraverso le tragedie, possono dare prova delle proprie virtù e, se le sofferenze risultassero insopportabili, è possibile darsi la morte volontariamente.
\subsubsection{De brevitate vitae}
Come già detto, il tema del tempo è fondamentale nel pensiero di Seneca. 
Al contrario di ciò che suggerisce il titolo, Seneca giunge alla conclusione che la vita non è breve, ma è l'uomo a sprecarla in attività inutili, particolarmente gli \emph{occupati}.
Il senso della vita quindi non sta nella quantità, ma nella qualità del tempo che si vive.
Per questo, Seneca invita ad utilizzare con cura il proprio tempo e a non farsi dominare da esso.
\subsubsection{Gli scritti politici}
Seneca scrisse anche alcune opere di carattere politico, come il \emph{De clementia}, dedicato a Nerone, in cui tratta del buon governo, all'insegna della moderazione e della giustizia. 
Il \emph{princeps} inotre dovrebbe garantire la pace e la prosperità, ispirato dalla ragione.
Un altro scritto con un fine politico è il \emph{De otio}, in cui Seneca elogia la vita intellettuale come utile al miglioramento morale: aiutando se stesso, il sapiente aiuta indirettamente il resto dell'umanità.
\subsubsection{Le \emph{Epistulae morales ad Lucilium}}
L'ultima opera di Seneca, scritta tra il 62 e il 65, consiste in 124 lettere indirizzate all'amico Lucilio.
Le lettere che Seneca scrive sono però scritte soprattutto per se stesso e per i posteri: in queste è infatti possibile identificare un vero e proprio testamento spirituale del filosofo.
Partendo da episodi comuni e quotidiani, Seneca affronta svariate riflessioni filosofiche in cui è presete la sintesi di tutto il suo pensiero con parole semplici e dirette. 
Qui il filosofo ribadisce che il percorso morale e intellettuale di un uomo non ha termine, ma è un continuo cammino. 
Nelle lettere torna il tema della felicità, che coincide con la ragione perfetta, e il tema del tempo e della morte: dare un senso alla morte infatti significa dare un senso alla vita stessa. 
\subsubsection{Le tragedie}
Eventi sanguinosi e incredibilmente violenti raccontati con toni estremamente crudi sono il fulcro delle tragedie di Seneca, in cui si assiste a un ribaltamento totale dei valori stoici che il filosofo predica.
Di fatto, queste tragedie ritraggono l'opposto del sapiente senecano, dominato dai propri impulsi e senza freni morali. 
Un personaggio tipico di queste tragedie è il tiranno, un concentrato di qualsiasi vizio ed infamia, che doveva fungere da monito a Nerone.
Il modello da cui Seneca attinge è Euripide.
\subsubsection{L'\emph{Apokolokỳntosis}}
L'\emph{Apokolokỳntosis}, o la \emph{Zucchificazione di Claudio}, è una satira menippea scritta nel 54, subito dopo la morte di Claudio. 
In quest'opera, l'imperatore viene deriso per i propri tic e deformità. 
È quindi un'opera diffamatoria, scritta però con il favore della corte e, probabilmente, fu sia una vendetta contro l'imperatore, sia una giustificazione di Agrippina e Nerone, gli assassini di Claudio.

\subsection{Lo stile}
Lo stile di Seneca è stato definito \emph{post-classico}, per la sua distanza dall'equilibrio di altri autori come Cicerone.
Il filosofo utilizza infatti una prosa nervosa e movimentata, con frasi brevi e paratattiche, piena di \emph{variatio}.
Per convogliare meglio i propri pensieri, Seneca utilizza spesso un effetti patetici, personificazioni, domande retoriche e immagini.

\section{Lucano (39 - 65 d.C.)}

\subsection{La vita}
Marco Anneo Lucano nacque a Cordova nel 39 d.C.
Fu educato a Roma e ad Atene e, grazie al potere dello zio Seneca, entrò presto nella stretta cerchia di Nerone.
Nel 65 d.C., Lucano, spinto dai suoi ideali repubblicani, prese parte alla congiura di Pisone e fu quindi costretto a suicidarsi a meno di 26 anni.

\subsection{Il \emph{Bellum civile}}
L'unica opera (incompiuta) a noi pervenuta di Lucano è il \emph{Bellum civile}, un poema epico-storico che narra la guerra civile tra Cesare e Pompeo.
Le fonti a cui attinse per la propria opera furono probabilmente le opere di Tito Livio, Seneca il Retore e Asinio Pollione, ma nonostante ciò il poema non pretende di avere un'attendibilità storica.

Lucano, con il \emph{Bellum civile}, si distacca radicalmente dalla tradizione virgiliana, in particolare:
\begin{itemize}
    \item rifiuta ogni giustificazione provvidenziale della storia;
    \item non è presente alcun personaggio completamente positivo;
    \item gli dei non intervengono, sono indifferenti alle vicende umane.
\end{itemize}
L'elemento fondamentale del poema è la tragicità degli eventi narrati, guerre \emph{plus quam civilia}, poiché i due contendenti sono parte della stessa famiglia.
Lucano vuole denunciare il disordine del mondo, da lui identificato nel crollo delle istituzioni e della legalità repubblicane, a favore delle forze distruttive di Cesare. 

I protagonisti del poema sono Cesare e Pompeo, antitetici ma accomunati dalla loro volontà di instaurare un dominio assoluto a Roma.
Cesare è presentato come l'incarnazione del male, spinto unicamente da sete di potere. 
Capovolgendo Enea, la caratteristica distintiva di Cesare è la sua empietà, raccontata nel banchetto dopo la battaglia di Farsàlo.
Pompeo invece è il difensore delle istituzioni repubblicane, ma viene presentato come debole e incapace di opporsi a Cesare.
Dopo la morte di Pompeo, Catone il Giovane prende la guida dell'esercito repubblicano. 
Egli rappresenta i valori che hanno reso grande la Roma repubblicana, è un esempio di saggio stoico. 
Catone stesso riconosce la malvagità del fato e l'indifferenza degli dèi e si impegna per tentare di contrastarli.

\section{Petronio (27 - 66 d.C.)}

\subsection{La vita}
Petronio nacque intorno al 27 d.C. e morì nel 66 d.C., in seguito alla congiura di Pisone. 
L'unica opera che scrisse fu il romanzo \emph{Satyricon}.
Negli \emph{Annales}, Tacito dipinge un ritratto a doppia faccia di Petronio: \say{dedicava le ore del giorno al sonno, quelle della notte ai suoi doveri e alle gioie della vita}.
Nonostante fosse un uomo ozioso, era anche estremamente raffinato, tanto da essere considerato \emph{arbiter elegantiae} nella corte di Nerone. 
Petronio fu nominato anche proconsole della Bitinia e governò con grande abilità e saggezza. 
Tuttavia, questa vita condotta nell'ambiguità a cavallo tra vizi e virtù si scontrò con la gelosia di Tigellino, prefetto del pretorio, che lo accusò di aver preso parte alla congiura di Pisone. 
Petronio fu quindi costretto a suicidarsi, ma lo fece con la stessa stravaganza con cui aveva vissuto: 
\say{non si tolse la vita con precipitazione, ma, secondo il suo capriccio, si fece tagliare le vene, poi richiudere, poi aprire di nuovo, mentre conversava con gli amici} solo di argomenti leggeri.
Come ultimo atto, decise di denunciare per iscritto tutte le malefatte del \emph{princeps} e, dopo aver apposto il proprio sigillo, lo spezzò \say{perché non dovesse più tardi servire a provocare altre vittime}

\subsection{Il \emph{Satyricon}}
Il titolo dell'opera deriva dall'aggettivo greco \emph{satyrikòs}, ovvero \say{relativo ai satiri, satiresco}. 
Questo sarebbe un riferimento al carattere piccante e lascivo della narrazione, proprio come i \emph{sàtyroi}.
Un'altra possibile origine sarebbe da ricondursi alla \emph{satura menippea}, un genere letterario caratterizzato dall'uso del prosimetro.
Gli unici stralci a noi pervenuti sono relativi ai libri XIV, XV e XVI. 
Questi frammenti sono stati recuperati dai cosiddetti \emph{excerpta maiora} e \emph{minora}, a cui manca la cena di Trimalchione, rinventa da Marino Statilio nel 1654 nel \emph{codex Traguriensis}.

\subsubsection{La trama}
Il \emph{Satyricon} può essere letto come una sorta di parodia dell'\emph{Odissea}: un terzetto di giovani, Encolpio, Ascilto e Gitone, si imbarcano in un viaggio incalzati dall'ira del dio Priapo. 
La narrazione inizia in una \emph{graeca urbs}, probabilmente Cuma. 
Dopo alcune peripezie e un'orgia espiatoria, i tre protagonisti (Encolpio, Ascilto e Gìtone) vanno a cena da Trimalchione, un liberto ricchissimo ma rozzo. 
Di ritorno dalla cena, i tre conoscono il poetastro Eumolpo, che recita una \emph{Troiae halosis}. 
Questa è probabilmete una beffa nei confronti di Nerone, che anch'egli aveva composti la \emph{Presa di Troia}.
Ad ogni modo, dopo aver scoperto che Ascilto è un violento ed averlo abbandonato, Encolpio, Gìtone ed Eumolpo partono con una nave. 
Dopo alcune tensioni dovute a vecchie conoscenze, la nave naufraga nei pressi di Crotone, dove Eumolpo si fa passare per un ricco possidente per farsi offrire cene e regali dai cacciatori di eredità. 
Quando gli abitanti della città sospettano l'inganno, Eumolpo scrive il proprio testamento, nel quale dichiara che la sua eredità andrà a chi si ciberà della sua carne di fronte al popolo.
I frammenti del \emph{Satyricon} si interrompono qui, pertanto è impossibile conoscere la conclusione della storia.

\subsubsection{I personaggi}
Nessun personaggio del romanzo può essere considerato completamente positivo, Petronio racconta figure dei bassifondi della società.
Tuttavia, nella sua narrazione Petronio non lascia mai trasparire alcuna condanna morale, ma si colloca al di sopra delle vicende che narra con sguardo ironico e divertito.
Petronio vuole probabilmente raccontare tutti gli strati della società con le sue sfumature. 

\subsubsection{La \emph{Cena Trimalchionis}}
La Cena di Trimalchione è l'episodio più famoso del \emph{Satyricon} e racconta di un banchetto stravagante a casa di Trimalchione. 
Lo stesso nome Trimalchio è un nome parlante: potrebbe infatti essere tradotto come \say{tre volte arricchito}. 
La narrazione di banchetti filosofici era un tema ricorrente nella letteratura antica, basti pensare al \emph{Simposio} di Platone, ma Petronio ne presenta una parodia comica e grottesca. 
Il banchetto è esagerato, così come l'anfitrione della serata, è il trionfo del cattivo gusto e della volgarità di cui Trimalchione e i suoi amici liberti sono il simbolo. 
Il racconto di questo episodio è anche una critica da parte di un aristocratico nei confronti dei liberti arricchiti che stavano scalando le gerarchie sociale. 

\subsubsection{I generi del \emph{Satyricon}}
A primo acchito, il \emph{Satyricon} può essere inquadrato nel genere del romanzo: l'opera è effettivamente un rovesciamento del romanzo greco, fatto di trame stereotipate caratterizzate da amori contrastati, viaggi, avventure, intrecci complessi e separrazioni e ricongiungimenti. 
Tuttavia, all'interno di questo romanzo è possibile riconoscere anche l'influenza delle \emph{fabulae Milesiae}, una raccolta di racconti erotici (come quello della matrona di Efeso) scritti da Aristide di Mileto e tradotti da Cornelio Sisenna in latino. 
Inoltre, il ricorso al prosimetro rimanda alla \emph{satura Menippea}, dove la parodia si univa alla varietà di contenuti e toni.
Insomma, è impossibile inquadrare il \emph{Satyricon} in un unico genere letterario.
Trimalchione

\subsubsection{Lo stile}
Nella sua narrazione, Petronio mescola il comico e il grottesco, una deformazione esagerata di situazioni e personaggi. 
Il suo raccontare molte sfaccettature della società si rispecchia anche nel linguaggio usato: un vero e proprio pastiche di registri. 
Un importante merito di Petronio è la documentazione accurata del \emph{sermo vulgaris}, la lingua parlata dei ceti più bassi, che si evolverà nel volgare. 
A questo si alterna un linguaggio aulico tipico dei personaggi più colte, numerosi diminutivi tipici del linguaggio parlato e alcuni \emph{hapax legomena}, parole utilizzate una volta sola nella letteratura. 

\section{Tacito}

\subsection{La vita}
Tacito nacque nel 55 d.C., la sua terra natia è incerta, forse la Gallia Narbonense oppure Terni. 
Suo suocero era Giulio Agricola, generale romano che conquistò la Britannia, a cui Tacito dedicò un'opera. 
La sua fu un'eduazione eccellente e ben presto si dedicò all'avvocatura e alla politica. 
Fu sempre di ideali repubblicani, ma riconosceva che l'impero fosse ormai l'unica forma di governo possibile. 
Lui stesso ci racconta del suo \emph{cursus honorum}: \say{quanto agli onori della carriera, non potrei negare che Vespasiano li abbia inaugurati, Tito accresciuti, Domiziano spinti  ancora più in là}. 
Sotto Domiziano infatti, Tacito fu propretore e successivamente, sotto Nerva, fu nominato \emph{consul suffectus} nel 97, e infine proconsole in Asia nel 112. 
Tacito morì tra il 116 e il 120 d.C.

\subsection{Le opere}
Di Tacito ci sono pervenute tre opere conservate per intero, l'\emph{Agricola}, la \emph{Germania} e il \emph{Dialogus de oratoribus}. 
Le sue due opere più importanti invece, le \emph{Historiae} e gli \emph{Annales}, ci sono giunte incomplete. 

\subsubsection{L'\emph{Agricola}}
L'\emph{Agricola} è un'opera biografica scritta nel 98 d.C. e dedicata a Giulio Agricola, suocero di Tacito morto in circostanze sospette nel 93 d.C. 
L'opera non è tuttavia priva di scopo: questa è infatti un encomio alla figura di Agricola, attraverso cui loda indirettamente se stesso e la propria famiglia, allontanandosi dai sospetti di complicità con il regime di Domiziano. 
Agricola viene rappresentato come un \emph{vir bonus}, che incarnava l'antica \emph{virtus} romana e che era stato capace di mantenere la propria onestà e correttezza anche sotto il tirannico governo di Domiziano. 

Già dall'inizio, Tacito loda il cambiamento di regime: adesso, sotto Nerva, è stata ristabilita la libertà di pensiero e di espressione. 
Agricola nacque nel 40 d.C. nell'odierna Fréjus e, dopo aver intrapreso una carriera militare, fu nominato governatore della Britannia nel 77 d.C. 
Il generale riuscì a sottomettere la tribù ribelle dei Caledoni, contro cui Tacito ci racconta la battaglia finale, segnata da un drammatico discorso tenuto da Calgàco, il loro comandante. 
Prima di scendere in battaglia infatti, Calgàco prende la parola e critica ferocemente l'imperialismo romano, affermando che sono \say{i rapinatori del mondo}. 
Dopo questa vittoria, Domiziano richiamò Agricola a Roma, dove morì in circostanze misteriose nel 93 d.C.

\subsubsection{La \emph{Germania}}
La \emph{Germania} è un trattato etnografico scritto nel 98 d.C. (forse in occasione della spedizione in Germania di Traiano) che racconta le abitudini e i costumi delle popolazioni germaniche. 
Questo trattato è diviso in due parti, una (capitoli 1-27) prettamente etnografica e l'altra (capitoli 28-46) più geopolitica. 
Tacito utilizzò come fonti il \emph{De bello Gallico} di Giulio Cesare e i \emph{Bella Germaniae} di Plinio il Vecchio. 
Oltre alla descrizione dei popoli germanici, il problema affrontato da Tacito è capire perché il potere di Roma si sia bloccato di fronte a queste tribù. 
Insieme alla narrazione degli usi barbari quindi sono presenti numerose riflessioni etiche e morali che portano Tacito ad identificare in loro quelle stesse virtù che avevano caratterizzato gli antichi Romani. 

In tempi più recenti, la \emph{Germania} di Tacito è stata utilizzata a supporto della teoria razziale tedesca. 
Infatti, l'inglese tedeschizzato H.S. Chamberlain, propose una traduzione incentrata sul travisamento intenzionale di un \emph{tamquam} nel capitolo 4. 
Questa piccola variazione modifica radicalmente la frase: da \say{Hanno anche le stesse caratteristiche fisiche, per quanto possibile in un numero così grande di persone} essa diventa \say{Hanno anche le stesse caratteristiche fisiche, nonostante il numero così grande di persone}. 
La variazione è stata utilizzata per supportare la presunta purezza della razza ariana. 
È da notare però che, nonostante il rispetto che Tacito provava per le virtù dei Germani, l'autore non intendeva esaltarli, riconoscendo in essi un popolo rozzo, ozioso e dedito a vizi come il bere e il gioco d'azzardo. 

\subsubsection{Le \emph{Historiae}}
Le \emph{Historiae} raccontano il periodo che va dal 69 d.C., il cosiddetto anno dei quattro imperatori, fino al 96 d.C., anno della morte di Domiziano e quindi della fine della dinastia flavia. 
Erano composte da quattordici libri, di cui tuttavia disponiamo solo dei primi quattro e i primi ventisei capitoli del quinto. 
L'opera è di impianto annalistico e, dopo aver tracciato un quadro dei fatti che intende narrare, Tacito racconta della nomina di Galba a imperatore. 
Galba è però un vecchio fragile e non riesce a mantenere il potere, cadendo vittima di una congiura ordita da Otone, che prese il trono. 
Ma il suo regno non era destinato a durare: le legioni della Germania avevano acclamato Vitellio imperatore. 
Seguì una feroce guerra civile tra le armate di Otone e quelle di Vitellio, in cui quest'ultimo ebbe la meglio con una battaglia nella valle del Po. 
Otone preferisce suicidarsi piuttosto che cadere nelle mani del nemico. 
Il potere di Vitellio non fu mai accettato da tutto l'esercito, infatti alcune legioni avevano acclmanato imperatore il generale Vespasiano, che in quel periodo stava sedando la rivolta degli Ebrei in Giudea. 
Le legioni di Vespasiano arrivarono in Italia e sconfissero le armate di Vitellio nello stesso luogo in cui esse avevano sconfitto Otone. 
Quando l'esercito di Vespasiano giunse a Roma, Vitellio fu portato in giro per la città e infine linciato dal popolo. 
Dal principato di Vespasiano, la narrazione passa a due rivolte che stavano minacciando l'impero: quella dei Batavi e quella dei Giudei. 
Si apre quindi un \emph{excursus} sugli usi e costumi della civiltà giudaica imbevuto di antisemitismo. 

\subsubsection{Gli \emph{Annales}}
Gli \emph{Annales} raccontano in sedici libri gli anni degli imperatori compresi tra la morte di Augusto e quella di Nerone. 
Anche quest'opera, al pari delle \emph{Historiae}, segue un impianto annalistico. 
I libri tramandati fino ad oggi riguardano il principato di Tiberio, parte di quello di Claudio e quasi tutto quello di Nerone. 

Tiberio era stato nominato erede da Augusto, era il miglior generale della sua epoca ma aveva un carattere cupo e sospettoso. 
Tacito dipinge il progressivo declino morale dell'imperatore verso la follia, che lo spinge a diventare un tiranno. 
Contrapposto al personaggio di Tiberio c'è l'eroe Germanico, generale che conduceva la guerra in Germania. 
La guerra raccontata nei libri I e II fu condotta per vendicare la disfatta di Quintilio Varo, che nel 9 d.C. era stato sconfitto dai Germani nella selva di Teutoburgo. 
Germanico stesso visiterà questa selva in un episodio commovente. 
Proprio quando pareva che Germanico stesse per domare tutta la Germania, Tiberio lo richiamò a Roma, geloso delle sue imprese. 
Germanico fu quindi inviato in Oriente, dove morì (forse avvelenato). 
Nei libri III e IV si racconta invece di Seiano, spietato prefetto del pretorio. 
Questi avviò un violento periodo di repressione del dissenso e, nel 27 d.C., Tiberio si ritirò a Capri, lasciando di fatto tutto il potere nelle mani di Seiano. 
Accortosi del pericolo rappresentato dal prefetto, decide di farlo uccidere ma questa parte della storia non ci è pervenuta. 
Della morte di Tiberio si racconta che una sera egli sembrò esalare l'ultimo respiro. 
Mentre il futuro Caligola si stava appropriando delle insegne imperiali, però, l'imperatore si risvegliò chiedendo del cibo. 
Macrone risolve la situazione tesa ordinando di soffocare Tiberio sotto un cumulo di panni. 
Ecco che l'imperatore Tiberio in morte conservò la \emph{dissimulatio} che l'aveva caratterizzato in vita. 

Non ci è pervenuto il racconto del principato di Caligola. 
Quando la narrazione riprende, è imperatore Claudio, fratello di Germanico.  
Egli non fu un cattivo imperatore ma Tacito lo giudica comunque con severità per via della sua insicurezza. 
Claudio venne sedotto da Agrippina Minore, che riuscì a contrarre matrimonio con lui. 
Questa riuscì a convincere Claudio ad adottare Nerone e nominarlo proprio erede al trono e, in seguito, avvelenò il marito. 

Il principato di Nerone fu costellato di delitti: egli fece uccidere Britannico e la madre Agrippina. 
L'imperatore non ha alcun freno morale, si affidò al prefetto del pretorio Tigellino e dilapidò le casse statali per compiacersi la plebe. 
Dopo l'incendio di Roma del 64 d.C., Nerone incolpò la comunità cristiana, di cui si raccontano le prime persecuzioni. 
Gli ultimi frammenti degli \emph{Annales} riguardano la congiura di Pisone, che costò la vita a figure come Seneca, Lucano e Petronio. 

\subsection{Gli ideali}
Tacito scrive la storia \emph{perché le virtù non siano passate sotto silenzio}. 
Riconosce quindi la precarietà della vita umana e l'affermarsi sempre più prepotente della smania di potere. 
La visione di Tacito è quindi tragicamente pessimista. 
Nell'\emph{incipit} degli \emph{Annales}, Tacito afferma l'imparzialità di ciò che scrive, criticando i suoi predecessori che si erano abbandonati all'ignoranza o al servilismo. 
Tuttavia, Tacito stesso rappresenta una prospettiva parziale, quella senatoria: per esempio egli non riconosce a Tiberio il merito di un'amministrazione oculata o a Claudio quello di saggio legislatore. 
Tacito considerava il senato come fondamentale all'interno delle istituzioni romane, ma tuttavia riconosceva impossibile un ritorno alla repubblica e inevitabile un governo assoluto imperiale. 
Di conseguenza, egli auspicava un buon principe, capace di far coesistere le libertà personali con la stabilità di potere. 
Per questo motivo, egli scelse come protagonisti della sua storiografia singoli personaggi straordinari, perché la storia era ormai decisa da pochi individui. 
Nelle sue opere, Tacito è stato fortemente influenzato da Sallustio, riprendendo da lui: 
\begin{itemize}
    \item la concezione pessimista della natura umana; 
    \item l'intento moralistico delle opere; 
    \item la centralità del singolo nel racconto; 
    \item l'indagine psicologica che evidenzia vizi e virtù dei personaggi; 
    \item l'\emph{inconcinnitas} della prosa. 
\end{itemize}

\section{Apuleio}

\subsection{La vita}
Apuleio nacque attorno al 125 d.C. a Madaura, nell'attuale Algeria. 
Della sua biografia sappiamo ben poco, principalmente da ciò che lui stesso racconta, soprattutto nella sua \emph{Apologia} e nei \emph{Florida}. 
Apuleio apparteneva a una famiglia agiata, grazie alla quale potè studiare a Cartagine e successivamente ad Atene. 
Nella sua vita si dedicò a numerosi culti misterici, come quello di Asclepio, di Demetra e successivamente di Iside e Osiride. 
Nell'\emph{Apologia} possiamo scoprire alcune informazioni sulla vita privata di Apuleio. 
Infatti, nel 155 d.C., Ponziano, amico di Apuleio, lo convinse a sposare la madre Pudentilla (di recente rimasta vedova) per metterla al riparo dai cacciatori di eredità. 
Poco dopo il matrimonio tuttavia Ponziano morì e Apuleio fu accusato dal resto della famiglia di aver praticato stregoneria per sedurre Pudentilla e farsi designare da lei come unico erede. 
L'\emph{Apologia} fu il suo discorso di difesa ed era articolata in tre parti in cui rispettivamente delegittimava l'accusa, spiegava i due tipi di magia esistenti (nera e bianca) e infine rivelava che nel testamento non era stato scelto come erede. 
Ritornò poi a Cartagine, dove morì dopo il 170 d.C.

\subsection{Le opere}
Apuleio fu un autore estremamente prolifico sia in greco che in latino, tuttavia le opere giunte a noi sono scarse. 
Tra le opere che si sono conservate, oltre ai \emph{Florida} e all'\emph{Apologia}, possiamo nominarne alcune a carattere filosofico: 
\begin{itemize}
    \item \emph{De mundo}, in cui tratta di questioni cosmologiche e teologiche; 
    \item \emph{De Platone ed eius dogmate}, in cui racconta la biografia di Platone e il pensiero del filosofo; 
    \item \emph{De deo Socratis}, in cui spiega il ruolo dei dèmoni, mediatori tra uomini e dei. 
\end{itemize}
In questi testi Apuleio si fa portavoce della filosofia del \say{medio platonismo}, una fusione tra elementi platonici e aristotelici. 
Secondo questa teoria, il cosmo è diviso in una sfera divina, caratterizzata dalla razionalità, e in una sfera umana, caratterizzata dalla passionalità. 
Le due sarebbero messe in comunicazione dai dèmoni, immortali come gli dèi ma passionali come gli uomini. 

\subsubsection{Le \emph{Metamorfosi} (o \emph{L'asino d'oro})}
Il capolavoro di Apuleio furono senza dubbio le \emph{Metamorfosi}, un romanzo scritto in undici libri probabilmente dopo il 155 d.C. 
La vicenda è raccontata in prima persona da Lucio, protagonista dell'opera.

Nei primi tre libri, il giovane Lucio si presenta e racconta del suo viaggio di affari a Ipata, in Tessaglia (una terra nota per la magia). 
Già dall'inizio del romanzo, è evidente la caratteristica che sarà il motore dell'azione: la \emph{curiositas} del protagonista. 
Durante il viaggio viene messo in guardia circa la pericolosità delle stregonerie che avvengono a Ipata, ma Lucio non si cura degli ammonimenti. 
In città viene ospitato da Milone e da sua moglie Pànfile. 
Lucio intreccia una relazione con Fòtide, serva di Milone e Panfile, grazie alla quale può osservare la padrona di casa trasformarsi in uccello per incontrare i propri amanti. 
Esterrefatto, Lucio vuole provare l'incantesimo su di sé ma per un errore viene trasformato in un asino. 
Per ritornare uomo, gli basterebbe mangiare delle rose contenute in giardino, ma durante la notte viene rubato: dovrà faticare a lungo prima di ottenere l'agognato ritorno all'umanità. 

All'interno del romanzo viene raccontata la vicenda di Amore e Psiche. 
Psiche è una ragazza talmente bella da suscitare l'invidia di Venere, che invia suo figlio Amore a farla innamorare di un mostro. 
Tuttavia, lo stesso Amore si invaghisce della giovane e i due intraprendono una relazione amorosa che sarebbe continuata a patto che Psiche non conoscesse l'identità del suo amante. 
Una sera, spinta dalla \emph{curiositas}, Psiche cerca di spiare Amore nel sonno. 
Cupido però si sveglia per via di una goccia di olio bollente e si allontana immediatamente. 
Dopo numerose peripezie, Amore e Psiche potranno finalmente sposarsi. 

Finito questo racconto nel racconto, riprendono le peripezie di Lucio, che passa di mano in mano a svariati proprietari. 
Dopo essere riuscito a fuggire, Lucio vede in sogno la dea Iside, che gli fornisce istruzioni per ritornare umano in occasione di una celebrazione religiosa nel corso della quale riuscirà finalmente a mangiare le fatidiche rose. 
Ritornato uomo, Lucio diviene sacerdote di Osiride e si dedica con successo all'avvocatura. 

Gli argomenti principali della storia narrata e dei vari racconti nel racconto sono la magia, l'adulterio, l'inganno e l'omicidio. 
Il motore dell'azione è la \emph{curiositas}, che non è solo rovinosa ma funge anche da consolazione a Lucio-asino. 
Un problema aperto delle \emph{Metamorfosi} è l'intervento della dea Iside nel libro XI. 
Quest'ultima parte risulta infatti discorde con il resto del romanzo, le due interpretazioni che ne seguono sono opposte:
\begin{itemize}
    \item secondo alcuni, l'ultimo libro è solo un'appendice seria per dare credibilità a un testo altrimenti completamente frivolo; 
    \item altri ritengono invece che l'ultimo libro conterrebbe la chiave di lettura di tutto il romanzo, che sarebbe un'allegoria al percorso di iniziazione ai misteri isiaci (molto in voga all'epoca perché proponevano una vita dopo la morte). 
\end{itemize}

Gli obiettivi principale di Apuleio con il suo romanzo erano \emph{delectare} il pubblico e al contempo \emph{docere} qualosa. 
Le \emph{Metamorfosi} possono essere inoltre inserite nel filone delle \emph{fabulae Milesiae}. 
Apuleio potrebbe inoltre aver preso spunto da un'opera di Luciano di Samosata (la cui trama di fondo è la stessa delle \emph{Metamorfosi}, ma meno complessa) che a sua volta andrebbe connesso alle opere di Lucio di Patre, ma i rapporti tra questi tre testi non sono ancora chiari. 

%\chapter{Italiano}
%\section{Giacomo Leopardi}
\section{La scapigliatura}
\section{Giovanni Verga}
\section{Giovanni Pascoli}
\section{Gabriele d'Annunzio}
\section{Luigi Pirandello}
\section{Italo Svevo}
\section{I crepuscolari}
\section{I futuristi}
\section{Giuseppe Ungaretti}
\section{Umberto Saba}
\section{Eugenio Montale}

\section{Carlo Emilio Gadda}

\subsection{La vita}
Carlo Emilio Gadda nacque a Milano nel 1983. 
Fin da giovane era attirato dagli studi letterari ma dovette dedicarsi all'ingegneria per via delle difficoltà economiche della famiglia. 
Nel 1915 venne arruolato nell'esercito per combattere la Prima Guerra Mondiale. 
In guerra morì suo fratello, questo evento scuoterà fortemente i rapporti familiari di Gadda fino a farlo approdare a una forte misoginia e infine alla misantropia. 

\subsection{Le opere}


\section{L'ermetismo}
%\section{Spazio e tempo}
%\subsection{in Leopardi}
%contro il progresso, pessimismo storico? 
%\subsection{in Verga}
%contro il progresso
%\subsection{nel futurismo}

%\section{La guerra}
%\subsection{in Leopardi}
%social catena, la ginestra
%\subsection{in Ungaretti}
%fratellanza tra tutti gli uomini
%\subsection{in Montale}
%la primavera Hitleriana, Clizia
%\subsection{in Svevo}
%fine de "La coscienza di Zeno"
%\subsection{nel futurismo}
%esaltazione della guerra
%\subsection{in d'Annunzio}
%esaltazione della guerra 

%\section{Scienza ed etica}
%\subsection{in Gadda}
%gnomero, impossibilità di districare la realtà

%\section{Democrazie e totalitarismi}
%\subsection{in Montale}
%la primavera Hitleriana, Clizia
%\subsection{in Gadda}
%parapagal, la cognizione del dolore

%\section{Ambiente e risorseeee}
%\subsection{in d'Annunzio}
%panismo estetizzante, c'era qualcun altro che parlava di panismo, forse montale? 

%\section{Salute e malattia}
%\subsection{nella scapigliatura}
%gusto dell'orrido
%\subsection{in Pirandello}
%la pazzia
%\subsection{in Svevo}
%la psicanalisi

\chapter{Filosofia}
\section{Arthur Schopenhauer (1788 - 1860)}
Schopenhauer fu capace di sintetizzare numerosi spunti filosofici e spirituali diversi, tra cui: 
\begin{itemize}
    \item la teoria delle idee di Platone; 
    \item l'impostazione soggettivistica della gnoseologia kantiana; 
    \item il filone materialistico illuminista;
    \item il tema del dolore romantico. 
\end{itemize}

\subsection{Il mondo come volontà e rappresentazione}
Il punto di partenza della filosofia di Schopenhauer è la distinzione kantiana tra \emph{fenomeno} e \emph{noumeno}, ovvero tra la cosa \say{come appare} e \say{in sé}. 
Secondo Schopenhauer, il fenomeno è solo un'illusione, che descrive con l'immagine orientale del \emph{velo di Maya}, mentre il noumeno è ciò che si nasconde dietro. 
Il fenomeno di cui parla Schopenhauer esiste solo in quanto \emph{rappresentazione} soggettiva all'interno di una coscienza. 
Il soggetto può conoscere il mondo per mezzo di tre forme a priori: 
\begin{itemize}
    \item spazio; 
    \item tempo; 
    \item causalità.
\end{itemize}
le quali sono come delle lenti attraverso cui si osservano le cose. 
Esiste però una realtà oltre il fenomeno su cui solo l'uomo, in quanto \say{animale metafisico}, si interroga. 

Vivendo il corpo dal suo interno possiamo giungere alla conclusione che l'essenza profonda del nostro io è la \emph{volontà} di vivere, che si manifesta esteriormente in tutti i nostri organi. 
Ma poiché vivendo il proprio corpo si smette di usare le forme a priori della conoscenza, l'esperienza della volontà non è più individuale ma risulta l'essenza della realtà stessa. 
La volontà che governa la realtà ha alcune caratteristiche: 
\begin{itemize}
    \item è inconscia;
    \item è unica;
    \item è eterna;
    \item è incausata;
    \item è senza scopo.
\end{itemize}

Essendo manifestazione di una volontà infinita, secondo Schopenhauer la vita è intrinsecamente dolore. 
Infatti, la volontà comporta desiderio che, per definizione, significa privazione e mancanza di qualcosa. 
Inoltre, il piacere che ogni tanto gli uomini provano non è altro che una cessazione temporanea del dolore che caratterizza la vita. 
Quando viene meno il dolore causato dal desiderio subentra la noia: secondo il filosofo quindi la vita è come un pendolo che oscilla incessantemente tra il dolore e la noia, passando attraverso l'intervallo fugace del piacere. 
Come ci si può liberare da questo dolore? 
Secondo Schopenhauer, fuggire la vita con il suicidio non è la risposta corretta perché: 
\begin{itemize}
    \item il suicidio è un atto di affermazione della volontà 
    \item il suicidio non scalfisce la volontà universale, ma solo una sua manifestazione fenomenica
\end{itemize}
Liberarsi dal dolore della vita significa quindi liberarsi dalla volontà stessa di vivere, trasformare in \emph{noluntas} la \emph{voluntas} attraverso l'arte, la morale e l'ascesi. 

L'arte, in quanto conoscenza libera e disinteressata delle idee, sottrae l'individuo ai desideri quotidiani e permette all'uomo di contemplare la vita, ovvero di iniziare quell'iter che permette di annullare la volontà di vivere. 
L'arte è però un conforto fugace, temporaneo e parziale. 
Per liberarsi davvero dal dolore della vita bisogna quindi intraprendere il sentiero della morale e dell'ascesi. 
La morale secondo Schopenhauer implica un impegno di compassione nei confronti del prossimo, ovvero avvertire come proprie le sofferenze degli altri. 
La morale si concretizza in: 
\begin{itemize} 
    \item giustizia, ovvero il non fare il male agli altri 
    \item carità, ovvero il fare del bene al prossimo 
\end{itemize} 
La morale però presuppone comunque un attaccamento alla vita, pertanto l'unico vero modo di terminare il dolore si identifica con l'ascesi. 
L'ascesi consiste nella rinuncia ai piaceri, nell'umiltà, nel digiuno, nella povertà, nel sacrificio e nell'automacerazione ed è l'unica possibilità di sopprimere la volontà di vivere. 

\section{Karl Marx (1818 - 1883)}
La filosofia di Marx si propose come un'analisi globale della società e della storia. 
Sarebbe infatti riduttivo marchiarlo unicamente come filosofo in quanto le sue idee influenzarono economia, filosofia, storia, teoria del diritto e dello Stato e oltre. 
Le influenze culturali alla base del marxismo furono:
\begin{itemize}
    \item la filosofia classica tedesca, da Hegel a Feuerbach;
    \item l'economia politica borghese, da Adam Smith a Ricardo;
    \item il pensiero socialista.
\end{itemize}

La filosofia di Marx prende un forte spunto da quella di Hegel, seppur muovendo sin da subito pesanti critiche. 
Da Hegel riprende infatti il processo dialettico con cui considera la storia, ma contesta in lui quello che è un \say{misticismo logico}. 
Secondo Marx infatti, Hegel avrebbe reso il concreto come la manifestazione dell'astratto dopo aver desunto l'astratto dal concreto. 
Oltre ad essere fallace dal punto di vista razionale, la filosofia di Hegel tende anche ad essere politicamente conservatrice perché rende manifestazioni necessarie dello Spirito quelli che sono in realtà dei semplici dati di fatto. 

Alla base dell'ideale di Marx è presente la critica della civiltà moderna e dello Stato liberale. 
Secondo Marx infatti, si è verificata una forte frattra tra la società civile e lo Stato. 
Quest'ultimo non è che uno strumento in mano alle classi dominanti per affermare il proprio potere e, proclamando l'uguaglianza \emph{formale} dei propri cittadini di fronte alla legge non fa altro che ratificare la loro disuguaglianza \emph{sostanziale}. 
Il sistema borghese sarebbe una società del \emph{bellum omnium contra omnes}, come aveva già affermato Hegel, in cui Marx identifica come caratteristiche fondamentali l'individualismo e l'atomismo. 
Quello che il filosofo auspica è una compenetrazione perfetta tra stato e società, in una democrazia \say{totale} che garantisca l'uguaglianza sostanziale eliminando il fondamento di ogni disuguaglianza: la proprietà privata dei mezzi di produzione. 

L'economia borghese, secondo Marx, ha il grande difetto di \say{eternizzare} il sistema capitalistico, considerandolo \emph{il} e non \emph{un} sistema economico. 
L'economia borghese inoltre non è in grado di scorgere le conflittualità che caratterizzano il capitalismo, le scissioni che Marx (riprendendo Feuerbach) chiama \say{alienazione}. 
Sempre riprendendo Feuerbach, secondo Marx l'alienazione è una condizione patologica di scissione, dipendenza ed estraniazione. 
Secondo Marx, l'operaio è alienato: 
\begin{itemize}
    \item \textbf{rispetto al prodotto} della sua attività, in quanto egli produce un oggetto che non gli appartiene; 
    \item \textbf{rispetto all'attività} stessa, in quanto viene vista come un lavoro forzato; 
    \item \textbf{rispetto alla propria essenza} di essere umano, che consiste nel lavoro libero, creativo e universale; 
    \item \textbf{rispetto al prossimo}, che egli vede come il capitalista sfruttatore. 
\end{itemize}
La causa dell'alienazione è la proprietà privata dei mezzi di produzione, che permettono che il capitalista prevarichi sugli operai. 
La disalienazione starebbe quindi nel superamento della proprietà privata dei mezzi di produzione con l'avvento del comunismo. 

Nella sua analisi storica, Marx innanzitutto riconosce l'esistenza delle \say{ideologie}, ovvero false rappresentazioni della realtà. 
Marx quindi identifica come chiave del processo storico un processo materiale (da cui il nome della sua filosofia, detta \say{materialismo storico}) fondato sulla dialettica bisogno-soddisfacimento. 
Alla base della storia ci sarebbe quindi il lavoro, creatore di civiltà e cultura nonché ciò che differenzia l'uomo dalle bestie. 

Le varie società storiche sono caratterizzate da certi \emph{modi di produzione}, a loro volta costituiti da \emph{forze produttive} (forza-lavoro, mezzi di produzione e conoscenze tecniche) e \emph{rapporti di produzione} che si instaurano tra gli uomini. 
L'insieme dei rapporti di produzione costituirebbe, secondo Marx, la \emph{struttura} di una data società sopra la quale viene costruita una \emph{sovrastruttura} di leggi, politica, etica, arte, religione e filosofia. 
Questi quindi non sono realtà a sé stanti ma espressioni dei rapporti economici di una certa società storica. 
Marx quindi svela che le vere forze motrici della Storia sono unicamente di natura economica. 

Per studiare il capitalismo, Marx analizza merce, lavoro e plusvalore. 
Una merce deve possedere un \say{valore d'uso} (ovvero un'utilità pratica) e un \say{valore di scambio} che le permetta di essere scambiata con altre merci. 
Il valore di scambio di una merce dovrebbe derivare dalla quantità di lavoro necessaria per produrre la merce. 
Marx contesta quindi il \say{feticismo delle merci}, che consiste nel considerare una merce come avente valore per sé. 

La caratteristica del capitalismo è la produzione finalizzata all'accumulo di denaro e non finalizzata al consumo (schema Denaro - Merce - Denaro). 
Ma da dove arriva questo plusvalore? 
Il capitalista compra la forza-lavoro come qualsiasi merce, ma l'operaio ha la capacità di produrre più di quanto gli sia corrisposto come salario. 
Da questo plusvalore derivera il profitto del capitalista. 
La società capitalista è quindi fondata sullo sfruttamento e sulla logica del profitto privato. 

Le contraddizioni della società borghese sono la base per teorizzare la \emph{rivoluzione proletaria}, per attuare il passaggio dal capitalismo al comunismo. 
Dopo aver abbattuto lo Stato borghese e socializzato i mezzi di produzione, avverrà necessariamente la \emph{dittatura del proletariato}, una fase di transizione per giungere al comunismo autentico. 
Il comunismo autentico infatti, si avrà solo quando l'uomo cesserà di intrattenere con il mondo rapporti di puro possesso e consumo, in una società per cui \say{ognuno secondo le sue capacità, a ognuno secondo i suoi bisogni}. 

\section{Friederich Nietzsche (1844 - 1900)}
Il pensiero di Nietzsche fu influenzato dalla lettura de \say{Il mondo come volontà e rappresentazione} di Schopenhauer. 
Il percorso di studi del filosofo fu inizialmente improntato alla filologia per poi rivolgersi alla filosofia. 
Nei suoi scritti Nietzsche attacca ferocemente la cultura occidentale, distruggendo tutte le certezze del passato e proponendo un nuovo tipo di umanità. 
Il suo stile di scrittura fu perlopiù aforistico e il suo pensiero asistematico. 

\subsection{Il periodo giovanile}

\subsubsection{La nascita della tragedia}
Il tema centrale di questo scritto, a metà tra la filologia e la filosofia, è la distinzione tra \emph{apollineo} e \emph{dionisiaco}. 
Con questi due termini Nietzsche intende indicare i due impulsi di base identificabili nella tragedia greca: 
\begin{itemize}
    \item l'\textbf{apollineo}, un atteggiamento di razionalità ed equilibrio; 
    \item il \textbf{dionisiaco}, un atteggiamento di irrazionalità e caos.
\end{itemize}
Secondo Nietzsche, la grande tragedia nasce dalla fusione perfetta di questi due elementi (per esempio in Eschilo e Sofocle). 
Questa sintesi viene meno quando inizia a prevalere l'apollineo, come nelle tragedie di Euripide. 
La prevaricazione dell'apollineo sul dionisiaco fu da attribuire all'insegnamento razionalistico di Socrate, in cui Nietzsche identifica l'inizio della decadenza della civiltà occidentale. 

\subsubsection{Le \emph{Considerazioni inattuali}}
Nella seconda delle quattro \emph{Considerazioni inattuali}, Nietzsche si schiera apertamente contro lo storicismo. 
Infatti nella vita è necessario il \say{fattore oblio} perché:
\begin{itemize}
    \item senza incoscienza non c'è felicità;
    \item per poter agire nel presente occorre saper dimenticare il passato.
\end{itemize}
Questo però non significa che la storia sia sempre nociva per la vita. 
Il filosofo infatti identifica tre modi di rapportarsi con la storia: 
\begin{itemize}
    \item la storia \textbf{monumentale}, tipica di chi guarda al passato alla ricerca di modelli;
    \item la storia \textbf{antiquaria}, tipica di chi guarda al passato con fedeltà e amore;
    \item la storia \textbf{critica}, tipica di chi guarda al passato come un peso di cui liberarsi.
\end{itemize}
Ognuno di questi tre tipi di storia può essere valido a patto che non venga utilizzato esclusivamente. 

\subsection{La filosofia del mattino}
In questo periodo, Nietzsche adotta un metodo critico e genealogico, ovvero eleva il sospetto a metodo di indagine e ricerca i processi all'origine di realtà etiche e metafisiche. 

Secondo Nietzsche, Dio è sostanzialmente: 
\begin{itemize}
    \item il simbolo di ogni trascendenza oltre a questo mondo; 
    \item la personificazione delle certezze dell'umanità. 
\end{itemize}
Dio e l'aldilà hanno sempre rappresentato una fuga dell'uomo di fronte alla vita. 
L'immagine di un Dio benevolo e di un cosmo ordinato sono solo costruzioni della mente per sopportare la durezza dell'esistenza. 
Le metafisiche e le religioni quindi non sono altro che menzogne millenarie da smascherare e distruggere. 
Nella \emph{Gaia scienza} il filosofo narra la \emph{morte di Dio} attraverso il racconto dell'\emph{uomo folle}, un profeta che va al mercato ad annunciare che Dio è morto e \say{siamo stati noi ad ucciderlo}. 
La morte di Dio provoca nell'uomo, non ancora pronto per questo annuncio, un senso di vertigine e smarrimento che può sopportare solo facendosi \emph{superuomo}. 
Infatti, solo chi ha il coraggio di guardare in faccia la realtà e il crollo delle certezze può compiere quel salto che separa l'uomo dal superuomo. 

\subsection{La filosofia del meriggio}
Nell'opera \emph{Così parlò Zarathustra}, il filosofo affronta tre temi principali:
\begin{itemize}
    \item il \textbf{superuomo};
    \item la \textbf{volontà di potenza}; 
    \item l'\textbf{eterno ritorno}.
\end{itemize}

\subsubsection{Il superuomo}
Il superuomo (\emph{Übermensch}) è colui che è in grado di accettare la dimensione tragica e dionisiaca dell'esistenza. 
È da notare che \emph{Übermensch} può essere tradotto anche con \say{oltreuomo}, indicando non un \say{supereroe}, un uomo potenziato, ma un uomo che supera i limiti dell'umanità. 
Nel discorso delle tre metamorfosi, Zarathustra elenca le tre metamorfosi a cui lo spirito deve sottoporsi per diventare oltreuomo:
\begin{itemize}
    \item il \textbf{cammello}, che sopporta i pesi della tradizione;
    \item il \textbf{leone}, che si libera dai fardelli metafisici ed etici;
    \item il \textbf{fanciullo}, che rappresenta l'oltreuomo, dionisiaco, un vero spirito libero.
\end{itemize}

\subsubsection{La volontà di potenza}
Secondo Nietzsche, la volontà di potenza si identifica con la vita stessa, è la spinta dell'autoaffermazione. 
È una forza creativa che può anche manifestarsi come sopraffazione e dominio. 

\subsubsection{L'eterno ritorno}
Il superuomo deve infine accettare l'eterno ritorno all'uguale, ovvero vivere la propria vita uguale per l'eternità. 

\subsection{La filosofia del crepuscolo}
Nelle sue ultime opere, Nietzsche critica la morale e il cristianesimo. 
La moralità infatti non è altro che \say{l'istino del gregge nel singolo}, ovvero la tendenza dell'uomo ad assoggettarsi a determinate pratiche sociali. 
Quella che anticamente era una morale \say{dei signori} successivamente, con l'avvento dell'ebraismo e del cristianesimo, viene ribaltata in una morale \say{degli schiavi}, che consiste in un risentimento contro la vita. 

Nietzsche tenta di superare il problema del nichilismo, da lui identificato come quella situazione di sgomento e nulla di fronte alla morte di Dio. 
Il nichilismo può essere suddiviso in:
\begin{itemize}
    \item \textbf{nichilismo incompleto}, in cui vengono distrutti i vecchi valori e ne vengono creati nuovi uguali ai precedenti;
    \item \textbf{nichilismo completo}, che si impegna a distruggere ogni rimasuglio di credenza rimasto. 
\end{itemize}

Nell'ultimo periodo, Nietzsche radicalizzò notevolmente il suo prospettivismo secondo cui non esistono cose o fatti, ma solo interpretazioni. 
Pertanto, il filosofo si schiera contro la scienza moderna, che tenta di dare un'interpretazione unica e meccanicistica a ciò che è in realtà libero e plurale. 

\section{Sigmund Freud (1856 -1939)}
Lavorando come psichiatra a fianco di Breuer (un dottore che sperimentava con tecniche ipnotiche per trattare i pazienti affetti da isteria), Freud ipotizzò che la maggior parte della vita mentale si svolgesse al di fuori della coscienza. 
Nella prima topica psicologicha (ovvero lo studio dei luoghi della psiche), Freud identifica tre \say{sistemi}: 
\begin{itemize}
    \item \textbf{conscio};
    \item \textbf{preconscio}, costituito da ricordi sopiti che possono essere facilmente richiamati alla memoria;
    \item \textbf{inconscio}, ovvero tutto ciò che è stato rimosso e può riemergere solo attraverso speciali tecniche. 
\end{itemize}
Nella seconda topica psicologica invece identifica tre \say{istanze}: 
\begin{itemize}
    \item l'\textbf{Es}, ovvero la forza impersonale e caotica delle pulsioni, che obbedisce solo al \say{principio di piacere};
    \item il \textbf{Super-io}, ovvero tutti i divieti instillati nell'individuo dall'esterno;
    \item l'\textbf{Io}, la parte organizzata della personalità che deve controllare le altre due istanze.
\end{itemize}
Un Io incapace di governare Super-io ed Es porta alla nevrosi. 

Nell'opera \emph{L'interpretazione dei sogni}, Freud identifica nel sogno una via per conoscere l'inconscio. 
Secondo lui infatti, i fenomeni onirici consisterebbero in un appagamento camuffato di un desiderio rimosso. 
All'interno dei sogni si possono infatti individuare un \say{contenuto manifesto} (ciò che vive il soggetto nel sogno) e un \say{contenuto latente} (l'insieme delle tendenze che danno luogo al sogno).
Un altra cosa che Freud prende in esame sono gli \say{atti mancati}, tutta quella serie di dimenticanze quotidiane che secondo lo psicanalista sono manifestazioni dell'inconscio. 

Freud elaborò anche una nuova teoria della sessualità in grado di spiegare atti come la sessualità infantile, la sublimazione e la perversione (ovvero la ricerca del piacere indipendentemente dal fine riproduttivo). 
Egli inannzitutto amplia il concetto di sessualità definendolo un'energia che può essere diretta verso le mete più diverse, ovvero la libido. 
Secondo Freud, il bambino è \say{un essere perverso e polimorfo}, che sviluppa la propria sessualità in tre fasi: 
\begin{itemize}
    \item \textbf{fase orale} (fino a 1.5 anni), la zona erogena è la bocca e l'attività principale il poppare;
    \item \textbf{fase anale} (da 1.5 a 3 anni), la zona erogena è l'ano ed è collegata all'escrezione;
    \item \textbf{fase genitale} (dai 3 anni in poi) in cui la zona erogena sono i genitali ed è a sua volta articolata in  
            \begin{itemize}
            \item \textbf{fase fallica}, in cui il bambino e la bambina scoprono il pene e soffrono di un complesso di castrazione; 
            \item \textbf{fase genitale in senso stretto}, in cui le pulsioni sessuali sono organizzate con il primato della zona genitale.
        \end{itemize}
\end{itemize}
Durante la fase fallica si sviluppa inoltre il complesso edipico, che consiste in un \say{attaccamento libidico verso il genitore di sesso opposto e atteggiamento ambivalente verso il genitore di egual sesso}. 
La risoluzione di tale complesso determina la futura personalità dell'infante. 

Secondo Freud, l'arte è analoga alla produzione onirica: anche questa è una manifestazione di desideri insoddisfatti. 
Il soddisfacimento di questo desiderio proibito avviene attraverso la \emph{sublimazione}, ovvero il trasferimento di pulsioni sessuali su oggetti non sessuali. 
L'artista è colui che è capace di sublimare questi desideri proibiti in forme socialmente accettabili. 
L'arte quindi funge come una sorta di terapia sia per l'artista che per lo spettatore. 

\section{Karl Popper (1902 - 1994)}
Popper nella sua filosofia combinò elementi neopositivistici e anti-neopositivistici per giungere a una teoria epistemologica completamente originale. 
I problemi affrontati da Popper furono quelli della demarcazione tra scienza e pseudoscienza e della certezza del sapere scientifico. 
Il pensiero di Popper può inoltre essere visto come una diretta conseguenza della rivoluzione scientifica operata da Einstein. 
Da Einstein infatti trae i principi di fondo della propria epistemologia: il \emph{falsificazionismo} e il \emph{fallibilismo}. 

In primo luogo, Popper procede a una riabilitazione della filosofia, ribadendone la necessità. 
Infatti esistono problemi di natura strettamente filosofica e, in un modo o nell'altro, la filosofia ha sempre a che fare con la conoscenza della realtà. 

Per quanto riguarda la demarcazione tra scienza e non-scienza, Popper riconosce che il verificazionismo è utopico: per verificare davvero una legge scientifica bisognerebbe avere presenti tutti i casi e questo non è possibile (quindi non è possibile il metodo induttivo). 
Di conseguenra, Popper propone il principio della \emph{falsificabilità}, per cui una teoria è scientifica solo quando questa può venir smentita dall'esperienza. 
La scienza si fonda quindi su un certo numero di asserzioni-base (su cui la comunità scientifica concorda) che possono essere sempre messe in discussione: da qui Popper deriva l'immagine della scienza \say{come un edificio costruito su palafitte} e non più come qualcosa di immutabile e perfetto. 
Popper inoltre riconosce l'asimmetria presente tra verificabilità e falsificabilità: miliardi di conferme non rendono certa una teoria ma una sola confutazione la rende invalida. 
Tuttavia, sebbene le ipotesi non possano venir verificate, possono essere \emph{corroborate} quando superano un'esperienza possibilmente falsificante. 
La corroborazione non può fungere da criterio di giustificazione delle teorie ma può servire come criterio di scelta tra teorie rivali. 

Secondo quanto detto prima, la metafisica non è una scienza poiché non è falsificabile. 
Questo però non significa che essa non abbia un senso: 
\begin{itemize}
    \item le scoperte scientifiche sono spesso spinte da credenze metafisiche; 
    \item le dottrine metafisiche possono comunque essere razionalmente discutibili. 
\end{itemize}
Popper si scaglia invece contro il marxismo e la psicanalisi freudiana perché prive di sufficiente falsificabilità e dirette ad aggirare qualsiasi smentita. 

Secondo Popper, non esiste alcun metodo per scoprire una teoria scientifica, in quanto queste sono frutto di congetture audaci ed intuizioni creative. 
Il \say{metodo scientifico} proposto da Popper sarebbe un processo di \emph{trial and error}, prova ed errore che va sottoposto al vaglio dell'esperienza. 
L'errore è quindi parte integrante del sapere scientifico ed ha un'importante funzione di crescita. 
Il modo in cui si evolve la scienza sarebbe quindi analogo all'evoluzione biologica teorizzata da Darwin. 

Il filosofo rifiuta inoltre l'\emph{osservazionismo} secondo cui lo scienziato osserva la natura senza alcuna ipotesi precostituita. 
Popper propone infatti l'immagine della mente come un faro, che a seconda delle ipotesi preconcette illumina la realtà con luce diversa. 
L'osservazione quindi non può mai essere completamente distaccata ma viene sempre eseguita con delle ipotesi e aspettative a monte. 
Popper rifiuta inoltre il \emph{fondazionalismo} e il \emph{giustificazionismo} del sapere, per cui la scienza avrebbe basi certe che la filosofia deve giustificare, affermando che:
\begin{itemize}
    \item il nostro sapere è strutturalmente incerto;
    \item la scienza è intrinsecamente fallibile e autocorreggibile;
    \item il problema di come possiamo giustificare la nostra conoscenza è privo di senso;
    \item l'uomo non potrà mai possedere la verità ma solo ricercarla senza conclusione (cfr. Seneca).
\end{itemize}
Lo scopo della scienza quindi non può essere la verità ma il raggiungimento di sempre maggiore verosimiglianza. 
È quindi necessario stabilire un criterio di preferenza tra teorie. 
È innnanzitutto ovvio che teorie scientifiche siano preferibili a teorie non scientifiche perché le prime possono essere controllate empiricamente. 
Tra teorie scientifiche invece la decisione dev'essere frutto di una discussione critica che tenga conto delle ipotesi in gioco. 
Per poter affermare questo, Popper stabilisce inoltre che teorie scientifiche che rispondano allo stesso problema possono essere confrontate. 
Popper approda quindi ad un'epistemologia evoluzionistica in cui le teorie migliori sopravvivono. 

Secondo Popper l'indeterminismo è un requisito necessario per ogni dottrina della libertà. 

Il pensiero di Popper si estese anche alla sfera politica. 
Nei suoi scritti politici, il filosofo difende la \say{società aperta} e critica ogni forma di assolutismo. 
Per prima cosa, Popper si schierra contro lo \say{storicismo}, ovvero quelle dottrine filosofiche con la pretesa di identificare un senso globale della storia (cfr Hegel, Marx). 
Non esiste un senso della storia precostituito perché gli uomini possono attribuirle ogni significato. 
Un altro errore dello storicismo è quello di fare confusione tra leggi e tendenze: per poter fare previsioni scientifiche bisogna basarsi sulle leggi. 
Secondo Popper quindi, nello storicismo vi sarebbero pretese totalitarie che produrrebbero solo sofferenza agli uomini. 
La società aperta è quella società fondata sulla salvaguardia delle libertà individuali attraverso istituzioni democratiche autocorreggibili. 
Una democrazia, secondo Popper, non è solo quello che viene tradizionalmente identificato come il \say{potere del popolo}, ma è quel sistema di governo in cui i governati hanno la possibilità di controllare i governanti senza ricorrere alla violenza. 
Il rifiuto della violenza è infatti categorico, con l'unica eccezione del ribaltamento di una tirannide. 
Il filosofo inoltre critica l'atteggiamento rivoluzionario esaltando il metodo riformista, superiore perché:
\begin{itemize}
    \item evita di promettere \say{paradisi};
    \item non pone fini assoluti che giustifichino qualsiasi mezzo;
    \item procede per via sperimentale;
    \item riesce a dominare meglio i mutamenti sociali;
    \item è in grado di salvaguardare la libertà.
\end{itemize}
Per questo, secondo Popper l'unico valore da conservare è il metodo della libertà e della democrazia, ovvero l'equivalente politico del metodo della scienza. 

\section{Carl Schmitt (1888 - 1985)}
La meditazione di Schmitt fu incentrata sulla politica. 
In particolare, in \emph{Teologia politica}, afferma che la sovranità non risiede nella norma bensì nella decisione (da cui \emph{decisionismo}) che la pone in essere. 
Tale decisione secondo il filosofo avviene in uno stato d'eccezione. 
Ne \emph{Il concetto di politico}, Schmitt tenta di chiarire l'essenza della politica, ormai divenuto necessario poiché Stato e società si compenetrano a vicenda. 
La risposta a cui giunge è che la politica è responsabile di determinare la coppia antitetica \emph{amico-nemico} su cui si fonda l'identità dello Stato. 
Per questa definizione, la politica è intrinsecamente conflittuale. 
La guerra è quindi una possibilità umana sempre presente. 
Schmitt evidenzia inoltre i limiti del parlamentarismo e del liberalismo, auspicando uno stato \say{totale} come quello nazista. 

Schmitt evidenzia inoltre che la società occidentale si sia sempre organizzata intorno a determinati \emph{centri di riferimento}, che condizionano la vita politica. 
Storicamente questi sono stati:
\begin{itemize}
    \item il teologico;
    \item il metafisico-scientifico;
    \item il morale-umanitario;
    \item l'economico;
    \item il tecnologico.
\end{itemize}

Nello \emph{Ius Publicum Europaeum}, Schmitt denuncia la crisi del diritto pubblico europeo, realizzatasi a partire dalla nascita della Società delle Nazioni, un'istituzione universale che si propone di abolire la guerra in tutto il mondo. 
Quest'organizzazione muterebbe il significato della guerra che, da un modo di relazionarsi tra Stato e Stato, diventerebbe un crimine contro l'umanità da bandire in modo assoluto. 

\section{Hannah Arendt (1906 - 1975)}
Nata da famiglia ebrea, Hannah Arendt fu costretta a fuggire dalla Germania dopo l'avvento del nazismo, rifugiandosi prima in Francia e poi negli Stati Uniti. 
Lei stessa si è sempre definita una \say{pensatrice politica}, autrice di opere come \emph{Le origini del totalitarismo}, \emph{Vita activa} e \emph{La banalità del male}. 

\subsection{Le origini del totalitarismo}
In quest'opera, Arendt analizza le cause e il funzionamento dei regimi totalitari, considerati una conseguenza diretta della società di massa. 
Il totaslitarismo è quel regime in cui:
\begin{itemize}
    \item tutto appare politico;
    \item tutto diventa pubblico;
    \item tutto è riferito a una legge superiore;
    \item viene dato enorme valore all'azione (per tenere le masse mobilitate);
    \item regna il discorso;
    \item si propone di creare un'umanità \say{nuova}.
\end{itemize}
Tuttavia questa apparenza va smascherata perché se non esiste alcun confine tra il politico il pubblico e il privato, tra il politico e il non politico, la politica scompare. 
Secondo la filosofa, è possibile identificare alcuni momenti significativi di vita politica, primo tra tutti la \emph{polis} greca antica. 
Qui infatti gli individui si riconoscono come uguali attraverso la discussione e la deliberazione comuni. 
Ne \emph{Le origini del totalitarimo}, Arendt inoltre riprende e amplia la riflessione di Montesquieu ne \emph{Lo spirito delle leggi}, identificando le caratteristiche fondamentali di ogni sistema di governo. 
\begin{center}
\begin{tabularx}{1.0 \textwidth}{ 
    >{\raggedright\arraybackslash}X 
  | >{\raggedright\arraybackslash}X 
  | >{\raggedright\arraybackslash}X 
  | >{\raggedright\arraybackslash}X }
    &\textbf{natura} &\textbf{principio d'azione} &\textbf{esperienza fondamentale}\\
    \hline
    \textbf{repubblica} &sovranità popolare &virtù &uguaglianza per nascita\\
    \hline
    \textbf{monarchia} &governo di uno solo subordinato a leggi &onore &disuguaglianza per nascita\\
    \hline
    \textbf{dispotismo} &governo assoluto di uno solo &paura &angoscia di fronte all'isolamento\\
    \hline
    \textbf{totadlitarismo} &terrore &ideologia &desolazione
\end{tabularx}
\end{center}
La novità rispetto all'antico dispotismo è che il totalitarismo distrugge anche la vita privata delle persone, estraniandole dalla società e rendendole nemiche tra loro. 
Il totalitarismo accentua quindi l'isolamento tipico degli uomini nella società di massa. 

\subsection{Vita activa}
In quest'opera, Arendt vuole insegnare a riconoscere e tutelare la \emph{res publica} e la politica. 
L'oggetto del saggio è la vita attiva, contrapposta alla vita contemplativa. 
Secondo Arendt, la \emph{vita activa} si articola in tre forme fondamentali: 
\begin{itemize}
    \item il \textbf{lavoro} (labour, tipico dell'\emph{animal laborans}), energia che si sprigiona e viene subito consumata; 
    \item l'\textbf{operare} (work, tipico dell'\emph{homo faber}), che tende a produrre trasformazioni durature;
    \item l'\textbf{agire} (action, tipico dello /emph{zóon politikón}), ciò che mette in relazione gli umani.
\end{itemize}
Altre due esperienze tipiche della condizione umana sono la natalità e la mortalità, entrambe strettamente collegate alla sfera dell'azione. 

L'azione ha alcune caratteristiche proprie:
\begin{itemize}
    \item è inizio, novità;
    \item rivela il "chi è" di una persona;
    \item è diversa dal comportamento abituale.
\end{itemize}
Essa presenta inoltre alcuni rischi:
\begin{itemize}
    \item ha effetti incontrollabili;
    \item non può essere compresa se non quando è totalmente compiuta;
    \item l'azione è rischio, necessita di coraggio di fronte all'ignoto.
\end{itemize}

Arendt identifica in Platone e Aristotele gli iniziatori di quel processo che porterà alla svalutazione della vita attiva a favore di quella contemplativa. 
Questa negazione si è affermata poi con il cristianesimo, con il quale l'agire politico è diventato impossibile. 

Alla fine emergono due modi radicalmente diversi di concepire la politica: 
\begin{itemize}
    \item come dominio di qualcuno sugli altri che richiede violenza;
    \item come organizzazione del potere da parte di uomini parlanti e agenti.
\end{itemize}
Secondo Arendt, il potere non dipende dal possesso di mezzi violenti, anzi, se si ricorre alla violenza non si ha o si è già perso il potere.

\chapter{Inglese}
\section{Romanticism}

\subsection{William Blake (1757 - 1827)}

\subsubsection{Life an´ works}
Blake hailed from a humble family and he remained poor throughout all his life. 
He worked as an engraver and painter, creating a new art that emphasized the power of imagination over reason. 
This meant that imagination was the only to get to know the world, making the poet a sort of prophet who can see more deeply into reality. 
Blake was always in favour of the French Revolution and he remained a radical all his life. 
He believed that progress corrupted man's soul and that the artist should be the guardian of the spirit and imagination. 
He had a strong sense of religion and his greates literary influence was the Bible. 
He invented the "illuminated printing", an artistic form combining poetry and pictures. 
His most notable works are the \emph{Songs of Innocence} (1789) and the \emph{Songs of Experience} (1794), but he also wrote prophetic books. 
Blake also theorized that \say{complementary opposites} (such as Innocence and Experience) exist in parallel and the possibility of growth lies in the tension between these two opposites (cfr Hegel). 
Blake's style is simple, full of symbols. 
His verse is linear and rythmical and is characterized by a heavy use of repetition. 

\subsubsection{\emph{Songs of Innocence}}
The \emph{Songs of Innocence} were written before the French Revolution. 
The narrator is a shepherd who, inspired by a child in a cloud, makes songs celebrating the divine presence in everything. 
The symbol of innocence is childhood, depicted as a period of happiness, freedom and imagination. 

\subsubsection{\emph{Songs of Experience}}
Blake wrote the \emph{Songs of Experience} during the period of the Terror after the French Revolution. 
This is in fact the counterpart of the \emph{Songs of Innocence}: here a bard questions the themes tackled in the first collection. 
These new songs are meant to be read in pair with the older ones and it can be seen that a more pessimistic view of life emerges. 
Experience is identified with adulthood but this does not replace childhood, in fact it merely coexists, providing another point of view on reality. 

\subsection{Mary Shelley (1797 - 1851)}

\subsubsection{Life and works}
Mary Shelley was the daughter of Mary Wollstonecraft (a feminist philosopher) and William Godwin (an anarchist and philosopher). 
Her mother died shortly after her birth and her father later remarried. 
Godwin's house was visited by some of the most notable charactes of the time, such as Samuel Taylor Coleridge and Percy Bysshe Shelley. 
The latter fell in love with Mary and the couple fled to Switzerland, where Mary found the inspiration to write \emph{Frankenstein, or The Modern Prometheus} which was published anonimously in 1818. 
In 1822 Percy set sail and was found drowned after a storm and the next year Mary returned to England, where she lived the rest of her life. 

\subsubsection{\emph{Frankenstein, or The Modern Prometheus}}
Victor Frankenstein, a Swiss scientist, is able to create a human being stitching together different parts of corpses. 
The result of the experiment however is appalling: he becomes a murderer and kills his creator. 
The story is told through a series of letters that Walton (a young explorer on a journey to the North Pole) writes to his sister Margaret. 
The novel is set throughout Europe, but the most important place is the North Pole, where Frankenstein is found chasing his creation. 

The theme of the double is recurring throughout the novel (cfr Pirandello, Svevo). 
Walton is a double of Frankenstein for they share the same ambition to overcome human limits, the same loneliness and the same pride. 
Frankenstein and his creature are complementary: they both suffer from isolation and alienation, they both desire to be good but get obsessed with hate and revenge. 
Frankenstein's rejection of his creature is what makes it an outcast and a murderer. 

Mary Shelley was heavily influenced by the latest advancements in science and it proposes the problem of its responsibility toward humanity. 
The creature can also be considered Rousseau's natural man, in a primitive state. 
She also took inspiration from Coleridge's \emph{The Rime of the Ancient Mariner} as both depict a crime against nature (the killing of the albatross is akin to the creation of the monster). 
The Greek myth of Prometheus was also an important influence for the novel. 

\subsection{William Wordsworth (1770 - 1850)}

\subsubsection{Life and works}
Born in England, Wordsworth came in contact with revolutionary France and he was filled with enthusiasm for democratic ideals. 
The subsequent developments of the Revolution brought him on the verge of a nervous breakdown. 
In 1795 he met Samuel Taylor Coleridge,  with whom he developed a strong friendship. 
Together they wrote the \emph{Lyrical Ballads}, a collection of poems published anonimously in 1798. 
The second edition (1800) also contained Wordsworth \emph{Preface}, which later became the Manifesto of English Romanticism. 
He carried on writing and growing his poetic reputation until he was made Poet Laureate in 1843. 
The last years of his life were characterized by a growing political conservatism. 

\subsubsection{The Manifesto of English Romanticism}
Wordsworth regarded poetry as a solitary act originating in the ordinary. 
In his preface, he describes it as \say{emotion recollected in tranquillity}. 
Hence, the subject matter should deal with everyday situations and ordinary people and the language should be simple. 
Therefore, the poet should be a man among men. 
Wordsworth also believed in the goodness of nature and that of the child. 
He also thought that man and nature are inseparable, offering a sort of pantheistic world view. 

\subsection{Samuel Taylor Coleridge (1772 - 1834)}

\subsubsection{Life and works}
Coleridge received an excellent education in the classics but failed to graduate at Cambridge. 
As a student, he was heavily influenced by the ideals of the French Revolution, becoming an enthusiastic republican. 
After his disillusionment with the Revolution, he and the poet Robert Southey planned to create a utopian society (called \emph{Pantisocracy}) in America in which private ownership would not exist (cfr Marx). 
The project would never come to life. 
In 1795 he met Wordsworth with whom he composed the \emph{Lyrical Ballads}. 
His masterpiece, \emph{The Rime of the Ancient Mariner} was written in 1798 and it is the first poem in the \emph{Lyrical Ballads}. 
Other of his works include \emph{Christabel} (an unfinished poem set in the Middle Ages), \emph{Kubla Khan} (an unfinished poem probably written under the influence of opium), and \emph{Biographia Literaria} (an autobiography and literary critic). 
Coleridge's concern was to write about extraordinary events in a credible way. 

\subsubsection{\emph{The Rime of the Ancient Mariner}}
In the first part, the ancient Mariner stops a wedding guest to tell him about his journey. 
The protagonist and his fellow reached the South Pole after a violent storm. 
After a few days, an albatross appeared and was considered a sign of good omen. 
The Mariner shot dead the albatross. 
This is seen as a crime against nature and in the next part the Mariner is punished for his misdeed. 
In the third part, the Mariner becomes conscious of what he has done. 
In this section, a phantom ship approaches the crew and Death and Life-in-Death cast dice. 
Death wins the Mariner's fellows who die, while Life-in-Death wins the Mariner's life. 
In the fourth part, the Mariner is alone and trying to reconcile with nature. 
In the fifth part, the process of revival of the soul continues. 
In the sixth part, the purification seems impeded by an unknown obstacle. 
In the seventh part, the Mariner gains the wedding guest's sympathy and is still haunted by a sense of guilt that will only end with his death. 

Coleridge saw nature as essential to poetic creativity for it stimulated the poet to find symbols that could reflect his feelings. 
The poem shares many of the features of medieval ballads such as the structure, the archaic language, the use of alliteration, repetition, and onomatopoeia. 
\emph{The Rime of the Ancient Mariner} has been interpreted in many different ways. 
It may be the description of a dream, an allegory of the life of the soul, or a description of the poetic journey of Romanticism. 

\subsection{George Gordon Byron (1788 - 1824)}

\subsubsection{Life and works}
Rich and handsome, he had a deformed foot and was quite unconventional. 
He was a brilliant mind and forced himself to become excellent at sports. 
He went on his Grand Tour in 1809 and during this time he found the inspiration to write the first two cantos of \emph{Childe Harold's Pilgrimage} which he published in 1812. 
This gave Byron great fame until he fled England in 1816 because a scandal broke out from his incestuous relationship with his half-sister. 
In 1816 he therefore moved to Geneva, becoming a close friend with Percy Bysshe Shelley. 
He also wrote the third canto of \emph{Childe Harold}. 
He then moved to Venice, where he wrote the tragedy \emph{Manfred}, the fourth and last canto of \emph{Childe Harold}, the mock-heroic poem \emph{Beppo}, and the mock-epic \emph{Don Juan}. 
In 1819 he moved to Milan where he plotted against Austrian rule over Italy. 
Later, he committed to Greek struggle of independence from Turkey. 
For his actions he is regarded as a hero in Greece, where his heart is buried. 
Byron never considered himself a romantic poet, in fact he criticized Wordsworth, Coleridge, and Keats. 
Nevertheless, he was the only poet of his time to gain international recognition and to influence the work of authors such as Dostoevskij, Pushkin, Goethe, and de Balzac. 

Byron firmly believed in individual liberty and fought against any kind of constraint. 
He wished all men to be free so he devoted his life to fight against tyrants across the world. 
The protagonists of his works are isolated men struggling against nature (cfr Leopardi) whose feelings are reflected in their surroundings. 
He denounced the evils of his society through a satirical style. 

\subsubsection{\emph{Childe Harold's Pilgrimage}}
The poem is divided in four independent cantos. 
The \emph{fil rouge} is given by the protagonist, Harold, a nobleman awaiting knighthood (this is the meaning of \say{Childe}) who travels around the world. 
Harold's boredom and disillusionment cause him to leave England and set off to exotic places. 
The first two cantos are set through Spain, Portugal, Albania, and Greece and they evoke the glorious past of these nations and the scenery. 
In the third canto, Byron reflects upon human ability to forget while narrating his own journey after he left England. 
In the last canto, Byron depicts the sea as the image of the sublime and of eternity. 

\subsection{Percy Bysshe Shelley (1792 - 1822)}

\subsubsection{Life and works}
Percy Bysshe Shelley was the son of a wealthy and conservative Member of Parliament. 
He rebelled against his conservative family publishing a pamphlet named \emph{The Necessity of Atheism} in 1811, which caused him to get expelled from Oxford University. 
He married Harriet Westbrook at age 19 and they moved to Ireland, where Shelley made revolutionary propaganda against Catholicism and English authority. 
Even though the enthusiasm for the French Revolution had died down, Shelley was a republican, vegetarian and advocate of free love. 
He was interested in occult sciences and scientific experiments. 
When he and his wife came back to England, they realized their marriage was not working so they separated and Percy later remarried with Mary Wollstonecraft Godwin. 
They went to live in Switzerland and later in Italy, where Percy wrote \emph{Ode to the West Wind} in 1819 and other works. 
His life ended tragically: after having set sail from Livorno, his ship was struck by a storm which drowned him. 

In his essay \emph{A Defence of Poetry}, Shelley regards poetic activity as an expression of imagination capable of revolutionary activity in an increasingly materialistic world. 
In Shelley's opinion, nature is not the real world but a beautiful veil concealing the eternal truth (cfr Schopenhauer). 
Nature also represents a shelter from the disappointment caused by the ordinary world. 
The poet for Shelley is both a prophet and a titan challenging the universe (cfr Leopardi). 
His task is to help humanity reach a world characterized by freedom, love, and beauty. 

\subsubsection{\emph{Ode to the West Wind}}
Published with \emph{Prometheus Unbound}, this is a lyrical composition with an elevated tone. 
In this work, Shelley identifies with Prometheus himself, the heroic titan. 
Much like Prometheus, Shelley hopes that his fire (his liberal philosophy) will enlighten humanity and liberate it from intellectual imprisonment. 

\section{The Victorian Age}

\subsection{Alfred Tennyson (1809 - 1892)}

\subsubsection{Life and works}
Alfred was the fourth son (out of twelve) of a clergyman. 
He was educated at Trinity College in Cambridge, showing off his intelligence and humour, but he dropped off without graduating. 
During his years in Cambridge he met Arthur Hallam, with whom he would travel to the Continent. 
Arthur died in Vienna in 1833 and Alfred would spend several years meditating on this tragic loss. 
In 1850 he was made Poet Laureate and in 1884 he was nominated Baron for his literary merits, joining the House of Lords. 

His first remarkable works were the dramatic monologues included in his collection \emph{Poems} (1842). 
His masterpiece \emph{Ulysses} is part of this collection. 
He also wrote the poem \emph{The Princess}, in favour of women's right to education, and the elegy \emph{In Memoriam}. 

\subsubsection{Ulysses} 
The inspiration for this monologue comes from Dante, who tells the story of Ulysses's last adventure in his \emph{Divina Commedia}. 
Ulysses is an overreacher, thirsty for knowledge at any cost. 
Tennyson depicts two different kinds of life through Ulysses and his son, Telemachus. 
In fact, while Ulysses represents an active, adventurous life, Telemachus embodies the typical Victorian man. 

\subsection{Charles Dickens (1812 - 1870)}

\subsubsection{Life and works}
Charles Dickens lived an unhappy childhood: his father was imprisoned for debt when Charles was 12. 
This forced him to go to work in a factory. 
When his father was freed, he was sent to a school in London. 
He began studying shorthand writing and by 1832 he became a successful reporter of parliamentary debates and began working as a writer for a newspaper. 
In 1833 he published his very first story. 
He wrote \emph{Sketches by Boz} and later \emph{The Pickwick Papers}. 
After its latest success, he began a full-time career as a novelist, producing works such as \emph{Oliver Twist} (1838), \emph{A Christmas Carol} (1843), \emph{David Copperfield} (1850), and \emph{Hard Times} (1854). 
He gained immense fame for his novels and was buried in Westminster Abbey. 

Dickens used to tell stories regarding the lower class world and he was always on the side of the poor. 
Children are often the most important characters in his novels. 
His aim was to school the upper classes and the rulers about the poor and the problems they faced. 

\subsubsection{\emph{Oliver Twist}}
This novel is heavily autobiographical, representing the financial insecurities and humiliation Dickens had to endure as a child. 
The name \say{Twist} itself represents the reversals of fortune he will experience. 
Oliver Twist is a poor boy, son of unknown parents, born in a workhouse in a small town near London. 
He is brought up in said workhouse but one day he commits the offence of asking for more food. 
He is therefore sent as an apprentice to anyone willing to take him. 
He is first sold to an undertaker who is cruel and makes Oliver run away to London. 
In London he falls into the hands of a gang of pickpockets trained by Fagin, who runs a school for thieves. 
Oliver is caught on his first attempt at theft: the victim, Mr Brownlow, rather than charging him with theft takes him home and takes care of him. 
Oliver is then kidnapped by Fagin's gang and forced to commit burglaries. 
During one job he is shot and wounded. 
Oliver is then adopted by Mr Brownlow and he receives the love and affection he has always lacked. 
Eventually, investigations are made about Oliver's origins and he is discovered to be of noble descent. 
In the end, the gang of pickpockets and Oliver's half-brother (who paid the thieves in order to ruin Oliver) are all arrested. 

The most important setting in this novel is London, of which he depicts three different social levels. 
First, the world of the parochial workhouse which is insensitive and rigid. 
Second, the criminal world of violent people driven by poverty and hunger. 
Last, the world of the Victorian middle class, a world of respectable people with high moral values. 

\subsubsection{\emph{Hard Times}}
The novel is set in Coketown, an imaginary industrial town. 
Thomas Gradgrind is an educator only interested in facts and statistics. 
He has founded a school and brigs up his two children, Louisa and Tom, repressing their imagination and feelings. 
Louisa is married to a rich banker and Tom is given a job at the bank. 
However, he robs his employer and is forced to leave the country. 
Mr Gradgrind eventually understands the harm he has done his children and gives up his philosophy. 

\emph{Hard Times} is divided into three sections, \say{Sowing}, \say{Reaping}, and \say{Garnering}. 
Coketown represents any real industrial town in the mid 1800s: a brick jungle, heavily polluted, and covered in soot. 
Dickens wants to warn against the dangers of the pilosophy of Utilitarianism. 

\subsection{Walt Whitman (1819 - 1892)}

\subsubsection{Life and works}
Born in Long Island, New York, Walt had little education and at 11 he started working first as an office boy and later as a printer's apprentice. 
He then became a journalist, supporting his radical democratic views. 
When he was about 30, he started travelling and studying by himself. 
In 1855 he published the first (out of nine) edition of \emph{Leaves of Grass}. 
After the third edition he caused the indignation of puritanical readers for his dealing with obscenity and homosexuality. 
During the Civil War he devoted himself to visiting wounded soldiers in the army hospitals. 
He strongly believed in the abolition of slavery and national integrity. 
He then retired and died in Camden, New Jersey. 
He became especially popular in Europe in the Aesthetic movemet. 
He is usually regarded as the father of American poetry. 

All his work is incorporated in \emph{Leaves of Grass}, giving a sense of unity and growth throughout the collection. 
In his poetry, Whitman regarded nature as a shelter from the stress of the materialistic world. 
The natural world is also the body of the Earth. 
\emph{Leaves of Grass} is written in free verse, it often gives a sense of fragmentation. 

Whitman's poetry is pervaded by optimism and a great faith in the future of the American nation. 
He celebrated America in all its variety, giving voice to common men. 
He also deals with the theme of physical love and sex. 
What he valued most is the dignity of the individual. 

\subsection{Rudyard Kipling (1865 - 1936)}

\subsubsection{Life and works}
Rudyard Kipling was born in Bombay, India. 
He was able to learn Hindi (alongside with English) and witness Indian life. 
When he was 6 he was sent to England to attend to school. 
In 1882 he returned to India, where he began working as a journalist and published collection of poems and short stories about Indian life. 
He then returned to London and eventually moved to the USA with his American wife. 
He went back to England in 1896 and he continued writing and in 1907 he was the first Englishman to receive the Nobel prize for Literature. 
During the First World War he worked as a correspondent. 
His ashes are buried in Poets' Corner in Westminster Abbey. 
In his works, Kipling exalted imperial power and the white men's superiority. 

\subsection{Oscar Wilde (1854 - 1900)}

\subsubsection{Life and works}
Oscar Wilde was born in Dublin, where he attended Trinity College before being sent to Oxford, where he gained a degree in Classics. 
He was known for being eccentric and he accepted the theory of \say{Art for Art's sake}. 
After his graduation, Oscar moved to London where he became famous for his style as a \emph{dandy} (cfr d'Annunzio). 
In 1881 he published a collection called \emph{Poems}, which won him a speaking tour in the USA. 
When he returned to Europe, he married Constance Lloyd, who bore him two children. 
In 1891 he published his masterpiece \emph{The Picture of Dorian Gray}. 
After this, he developed an interest in drama and produced a series of plays. 
In 1895, Wilde's triumph ended dramatically when he was put to trial and then sentenced to two-years' hard labour for his homosexuality. 
When he was released in 1997, he went into voluntary exile in France, where he died of meningitis in absolute poverty. 

\subsubsection{\emph{The Picture of Dorian Gray}}
The novel is set in London at the end of the 19\textsuperscript{th} century. 
A painter, Basil Hallward, is fascinated by the beauty of young man Dorian Gray and decides to paint his portrait. 
Dorian then throws himself into a life of pleasure, led by the corrupt Lord Henry Wotton. 
All the young man's wishes are satisfied, including that of eternal youth. 
In fact, the signs of age appear on the portrait instead of Dorian. 
He is completely insensitive to the pain he causes around him. 
Eventually Dorian wants to free himself of the painting, witness of all his misdeeds. 
He proceeds to stab the painting, but in doing so he kills himself and the painting goes back to its original purity. 

The picture represents Dorian's dark side, which he tries to forget by locking it in a room. 
The moral of this novel is that in the end every excess must be punished and there is no possible escape from it. 
The corrupted picture can be seen as a symbol of the immorality and bad conscience of the Victorian middle class. 
In the end we can see Wilde's theory of art: art survives people and is eternal. 

\section{Modernism}

\subsection{The War Poets}

\subsubsection{Rupert Brooke (1887 - 1915)}
Rupert Brooke was educated at Rugby School and later went to King's College. 
He was a great student and athlete and was deemed very handsome. 
He got acquainted with literary circles such as the Bloomsbury Group and other important intellectual and political figures. 
A convinced jingoist, he joined the army at the beginning of the conflict but died of blood poisoning as soon as he was deployed to Greece. 
In his sonnets, Brooke claims that war is clean and cleansing. 

\subsubsection{Wilfred Owen (1893 - 1918)}
Wilfred Owen was working as an English teacher in France when he visited a war hospital and decided to return to England and enlist. 
In 1917 he was deployed to France and took part the fight. 
In March of the same year, he was sent to Edinburgh to recover from shell shock. 
There he met Sigfried Sassoon, who encouraged him to continue writing and later contributed to him getting recognized as a poet. 
In 1918, Owen returned to the battlefield and was killed in a German machine gun attack, just seven days before the armistice. 
His poems are dramatic as they offer a truthful description of the pain the soldiers had to endure. 

\subsubsection{Sigfried Sassoon (1886 - 1967)}
Sigfried Sassoon joined the war in 1915 and was sent to France. 
Sassoon expressed his disillusionment with the war through the irony in his poems. 
He also protested publicly against the war. 
A friend of his was able to prevent him from being cour-martialled by convincing everyone Sassoon was suffering from shell shock. 
Sassoon was therefore sent to Edinburgh to recover, where he met Wilfred Owen. 
In his poems he denounced the reality soldiers were being put through. 

\subsection{Thomas Stearns Eliot (1888 - 1965)}

\subsubsection{Life and works}
T.S. Eliot was born in St Louis, Missouri, and educated at Harvard, but his cultural background was English and European. 
He studied Metaphysical poets, John Donne, and he learned Italian by reading Dante. 
In 1910 Eliot went to Paris where he attended Bergson's lectures and was exposed to the works of French Symbolists. 
As the First World War broke out, Eliot stayed in London and began working as a clerk in a bank. 
In 1915 he married Vivienne Haigh-Wood, despite his parents' worries about her mental instability. 
His first important work was the collection of poems \emph{Prufrock and Other Observations}. 
He then founded \emph{The Criterion}, a literary magazine, and later became director for the publishers \emph{Faber \& Faber}. 
During this time his wife's mental health deteriorated and Eliot himself was under considerable emotional strain. 
He therefore spent some time in a sanatorium in Switzerland and poetry became his refuge from the world. 
In his poems, he expresses the crisis of Western culture (cfr Nietzsche). 
In 1922 he published his masterpiece \emph{The Waste Land}. 
In 1927 Eliot became a citizen of the United Kingdom and in the same year he joined the Church of England. 
In this period he wrote some religious poems. 
Eventually Eliot decided to separate from his wife, who died in an asylum in 1947. 
Guilt haunted Eliot for her death. 
In those years, the poet had got closer to theatre, becoming one of the major exponents of poetic drama. 
In 1948 he won the Nobel Prize for Literature. 
He died in London and his ashes are buried in St Michael and All Angels Church in East Coker.

Before his conversion to Anglicanism his works were characterized by a pessimistic view of the world, while purification, hope, and joy are found after his conversion. 

\subsubsection{\emph{The Waste Land}}
The poem consists of five sectionn:
\begin{itemize}
    \item \textbf{The Burial of the Dead}, dealing with the coming of spring in a barren land; 
    \item \textbf{A Game of Chess}, which compares the present squalor to and ambiguous past splendor; 
    \item \textbf{The Fire Sermon}, where the theme of alienation is explored through a loveless sexual encounter; 
    \item \textbf{Death by Water}, about a drowned Phoenician sailor, Phlebas; 
    \item \textbf{What the Thunder Said}, which evokes religios from all over the world but ends in utter desolation. 
\end{itemize}

It deals with the contrast between the fertility of a mythical past and the present spiritual sterility. 
The fragmentation of the poem reflects the breakdown of the social and cultural order caused by the First World War. 
The present and the past exist simultaneously in \emph{The Waste Land}. 
Eliot used the technique of the \emph{object correlative} (cfr Montale), using a combination of objects to evoke the appropriate emotion. 

\subsubsection{\emph{The Love Song of J. Alfred Prufrock}}

\subsection{James Joyce (1882 - 1941)}

\subsubsection{Life and works}
James was born in Dublin and was educated at Jesuits schools before attending University College, where he gained a Bacelor of Arts with a focus on modern languages. 
He was interested in a broad European culture, leading him to think of himself as a European rather than an Irishman. 
Joyce believed that the only way to raise Ireland's awareness was to offer a realistic portrait of it from a cosmopolitan point of view. 
On the 16\textsuperscript{th} of June 1904 he had his first date with his would-be wife, Nora Barnacle. 
The following year the couple moved to Trieste, where Joyce began teaching English and befriended Italo Svevo. 
While in Trieste, he finished writing his masterpiece \emph{Dubliners} and published it in 1914. 
In 1915 Joyce and his family moved to Zurich to flee from the First World War. 
In 1920 they moved again to Paris, where Joyce was able to publish his novel \emph{Ulysses}. 
This work drew both high praise and strong criticism. 
In 1940 Joyce and his family moved back to Zurich as the nazis occupied France. 
He died as a result of an intestinal operation and is buried in Zurich. 

Joyce wanted to give a realistic portrait of the life of ordinary people in his work. 
He did so while challenging Catholicism and traditions. 
His style was full of symbols, free direct speech, and streams of consciousness. 

\subsubsection{\emph{Dubliners}}
\emph{Dubliners} consists of 15 short stories organized into 4 nuclei:
\begin{itemize}
    \item childhood;
    \item adolescence;
    \item maturity;
    \item public life;
\end{itemize}
The Dublin he describes is a place where true feelings and compassion for others do not exist, as city life degrades its citizens. 
Everyone in Dublin is oppressed by religious, political, cultural, and economic forces and everyone is in a state of \emph{paralysis}. 
The description of each story is realistic but full of symbolism. 
Joyce makes heavy use of the \emph{epiphany}, a trivial moment in one's life that causes a sudden revelation. 

\subsubsection{\emph{Ulysses}}

\subsection{Virginia Woolf (1882 - 1941)}

\subsubsection{Life and works}
Virginia Stephen was born in London, daughter of an eminent Victorian intellectual. 
Her education consisted in private Greek lessons and, most important, access to her father's library. 
She spent her summers in Cornwall and the sea remained a central theme in her art (cfr Montale). 
Her mother's death led to Virginia's first nervous breakdown and to her rebelling against her father's being aggressive and tyrannical. 
When her father died in 1904, Virginia felt free to pursue her own literary career. 
She became a member of the Bloomsbury Group, an important literary avant-garde that disdained Victorian traditions. 
In 1912 Virginia married Leonard Woolf and in 1915 she published her first novel, \emph{The Voyage Out}. 
During this period she attempted suicide by taking drugs. 
In 1925 she published \emph{Mrs Dalloway}, a novel in which she experimented with new narrative techniques. 
Woolf was also a talented literary critic and she published a volume of essays named \emph{The Common Reader}. 
She was also devoted to the feminist movement, delivering two lectures which then became \emph{A Room of One's Own}. 
As World War II broke out, her anxiety increased and she became haunted by the fear of losing her mind. 
She drowned herself in the River Ouse in 1941. 

Woolf was interested in giving voice to the inner world of her characters. 
What mattered for her were not the events that made up a story, but the impressions they left on who experienced them. 
In her novels, the point of view shifted between the different characters' minds. 

\subsubsection{\emph{Mrs Dalloway}}
In the morning, Clarissa Dalloway goes to buy some flowers for a party she is giving that same evening. 
Her attention shifts toward Septimus and Lucrezia Warren Smith, a war veteran suffering from shell shock and an Italian girl walking on the street. 
Septimus is affected by mental disorder. 
Clarissa goes back home and she is visited by Peter Walsh, a man she loved in her youth. 
Peter then goes to Regent's Park, where he sees Septimus and Lucrezia going to see a doctor. 
In the evening, Septimus jumps out of his window. 
Everyone is present at Clarissa's party. 
Clarissa hears of Septimus's death and she feels a strong connection with him. 

The novel, much like Joyce's \emph{Ulysses}, takes place on an ordinary day in June. 
\emph{Mrs Dalloway} is set in a small area of London. 
Through what she defines as \say{tunneling technique}, Woolf allows the reader to access her characters' recollection of their past, providing background and personal history 

The characters all belong to the upper-middle class. 
Clarissa is married to Richard Dalloway, a conservative Member of Parliament. 
She is characterized by opposing feelings: a need for freedom and independence and a class consciousness. 
Septimus Warren Smith is a young poet and an extremely sensitive man. 
He is prone to panicking and haunted by guilt for the death of his best friend Evans during World War I. 

The novel deals with the way people react to new situations and it provides insights on social changes at the time. 
Woolf makes use of cinematic devices such as close-ups, flashbacks, and tracking shots. 
She also adopts a motif: the striking of Big Ben and clocks in general, representing the passing of time and its flowing into death. 

Woolf makes heavy use of streams of consciousness, but unlike Joyce, she maintains a logical and grammatical structure. 
She also describes \emph{moments of being}, rare occasions of insight similar to Joyce's epiphanies. 

\subsection{George Orwell (1903 - 1950)}

\subsubsection{Life and works}
George Orwell was born Eric Blair in India, son of a colonial officer. 
When he was little, he was taken to England by his mother. 
He was educated at Eton College, where he could not stand the lack of privacy and the humiliating punishments. 
He grew indifferent to socially accepted values, developing an independent-minded personality. 
He was also an atheist and socialist. 
He became a member of the Indian Imperial Police but in 1927 he went on leave and never returned to \say{escape from every form of man's dominion over man}. 
Back in London, he started a social experiment to experience poverty first-hand. 
He began publishing his works in 1933, starting with \emph{Down and Out in Paris and London}, about his experiences among the poor, and \emph{Burmese Days}, about his service as a colonial officer. 
He married Eileen O'Shaughnessy in 1936, a like-minded socialist and intellectual. 
In the same year he investigated the living conditions to which miners were subject and wrote the report \emph{The Road to Wigan Pier}. 
In December 1936 he moved to Catalonia to report on the Spanish Civil War and fought in the trenches of the Aragon front with the Socialist militia. 
Back in England, George and his wife adopted a child and called him Richard.  
When World War II broke out, Orwell moved to London and joined the BBC as a speaker. 
In 1943 he resigned and became the editor of \emph{The Tribune}, a socialist weekly magazine. 
He also began writing \emph{Animal Farm} (published 1945) and \emph{Nineteen Eighty-four} (published 1949). 
Orwell died of tubercolosis in 1950. 

Thanks to his experiences abroad, Orwell was able to see his country from the outside and give a realistic account of its strenghts and weaknesses. 
He was also receptive to new ideas and impressions. 
He believed that a writer should be completely independet and that writing intepreted reality and hence served a useful social function. 
Orwell was inspired by Charles Dickens to write about social themes and to use a simple, factual language. 
He insisted on tolerance and justice and warned against the artificiality of urban civilization.

\subsubsection{\emph{Nineteen Eighty-four}}
The world is divided into three blocks: Oceania, Eurasia and Eastasia. 
Oceania is ruled by the Party, led by \emph{Big Brother}. 
The Party is creating a new language, \emph{Newspeak}, in order to control what people can say and think, also with the aid of \emph{Though Police}. 
Winstom Smith illegaly buys a diary and starts to write his thoughts and memories on it. 
His job is to rewrite history to bend it at the Party's will. 
At the \emph{Ministry of Truth} he meets an attractive girl called Julia. 
They begin a secret affair but are eventually discovered and arrested because of O'Brien, a Party spy. 
Winston is taken to the \emph{Ministry of Love}, where he is tortured ruthlessly for months. 
He has to face his biggest fear in \emph{room 101} and his will is completely broken. 
In the end, he meets Julia but he no longer loves her. 
He has given up on his identity and devoted himself completely to \emph{Big Brother}. 

The novel is set in a state of perpetual war akin to World War II. 
The idea for the three countries came to Orwell after the \emph{Tehran Conference} where Roosvelt, Churchill, and Stalin decided about the fate of the world after the war. 
The society reflects the political atmosphere present in totalitarism (cfr Arendt). 
Winston Smith represents a common Englishman. 
He is the last with a desire for spiritual and moral integrity. 
His main concern is the manipulation of history committed by the Party. 

\section{The Present Age}

\subsection{Samuel Beckett (1906 - 1989)}

\subsubsection{Life and works}
Samuel Beckett was born in Dublin, where he attended a boarding school and later Trinity College. 
After graduating in French and Italian, he moved to Paris where he taught English and came to know about existentialist philosophy. 
He also became friends with James Joyce and his literary circle. 
His works were mostly written in French and later traslated into English. 
He started writing short stories and novels and then settled onto theatrical plays. 
Beckett was part of a group of dramatists that developed the so-called \emph{Theatre of the Absurd}, in which man's life appears to be meaningless and purposeless. 
His play \emph{Waiting for Godot} (1952) was the first one in this style and it achieved immense success. 
He went on writing plays and he gained a Nobel Prize for Literature in 1969. 
He died in France in December 1989. 

\subsubsection{\emph{Waiting for Godot}}
Two tramps, Vladimir \say{Didi} and Estragon \say{Gogo}, meet by a leafless tree to wait for a misterious man called Godot. 
Godot sends a boy to inform them that he would not arrive that thay but will surely come the following day. 
The two bums are enraged and want to leave, but remain still. 
They also meet Pozzo and Lucky, a traveller and his slave. 
The following day, Didi and Gogo think about separation and even suicide but do not accomplish anything. 
The boy arrives to tell them that Godot cannot come, but he will surely come the following day. 
In the end, Vladimir and Estragon want to leave but they remain still as the scene fades to black. 

The play has no development in time: it has neither past nor future, only a meaningless and repetitive present. 
It has no setting either but a country road and a bare tree. 
What gives the play its unity is its symmetry: the stage is divided by the tree, humanity is split into two (Didi and Gogo). 
The characters' actions are also symmetrical. 

The characters are two humans concerned about the nature of the self, the world, and God. 
Vladimir is more practical, while Estragon is a dreamer. 
Estragon cannot remember anything about his past while Vladimir has some memories but he distrusts them. 
Both of them serve to remember the other of his own existence. 
Pozzo and Lucky are bonded to each other by a rope. 
Lucky is the slave and represents the power of the mind, while Pozzo is the oppressor and represents the power of the body. 

The language of the play is informal. 
The lack of communication is depicted through the use of para-verbal language such as pauses, silences, and gaps. 
Repeated phrases, lines, and words represent the senseless repetition and the relentless flow of time. 

\chapter{Fisica}
\section{Il magnetismo}

\subsection{Il campo magnetico}
Il campo magnetico è un campo vettoriale che descrive l'influenza magnetica su cariche in movimento e correnti elettriche. 
In un certo punto dello spazio, il campo magnetico ha:
\begin{itemize}
    \item come \textbf{direzione} quella lungo la quale si disporrebbe un ago magnetico libero di ruotare in un dato punto; 
    \item come \textbf{verso} quello che va dal polo sud al polo nord dell'ago.
\end{itemize}
Un punto di un filo percorso da corrente risente di una forza $\dd \vec{F}$ pari a:
\begin{equation}
    \label{forza magnetica}
    \dd \vec{F} = I \dd \vec{\ell} \times \vec{B}
\end{equation}
Dove $I$ è l'intensità di corrente che scorre nel filo, $\dd \vec{l}$ è l'intervallo infinitesimo di filo orientato nello stesso verso della corrente e $\vec{B}$ è il vettore campo magnetico. 
La forza totale che agisce sul filo $\gamma$ sarà quindi:
\begin{equation}
    \int_{\gamma} \dd \vec{F}
\end{equation}
Che, nel caso di un filo rettilineo immerso in un campo uniforme, può essere semplificata come:
\begin{equation}
    \vec{F} = I \vec{\ell} \times \vec{B}
\end{equation}
Dall'equazione \ref{forza magnetica}, possiamo inoltre ricavare il valore del modulo del campo magnetico in un determinato punto:
\begin{equation}
    \abs{\vec{B}} = \frac{\abs{\dd \vec{F}}}{I\abs{\dd \vec{\ell}} \sin{\alpha}}
\end{equation}
Dove $\alpha$ è l'angolo tra il vettore $\dd \vec{L}$ e il vettore $\dd \vec{B}$. L'unità di misura per il campo magnetico risulta:
\begin{equation}
    \left[\vec{B}\right] = \frac{N}{A \cdot m} = T
\end{equation}

\subsection{La forza di Lorentz}
Una particella carica immersa in un campo magnetico subisce una forza $\vec{F}$ pari a:
\begin{equation}
    \label{forza di Lorentz}
    \vec{F} = q \vec{v} \times \vec{B}
\end{equation}
Considerando i moduli invece avremo:
\begin{equation}
    F = \abs{q}vB\sin{\alpha}
\end{equation}
Dove $\alpha$ è l'angolo tra i vettori $\vec{v}$ e $\vec{B}$. 
Dalle equazioni precedenti possiamo dedurre che:
\begin{itemize}
    \item se $\vec{v} \parallel \vec{B}$ la carica non risente di alcuna forza (poiché $\vec{v} \times\vec{B} = 0$); 
    \item se $\vec{v} \perp \vec{B}$ la carica si muove di moto circolare uniforme; 
    \item se $\vec{v}$ e $\vec{B}$ sono incidenti la carica si muove di moto elicoidale. 
\end{itemize}
Analizziamo meglio gli ultimi due casi. 
Quando $\vec{v} \perp \vec{B}$, $\vec{F} \perp \vec{v}$, pertanto $\abs{\vec{v}}$ è costante. 
La carica si muove quindi di moto circolare uniforme e la forza di Lorentz è la forza centripeta. 
Poiché $\alpha = 90 \degree$, $\sin{\alpha} = 1$ e possiamo quindi affermare che: 
\begin{equation}
    F = \abs{q}vB = \frac{mv^2}{r}
\end{equation}
Da cui possiamo ricavare il raggio della traiettoria circolare che la carica seguirà:
\begin{equation}
    r = \frac{mv}{\abs{q}B}
\end{equation}
Da cui possiamo poi ricavare altre misure come la velocità angolare $\omega$ e la frequenza $f$.
Qualora $\vec{v}$ sia incidente a $\vec{B}$, possiamo scomporre $\vec{v}$ in $v_{\parallel} = v \cos{\alpha}$ e $v_{\perp} = v \sin{\alpha}$. 
Il procedimento è poi analogo a quanto riportato sopra. 
Possiamo inoltre calcolare il passo dell'elica, ovvero la distanza tra due \say{spire} consecutive. 

\subsection{Il ciclotrone}
Il ciclotrone è un macchinario utilizzato per accelerare particelle. 
Fu inventato da Ernest Orlando Lawrence nel 1932 e viene sfruttato ancora oggi per accelerare ioni leggeri. 
Due conduttori semicilindrici cavi (in gergo chiamati D) sono sottoposti a un campo magnetico uniforme (perpendicolare alle basi delle D). 
Le due D sono collegate a un alternatore che crea un campo elettrico tra loro. 
Nella zona centrale, una sorgente emette particelle cariche che vengono messe in moto dal campo elettrico. 

Se si sincronizza il moto delle particelle con la frequenza della tensione alternata, ogni volta che la particella torna nella zona centrale incontra un campo favorevole che la accelera ulteriormente. 
Perché questo avvenga è necessario che il tempo $\Delta t$ sia: 
\begin{equation}
    \Delta t = \frac{T}{2} = \frac{\pi m}{\abs{q} B}
\end{equation}
Questo è indipendente dalla velocità, quindi la particella mantiene la stessa frequenza nonostante cambi la sua energia cinetica. 
La particella e la tensione alternata resteranno quindi sincronizzate purché quest'ultima sia: 
\begin{equation}
f = \frac{\abs{q} B}{2\pi m}
\end{equation}
Dal macchinario esce quindi un fascio di ioni ad alta energia utilizzati per ricerca o per fini medici come la cura di tumori. 

\begin{figure}[ht]
    \centering
    \includegraphics[width=\linewidth]{cyclotron.jpg}
    \caption{Il funzionamento schematico di un ciclotrone}
\end{figure}

\subsection{Il selettore di velocità}
Questo macchinario sfrutta la sovrapposizione del campo magnetico e del campo elettrico in una stessa regione di spazio per ottenere un fascio di particelle alla stessa velocità. 
Per costruire tale macchinario consideriamo la forza a cui è soggetta una carica immersa in un campo elettrico e magnetico: 
\begin{equation}
    \vec{F} = q \vec{E} + q \vec{v} \times \vec{B} = q(\vec{E} + \vec{v} \times \vec{B})
\end{equation}
Perché una particella carica ($q \neq 0$) non risulti né deflessa né accelerata, è necessario che $\vec{F} = 0$. 
Questo può succedere solo quando $\vec{E} + \vec{v} \times \vec{B} = 0$ ovvero:
\begin{equation}
    \vec{E} \perp \vec{B} \\
\end{equation}
\begin{equation}
    v = \frac{E}{B}
\end{equation}

\newpage

\subsection{Lo spettrometro di massa}

\begin{wrapfigure}{l}{0.5\textwidth}
    \includegraphics[width=0.5\textwidth]{spettrometro}
\end{wrapfigure}

Lo spettrometro di massa è uno strumento utilizzato per misurare la massa degli ioni. 
Il campione da esaminare viene innanzitutto ionizzato. 
Per semplicità, consideriamo la prima ionizzazione, ovvero le particelle avranno carica $q=e$. 
Gli ioni prodotti vengono accelerati con un campo elettrico e successivamente vengono fatti passare in un selettore di velocità. 
Successivamente, gli ioni entrano in una regione dove è presente un campo magnetico uniforme che devia le particelle lungo traiettorie circolari. 
Il raggio di tali traiettorie, come abbiamo osservato precedentemente, è:
\begin{equation}
    r = \frac{v}{eB} m
\end{equation}
È quindi possibile separare ioni di massa diversa a seconda del raggio ($r \propto m$). 

\section{L'elettromagnetismo}

\subsection{Le leggi di Maxwell}
Lo studio dell'elettromagnetismo si basa sulle quattro equazioni di Maxwell.
\begin{gather}
    \label{gauss elettrico} \Phi_{S_C}(\vec{E}) = \frac{Q}{\epsilon_0} \\
    \label{gauss magnetico} \Phi_{S_C}(\vec{B}) = 0 \\
    \label{circuitazione elettrico} C_{\gamma}(\vec{E}) = -\diff{\Phi_{\gamma}(\vec{B})}{t} \\
    \label{circuitazione magnetico} C_{\gamma}(\vec{B}) = \mu_0\left(\sum I_k + \epsilon_0 \diff{\Phi_S(\vec{E})}{t}\right)
\end{gather}
Analizziamo una per una cosa significano. 

La legge di Gauss per il campo elettrico (\ref{gauss elettrico}), afferma che il flusso del campo elettrico uscente da una superficie chiusa $S_C$ è direttamente proporzionale alla carica $Q$ racchiusa al suo interno. 
Da questa legge si deduce che le linee di campo elettrico possono essere aperte e quindi possiamo affermare che esistono \say{pozzi} e \say{sorgenti} di campo elettrico. 

La legge di Gauss per il campo magnetico (\ref{gauss magnetico}) afferma che il flusso del campo magnetico uscente da una superficie chiusa $S_C$ è sempre nullo. 
Questa legge ci dice che non esistono monopoli magnetici e che le linee di campo magnetico sono sempre chiuse: non possono esistere né \say{pozzi} né \say{sorgenti}. 

La circuitazione del campo elettrico (\ref{circuitazione elettrico}) lungo una linea chiusa e orientata $\gamma$ è direttamente proporzionale alla derivata del flusso del campo magnetico concatenato con $\gamma$. 
Da questa possiamo dedurre che il campo elettrostatico è conservativo poiché $C_{\gamma}(\vec{E}_{statico}) = 0$, mentre il campo elettrico indotto no ($C_{\gamma}(\vec{E}_{indotto}) \neq 0$). 

La circuitazione del campo magnetico (\ref{circuitazione magnetico}) è direttamente proporzionale alla somma tra correnti conduzione e correnti di spostamento. 
Questa legge comporta che, così come il campo elettrico può essere generato da un campo magnetico variabile, anche il campo magnetico può essere generato da un campo elettrico variabile. 

\subsection{Le onde elettromagnetiche}
\begin{figure}[h]
    \centering
    \includegraphics[width=\textwidth]{onda}
    \caption{Un'onda elettromagnetica piana sinusoidale}
\end{figure}
Un'importante previsione delle leggi di Maxwell fu l'esistenza delle onde elettromagnetiche. 
Queste onde hanno alcune caratteristiche particolari:
\begin{itemize}
    \item possono propagarsi anche nel vuoto;
    \item sono onde trasversali, ovvero i campi elettrico e magnetico vibrano perpendicolarmente alla direzione di propagazione dell'onda;
    \item il campo elettrico $\vec{E}$ e quello magnetico $\vec{B}$ sono sempre perpendicolari;
    \item il verso di propagazione dell'onda è quello del prodotto vettoriale $\vec{E} \times \vec{B}$;
    \item $\vec{E}$ e $\vec{B}$ oscillano in fase, $E = c B$;
    \item la velocità delle onde elettromagnetiche nel vuoto è $c = \frac{1}{\sqrt{\epsilon_0 \mu_0}} = 299\,792\,458 m/s$;
\end{itemize}

%\chapter{Scienze}
%\input{materie/scienze.tex}

\end{document}  
